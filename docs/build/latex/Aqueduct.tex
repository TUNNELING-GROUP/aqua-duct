% Generated by Sphinx.
\def\sphinxdocclass{report}
\documentclass[a4paper,10pt,english]{sphinxmanual}
\usepackage[utf8]{inputenc}
\DeclareUnicodeCharacter{00A0}{\nobreakspace}
\usepackage{cmap}
\usepackage[T1]{fontenc}
\usepackage{babel}
\usepackage{times}
\usepackage[Bjarne]{fncychap}
\usepackage{longtable}
\usepackage{sphinx}
\usepackage{multirow}


\title{Aqueduct Documentation}
\date{June 13, 2016}
\release{0.2.9}
\author{Tomasz Magdziarz}
\newcommand{\sphinxlogo}{}
\renewcommand{\releasename}{Release}
\makeindex

\makeatletter
\def\PYG@reset{\let\PYG@it=\relax \let\PYG@bf=\relax%
    \let\PYG@ul=\relax \let\PYG@tc=\relax%
    \let\PYG@bc=\relax \let\PYG@ff=\relax}
\def\PYG@tok#1{\csname PYG@tok@#1\endcsname}
\def\PYG@toks#1+{\ifx\relax#1\empty\else%
    \PYG@tok{#1}\expandafter\PYG@toks\fi}
\def\PYG@do#1{\PYG@bc{\PYG@tc{\PYG@ul{%
    \PYG@it{\PYG@bf{\PYG@ff{#1}}}}}}}
\def\PYG#1#2{\PYG@reset\PYG@toks#1+\relax+\PYG@do{#2}}

\expandafter\def\csname PYG@tok@gd\endcsname{\def\PYG@tc##1{\textcolor[rgb]{0.63,0.00,0.00}{##1}}}
\expandafter\def\csname PYG@tok@gu\endcsname{\let\PYG@bf=\textbf\def\PYG@tc##1{\textcolor[rgb]{0.50,0.00,0.50}{##1}}}
\expandafter\def\csname PYG@tok@gt\endcsname{\def\PYG@tc##1{\textcolor[rgb]{0.00,0.27,0.87}{##1}}}
\expandafter\def\csname PYG@tok@gs\endcsname{\let\PYG@bf=\textbf}
\expandafter\def\csname PYG@tok@gr\endcsname{\def\PYG@tc##1{\textcolor[rgb]{1.00,0.00,0.00}{##1}}}
\expandafter\def\csname PYG@tok@cm\endcsname{\let\PYG@it=\textit\def\PYG@tc##1{\textcolor[rgb]{0.25,0.50,0.56}{##1}}}
\expandafter\def\csname PYG@tok@vg\endcsname{\def\PYG@tc##1{\textcolor[rgb]{0.73,0.38,0.84}{##1}}}
\expandafter\def\csname PYG@tok@m\endcsname{\def\PYG@tc##1{\textcolor[rgb]{0.13,0.50,0.31}{##1}}}
\expandafter\def\csname PYG@tok@mh\endcsname{\def\PYG@tc##1{\textcolor[rgb]{0.13,0.50,0.31}{##1}}}
\expandafter\def\csname PYG@tok@cs\endcsname{\def\PYG@tc##1{\textcolor[rgb]{0.25,0.50,0.56}{##1}}\def\PYG@bc##1{\setlength{\fboxsep}{0pt}\colorbox[rgb]{1.00,0.94,0.94}{\strut ##1}}}
\expandafter\def\csname PYG@tok@ge\endcsname{\let\PYG@it=\textit}
\expandafter\def\csname PYG@tok@vc\endcsname{\def\PYG@tc##1{\textcolor[rgb]{0.73,0.38,0.84}{##1}}}
\expandafter\def\csname PYG@tok@il\endcsname{\def\PYG@tc##1{\textcolor[rgb]{0.13,0.50,0.31}{##1}}}
\expandafter\def\csname PYG@tok@go\endcsname{\def\PYG@tc##1{\textcolor[rgb]{0.20,0.20,0.20}{##1}}}
\expandafter\def\csname PYG@tok@cp\endcsname{\def\PYG@tc##1{\textcolor[rgb]{0.00,0.44,0.13}{##1}}}
\expandafter\def\csname PYG@tok@gi\endcsname{\def\PYG@tc##1{\textcolor[rgb]{0.00,0.63,0.00}{##1}}}
\expandafter\def\csname PYG@tok@gh\endcsname{\let\PYG@bf=\textbf\def\PYG@tc##1{\textcolor[rgb]{0.00,0.00,0.50}{##1}}}
\expandafter\def\csname PYG@tok@ni\endcsname{\let\PYG@bf=\textbf\def\PYG@tc##1{\textcolor[rgb]{0.84,0.33,0.22}{##1}}}
\expandafter\def\csname PYG@tok@nl\endcsname{\let\PYG@bf=\textbf\def\PYG@tc##1{\textcolor[rgb]{0.00,0.13,0.44}{##1}}}
\expandafter\def\csname PYG@tok@nn\endcsname{\let\PYG@bf=\textbf\def\PYG@tc##1{\textcolor[rgb]{0.05,0.52,0.71}{##1}}}
\expandafter\def\csname PYG@tok@no\endcsname{\def\PYG@tc##1{\textcolor[rgb]{0.38,0.68,0.84}{##1}}}
\expandafter\def\csname PYG@tok@na\endcsname{\def\PYG@tc##1{\textcolor[rgb]{0.25,0.44,0.63}{##1}}}
\expandafter\def\csname PYG@tok@nb\endcsname{\def\PYG@tc##1{\textcolor[rgb]{0.00,0.44,0.13}{##1}}}
\expandafter\def\csname PYG@tok@nc\endcsname{\let\PYG@bf=\textbf\def\PYG@tc##1{\textcolor[rgb]{0.05,0.52,0.71}{##1}}}
\expandafter\def\csname PYG@tok@nd\endcsname{\let\PYG@bf=\textbf\def\PYG@tc##1{\textcolor[rgb]{0.33,0.33,0.33}{##1}}}
\expandafter\def\csname PYG@tok@ne\endcsname{\def\PYG@tc##1{\textcolor[rgb]{0.00,0.44,0.13}{##1}}}
\expandafter\def\csname PYG@tok@nf\endcsname{\def\PYG@tc##1{\textcolor[rgb]{0.02,0.16,0.49}{##1}}}
\expandafter\def\csname PYG@tok@si\endcsname{\let\PYG@it=\textit\def\PYG@tc##1{\textcolor[rgb]{0.44,0.63,0.82}{##1}}}
\expandafter\def\csname PYG@tok@s2\endcsname{\def\PYG@tc##1{\textcolor[rgb]{0.25,0.44,0.63}{##1}}}
\expandafter\def\csname PYG@tok@vi\endcsname{\def\PYG@tc##1{\textcolor[rgb]{0.73,0.38,0.84}{##1}}}
\expandafter\def\csname PYG@tok@nt\endcsname{\let\PYG@bf=\textbf\def\PYG@tc##1{\textcolor[rgb]{0.02,0.16,0.45}{##1}}}
\expandafter\def\csname PYG@tok@nv\endcsname{\def\PYG@tc##1{\textcolor[rgb]{0.73,0.38,0.84}{##1}}}
\expandafter\def\csname PYG@tok@s1\endcsname{\def\PYG@tc##1{\textcolor[rgb]{0.25,0.44,0.63}{##1}}}
\expandafter\def\csname PYG@tok@gp\endcsname{\let\PYG@bf=\textbf\def\PYG@tc##1{\textcolor[rgb]{0.78,0.36,0.04}{##1}}}
\expandafter\def\csname PYG@tok@sh\endcsname{\def\PYG@tc##1{\textcolor[rgb]{0.25,0.44,0.63}{##1}}}
\expandafter\def\csname PYG@tok@ow\endcsname{\let\PYG@bf=\textbf\def\PYG@tc##1{\textcolor[rgb]{0.00,0.44,0.13}{##1}}}
\expandafter\def\csname PYG@tok@sx\endcsname{\def\PYG@tc##1{\textcolor[rgb]{0.78,0.36,0.04}{##1}}}
\expandafter\def\csname PYG@tok@bp\endcsname{\def\PYG@tc##1{\textcolor[rgb]{0.00,0.44,0.13}{##1}}}
\expandafter\def\csname PYG@tok@c1\endcsname{\let\PYG@it=\textit\def\PYG@tc##1{\textcolor[rgb]{0.25,0.50,0.56}{##1}}}
\expandafter\def\csname PYG@tok@kc\endcsname{\let\PYG@bf=\textbf\def\PYG@tc##1{\textcolor[rgb]{0.00,0.44,0.13}{##1}}}
\expandafter\def\csname PYG@tok@c\endcsname{\let\PYG@it=\textit\def\PYG@tc##1{\textcolor[rgb]{0.25,0.50,0.56}{##1}}}
\expandafter\def\csname PYG@tok@mf\endcsname{\def\PYG@tc##1{\textcolor[rgb]{0.13,0.50,0.31}{##1}}}
\expandafter\def\csname PYG@tok@err\endcsname{\def\PYG@bc##1{\setlength{\fboxsep}{0pt}\fcolorbox[rgb]{1.00,0.00,0.00}{1,1,1}{\strut ##1}}}
\expandafter\def\csname PYG@tok@kd\endcsname{\let\PYG@bf=\textbf\def\PYG@tc##1{\textcolor[rgb]{0.00,0.44,0.13}{##1}}}
\expandafter\def\csname PYG@tok@ss\endcsname{\def\PYG@tc##1{\textcolor[rgb]{0.32,0.47,0.09}{##1}}}
\expandafter\def\csname PYG@tok@sr\endcsname{\def\PYG@tc##1{\textcolor[rgb]{0.14,0.33,0.53}{##1}}}
\expandafter\def\csname PYG@tok@mo\endcsname{\def\PYG@tc##1{\textcolor[rgb]{0.13,0.50,0.31}{##1}}}
\expandafter\def\csname PYG@tok@mi\endcsname{\def\PYG@tc##1{\textcolor[rgb]{0.13,0.50,0.31}{##1}}}
\expandafter\def\csname PYG@tok@kn\endcsname{\let\PYG@bf=\textbf\def\PYG@tc##1{\textcolor[rgb]{0.00,0.44,0.13}{##1}}}
\expandafter\def\csname PYG@tok@o\endcsname{\def\PYG@tc##1{\textcolor[rgb]{0.40,0.40,0.40}{##1}}}
\expandafter\def\csname PYG@tok@kr\endcsname{\let\PYG@bf=\textbf\def\PYG@tc##1{\textcolor[rgb]{0.00,0.44,0.13}{##1}}}
\expandafter\def\csname PYG@tok@s\endcsname{\def\PYG@tc##1{\textcolor[rgb]{0.25,0.44,0.63}{##1}}}
\expandafter\def\csname PYG@tok@kp\endcsname{\def\PYG@tc##1{\textcolor[rgb]{0.00,0.44,0.13}{##1}}}
\expandafter\def\csname PYG@tok@w\endcsname{\def\PYG@tc##1{\textcolor[rgb]{0.73,0.73,0.73}{##1}}}
\expandafter\def\csname PYG@tok@kt\endcsname{\def\PYG@tc##1{\textcolor[rgb]{0.56,0.13,0.00}{##1}}}
\expandafter\def\csname PYG@tok@sc\endcsname{\def\PYG@tc##1{\textcolor[rgb]{0.25,0.44,0.63}{##1}}}
\expandafter\def\csname PYG@tok@sb\endcsname{\def\PYG@tc##1{\textcolor[rgb]{0.25,0.44,0.63}{##1}}}
\expandafter\def\csname PYG@tok@k\endcsname{\let\PYG@bf=\textbf\def\PYG@tc##1{\textcolor[rgb]{0.00,0.44,0.13}{##1}}}
\expandafter\def\csname PYG@tok@se\endcsname{\let\PYG@bf=\textbf\def\PYG@tc##1{\textcolor[rgb]{0.25,0.44,0.63}{##1}}}
\expandafter\def\csname PYG@tok@sd\endcsname{\let\PYG@it=\textit\def\PYG@tc##1{\textcolor[rgb]{0.25,0.44,0.63}{##1}}}

\def\PYGZbs{\char`\\}
\def\PYGZus{\char`\_}
\def\PYGZob{\char`\{}
\def\PYGZcb{\char`\}}
\def\PYGZca{\char`\^}
\def\PYGZam{\char`\&}
\def\PYGZlt{\char`\<}
\def\PYGZgt{\char`\>}
\def\PYGZsh{\char`\#}
\def\PYGZpc{\char`\%}
\def\PYGZdl{\char`\$}
\def\PYGZhy{\char`\-}
\def\PYGZsq{\char`\'}
\def\PYGZdq{\char`\"}
\def\PYGZti{\char`\~}
% for compatibility with earlier versions
\def\PYGZat{@}
\def\PYGZlb{[}
\def\PYGZrb{]}
\makeatother

\begin{document}

\maketitle
\tableofcontents
\phantomsection\label{index::doc}


Contents:


\chapter{Aqueduct installation guide}
\label{aqueduct_install:welcome-to-aqueduct-s-documentation}\label{aqueduct_install::doc}\label{aqueduct_install:aqueduct-installation-guide}

\section{Overview}
\label{aqueduct_install:overview}
This package comprises of two elements:
\begin{enumerate}
\item {} 
aqueduct,

\item {} 
valve.

\end{enumerate}

Aqueduct is a Python module. It is a collection of tools to trace residues in MD simulation. Valve is a driver Python script. It uses aqueduct to perform such a tracing.


\section{Install}
\label{aqueduct_install:install}

\subsection{Aqueduct}
\label{aqueduct_install:aqueduct}
Installation was tested on limited number of POSIX-like systems.

In some specific cases installation is very simple:
\begin{enumerate}
\item {} 
Download \code{aqueduct.tar.gz} bundle file..

\item {} 
Unpack aqueduct bundle file.

\item {} 
Go to src directory.

\item {} 
Type:

\begin{Verbatim}[commandchars=\\\{\}]
python setup.py install
\end{Verbatim}

\end{enumerate}

Aqueduct requires several Python modules to work and in particular it requires MDAnalysis with AMBER support. This, on the other hand, requires netCDF4. Installation of this combination is sometimes cumbersome. General procedure is following:
\begin{enumerate}
\item {} 
Install libnetcdf4 and libhdf5 development libraries.

\item {} 
Install netCDF4:

\begin{Verbatim}[commandchars=\\\{\}]
pip install netCDF4
\end{Verbatim}

\item {} 
Try to install aqueduct.

\end{enumerate}

If, by chance, you are on Ubuntu 14.04 you can try helper script \code{ubuntu\_mdanalysis\_install\_helper.sh}.


\subsection{Valve}
\label{aqueduct_install:valve}
Valve does not need installation \emph{per se}. Once aqueduct is installed, valve can be run by a following command:

\begin{Verbatim}[commandchars=\\\{\}]
python valve.py \PYGZhy{}\PYGZhy{}help
\end{Verbatim}

Valve script, ie valve.py, is located in apps directory.


\subsection{Extras}
\label{aqueduct_install:extras}
Access to some visualization capabilities of Aqueduct requires additional Python modules:
\begin{enumerate}
\item {} 
matplotlib,

\item {} 
pymol.

\end{enumerate}

These are usually easy to install.


\chapter{\emph{Valve} manual}
\label{valve/valve_manual::doc}\label{valve/valve_manual:valve-manual}
\emph{Valve} application is a driver that uses {\hyperref[aqueduct:module-aqueduct]{\code{aqueduct}}} module to perform analysis of trajectories of selected residues in MD simulation.


\section{\emph{Valve} invocation}
\label{valve/valve_manual:valve-invocation}
Once {\hyperref[aqueduct:module-aqueduct]{\code{aqueduct}}} module is installed (see {\hyperref[aqueduct_install::doc]{\emph{Aqueduct installation guide}}}) properly on the machine \emph{Valve} is available as \code{valve.py} command line tool.


\subsection{Usage}
\label{valve/valve_manual:usage}
Basic help of \emph{Valve} usage can be displayed by following command:

\begin{Verbatim}[commandchars=\\\{\}]
\PYG{n}{valve}\PYG{o}{.}\PYG{n}{py} \PYG{o}{\PYGZhy{}}\PYG{o}{\PYGZhy{}}\PYG{n}{help}
\end{Verbatim}

It should display following information:

\begin{Verbatim}[commandchars=\\\{\}]
usage: valve.py [\PYGZhy{}h] [\PYGZhy{}\PYGZhy{}dump\PYGZhy{}template\PYGZhy{}config] [\PYGZhy{}t THREADS] [\PYGZhy{}c CONFIG\PYGZus{}FILE]
                [\PYGZhy{}\PYGZhy{}max\PYGZhy{}frame MAX\PYGZus{}FRAME]

Valve, Aqueduct driver

optional arguments:
  \PYGZhy{}h, \PYGZhy{}\PYGZhy{}help            show this help message and exit
  \PYGZhy{}\PYGZhy{}dump\PYGZhy{}template\PYGZhy{}config
                        Dumps template config file. Suppress all other output
                        or actions. (default: False)
  \PYGZhy{}t THREADS            Limit Aqueduct calculations to given number of
                        threads. (default: None)
  \PYGZhy{}c CONFIG\PYGZus{}FILE        Config file filename. (default: None)
  \PYGZhy{}\PYGZhy{}max\PYGZhy{}frame MAX\PYGZus{}FRAME
                        Limit number of frames. (default: None)
  \PYGZhy{}\PYGZhy{}version             Prints versions and exits.. (default: False)
\end{Verbatim}


\subsection{Configuration file template}
\label{valve/valve_manual:configuration-file-template}
Configuration file used by \emph{Valve} is of moderate length and complexity. It can be easily prepared with a template file that can be printed by \emph{Valve}. Use following command to print configuration file template on the screen:

\begin{Verbatim}[commandchars=\\\{\}]
\PYG{n}{valve}\PYG{o}{.}\PYG{n}{py} \PYG{o}{\PYGZhy{}}\PYG{o}{\PYGZhy{}}\PYG{n}{dump}\PYG{o}{\PYGZhy{}}\PYG{n}{template}\PYG{o}{\PYGZhy{}}\PYG{n}{config}
\end{Verbatim}

Configuration file template can also be easily saved in to a file with:

\begin{Verbatim}[commandchars=\\\{\}]
\PYG{n}{valve}\PYG{o}{.}\PYG{n}{py} \PYG{o}{\PYGZhy{}}\PYG{o}{\PYGZhy{}}\PYG{n}{dump}\PYG{o}{\PYGZhy{}}\PYG{n}{template}\PYG{o}{\PYGZhy{}}\PYG{n}{config} \PYG{o}{\PYGZgt{}} \PYG{n}{config}\PYG{o}{.}\PYG{n}{txt}
\end{Verbatim}

Where config.txt is a configuration file template.

For detailed description of configuration file and available options see {\hyperref[valve/valve_config::doc]{\emph{Configuration file options}}}


\subsection{\emph{Valve} calculation run}
\label{valve/valve_manual:valve-calculation-run}
Once configuration file is ready \emph{Valve} calculations can be run with a following simple command:

\begin{Verbatim}[commandchars=\\\{\}]
valve.py \PYGZhy{}c config.txt
\end{Verbatim}

Some of \emph{Valve} calculations can be run in parallel. By default all available CPU cores are used. This is not always desired - limitation of used CPU cores can be done with \code{-t} option which limits number of concurrent threads used by \emph{Valve}. If it equals 1 no parallelism is used.

\begin{notice}{note}{Note:}
Specifying number of threads greater then available CPU cores is generally not optimal.

However, in order to maximize usage of available CPU power it is recommended to set it as number of cores + 1. The reason is that \emph{Valve} uses one thread for the main process and the excess over one for processes for parallel calculations. When parallel calculations are executed the main threads waits for results.
\end{notice}

\begin{notice}{note}{Note:}
Option \code{-{-}max-frame} can be used for testing or debugging purposes. It allows to limit number of frames processed by \emph{Valve}.
If it set, for example, to \code{1000} only first 1000 frames will be processed making all calculations very fast.
\end{notice}


\section{How does \emph{Valve} work}
\label{valve/valve_manual:how-does-valve-work}
Application starts with parsing input options. If \code{-{-}help} or \code{-{-}dump-template-config} options are provided appropriate messages are printed on the screen and \emph{Valve} quits with signal \code{0}.

\begin{notice}{note}{Note:}
In current version \emph{Valve} does not check the validity of the config file.
\end{notice}

If config file is provided \emph{Valve} parse it quickly and regular calculations starts according to its content. Calculations performed by \emph{Valve} are done in several stages described in the next sections.


\subsection{Traceable residues}
\label{valve/valve_manual:traceable-residues}
The first stage finds all residues that should be traced and appends them to the list of \emph{traceable residues}. It is done in a loop over all frames. In each frame residues of interest are searched and appended to the list but only if they are not already present on the list.

The search of the residues is done according to user provided definitions.. Two requirements have to be met to append residue to the list:
\begin{enumerate}
\item {} 
The residue have to be found according to the \emph{Object} definition.

\item {} 
The residues have to be within the \emph{Scope} of interest.

\end{enumerate}

The \emph{Object} definition encompasses usually the active site of the protein. The \emph{Scope} of interest defines, on the other hand, the boundaries in which residues are traced and is usually defined as protein.

Since {\hyperref[aqueduct:module-aqueduct]{\code{aqueduct}}} in its current version uses \href{http://www.mdanalysis.org/}{MDAnalysis} Python module for reading, parsing and searching of MD trajectory data, definitions of \emph{Object} and \emph{Scope} have to be given as its \emph{Selection Commands}.


\subsubsection{Object definition}
\label{valve/valve_manual:object-definition}\label{valve/valve_manual:id1}
\emph{Object} definition have to comprise of two elements:
\begin{enumerate}
\item {} 
It have to define residues to trace.

\item {} 
It have to define spatial boundaries of the \emph{Object} site.

\end{enumerate}

For example, proper Object definition could be following:

\begin{Verbatim}[commandchars=\\\{\}]
(resname WAT) and (sphzone 6.0 (resnum 99 or resnum 147))
\end{Verbatim}

It defines \code{WAT} as residues that should be traced and defines spatial constrains of the \emph{Object} site as spherical zone within 6 Angstroms of the center of masses of residues with number 99 and 147.


\subsubsection{Scope definition}
\label{valve/valve_manual:scope-definition}\label{valve/valve_manual:id2}
\emph{Scope} can be defined in two ways: as \emph{Object} but with broader boundaries or as the convex hull of selected molecular object.

In the first case definition is very similar to \emph{Object} and it have to follow the same limitations. For example, proper \emph{Scope} definition could be following:

\begin{Verbatim}[commandchars=\\\{\}]
resname WAT around 2.0 protein
\end{Verbatim}

It consequently have to define \code{WAT} as residues of interest and defines spatial constrains as all \code{WAT} residues that are within 2 Angstroms of the protein.

If the \emph{Scope} is defined as the convex hull of selected molecular object (which is recommended), the definition itself have to comprise of this molecular object only, for example \code{protein}. In that case the scope is interpreted as the interior of the convex hull of atoms from the definition. Therefore, \emph{traceable residues} would be in the scope only if they are within the convex hull of atoms of \code{protein}.


\subsection{Raw paths}
\label{valve/valve_manual:raw-paths}
The second stage of calculations uses the list of all traceable residues from the first stage and finds coordinates of center of masses for each residue in each frame. As in the first stage, it is done in a loop over all frames. For each residue in each frame \emph{Valve} calculates or checks two things:
\begin{enumerate}
\item {} 
Is the residue in the \emph{Scope} (this is always calculated according to the Scope definition).

\item {} 
Is the residue in the \emph{Object}. This information is calculated in the first stage and can be reused in the second. However, it is also possible to recalculate this data according to the new \emph{Object} definition.

\end{enumerate}

For each of the \emph{traceable residues} a special \emph{Path} object is created. If the residue is in the \emph{Scope} its center of mass is added to the appropriate \emph{Path} object together with the information if it is in the \emph{Object} or not.


\subsection{Separate paths}
\label{valve/valve_manual:separate-paths}
The third stage uses collection of \emph{Path} objects to create \emph{Separate Path} objects. Each \emph{Path} comprise data for one residue. It may happen that the residue enters and leaves the \emph{Scope} and the \emph{Object} many times over the entire MD. Each such an event is considered by \emph{Valve} as a separate path.

Each \emph{separate path} comprises of three parts:
\begin{enumerate}
\item {} 
\emph{Incoming} - Defined as a path that leads from the point in which residue enters the \emph{Scope} and enters the object for the firs time.

\item {} 
\emph{Object} - Defined as a path that leads from the point in which residue enters the \emph{Object} for the first time and leaves it for the last time.

\item {} 
\emph{Outgoing} - Defined as a path that leads from the point in which residue leaves the \emph{Object} for the last lime and leaves the \emph{Scope}.

\end{enumerate}


\subsection{Clusterization of inlets}
\label{valve/valve_manual:clusterization-of-inlets}\label{valve/valve_manual:id3}
Each of the separate paths has beginning and end. If either of them are at the boundaries of the \emph{Scope} they are considered as \emph{Inlets}, i.e. points that mark where the \emph{traceable residues} enters or leaves the \emph{Scope}. Clusters of inlets, on the other hand, mark endings of tunnels or ways in the system which was simulated in the MD.

Clusterization of inlets is performed in following steps:
\begin{enumerate}
\item {} 
Initial clusterization. Depending on the method, some of the inlets might not be arranged to any cluster and are considered as outliers.

\item {} 
{[}Optional{]} Outliers detection. Arrangement of inlets to clusters is sometimes far from optimal. In this step, \emph{inlets} that do not fit to cluster are detected and annotated as outliers. This step can be executed in two modes:
\begin{enumerate}
\item {} 
Automatic mode. Inlet is considered to be an outlier if its distance from the centroid is greater then mean distance + 4 * standard deviation of all distances within the cluster.

\item {} 
Defined threshold. Inlet is considered to be an outlier if its minimal distance from any other point in the cluster is greater then the threshold.

\end{enumerate}

\item {} 
{[}Optional{]} Reclusterization of outliers. It may happen that the outliers form actually clusters but it was not recognized in initial clusterization. In this step clusterization is executed for outliers only and found clusters are appended to the clusters identified in the first step. Rest of the inlets are marked as outliers.

\end{enumerate}


\subsection{Analysis}
\label{valve/valve_manual:analysis}
Fifth stage of \emph{Valve} calculations analyses results calculated in stages 1 to 4. Results of the analysis is displayed on the screen or can be save to text file and comprise of following parts:
\begin{itemize}
\item {} 
Tile and data stamp.

\item {} 
{[}Optional{]} Dump of configuration options.

\item {} \begin{description}
\item[{Basic information on traceable residues and separate paths.}] \leavevmode\begin{itemize}
\item {} 
Number of traceable residues.

\item {} 
Number of separate paths.

\end{itemize}

\end{description}

\item {} \begin{description}
\item[{Basic information on inlets.}] \leavevmode\begin{itemize}
\item {} 
Number of inlets.

\item {} 
Number of clusters.

\item {} 
Are outliers detected.

\end{itemize}

\end{description}

\item {} \begin{description}
\item[{Summary of inlets clusters. Table with 5 columns:}] \leavevmode\begin{enumerate}
\item {} 
\textbf{Nr}: Row number, starting from 0.

\item {} 
\textbf{Cluster}: ID of the cluster. Outliers have 0.

\item {} 
\textbf{Size}: Size of the cluster.

\item {} 
\textbf{INCOMING}: Number of inlets corresponding to separate paths that enter the scope.

\item {} 
\textbf{OUTGOING}: Number of inlets corresponding to separate paths that leave the scope.

\end{enumerate}

\end{description}

\item {} \begin{description}
\item[{Summary of separate paths clusters types. Table with 9 columns.}] \leavevmode\begin{enumerate}
\item {} 
\textbf{Nr}: Row number, starting from 0.

\item {} 
\textbf{CType}: Separate path Cluster Type.

\item {} 
\textbf{Size}: Number of separate paths belonging to Cluster type.

\item {} 
\textbf{Inp}: Average length of incoming part of the path. If no incoming part is available it is nan.

\item {} 
\textbf{InpStd}: Standard deviation of length Inp.

\item {} 
\textbf{Obj}: Average length of object part of the path. If no incoming part is available it is nan.

\item {} 
\textbf{ObjStd}: Standard deviation of length Inp.

\item {} 
\textbf{Out}: Average length of outgoing part of the path. If no incoming part is available it is nan.

\item {} 
\textbf{OutStd}: Standard deviation of length Inp.

\end{enumerate}

\end{description}

\item {} \begin{description}
\item[{List of separate paths and their properties. Table with 17 columns.}] \leavevmode\begin{enumerate}
\item {} 
\textbf{Nr}: - Row number, starting from 0.

\item {} 
\textbf{ID}: - Separate path ID.

\item {} 
\textbf{BeginF}: Number of frame in which the path begins.

\item {} 
\textbf{InpF}: Number of frame in which path begins Incoming part.

\item {} 
\textbf{ObjF}: Number of frame in which path begins Object part.

\item {} 
\textbf{OutF}: Number of frame in which path begins Outgoing part.

\item {} 
\textbf{EndF}: Number of frame in which the path ends.

\item {} 
\textbf{InpL}: Length of Incoming part. If no incoming part nan is given.

\item {} 
\textbf{ObjL}: Length of Object part.

\item {} 
\textbf{OutL}: Length of Outgoing part. If no outgoing part nan is given.

\item {} 
\textbf{InpS}: Average step of Incoming part. If no incoming part nan is given.

\item {} 
\textbf{InpStdS}: Standard deviation of InpS.

\item {} 
\textbf{ObjS}: Average step of Object part.

\item {} 
\textbf{ObjStdS}: Standard deviation of ObjS.

\item {} 
\textbf{OutS}: Average step of Outgoing part. If no outgoing part nan is given.

\item {} 
\textbf{OutStdS}: Standard deviation of OutS.

\item {} 
\textbf{CType}: Cluster type of separate path.

\end{enumerate}

\end{description}

\end{itemize}


\subsubsection{Separate path ID}
\label{valve/valve_manual:separate-path-id}
Separate Paths IDs are composed of two numbers separated by colon. First number is the residue number. Second number is consecutive number of the separate path made by the residue. Numeration starts with 0.


\subsubsection{Cluster Type of separate path}
\label{valve/valve_manual:cluster-type-of-separate-path}
Each separate paths has two ends: beginning and end. Both of them either belong to one of the inlets clusters, or are among outliers, or are inside the scope. If an end belongs to one of the clusters (including outliers) it has ID of the cluster. If it is inside the scope it has special ID of \code{N}. Cluster type is an ID composed of IDs of both ends of separate path separated by colon charter.


\subsection{Visualization}
\label{valve/valve_manual:visualization}
Sixth stage of \emph{Valve} calculations visualizes results calculated in stages 1 to 4. Visualization is done with PyMOL. \emph{Valve} automatically starts PyMOL and loads visualizations in to it.
Molecule is loaded as PDB file. Other objects like Inlets clusters or paths are loaded as CGO objects.

Following is a list of objects created in PyMOL (all of them are optional). PyMOL object names given in \textbf{bold} text or short explanation is given.
\begin{itemize}
\item {} 
Selected frame of the simulated system. Object name: \emph{molecule}.

\item {} 
Inlets clusters, each cluster is a separate object. Object name: \textbf{cluster\_} followed by cluster annotation: otliers are annotated as Out; regular clusters by ID.

\item {} 
List of cluster types, raw paths. Each cluster type is a separate object. Object name composed of cluster type (colon replaced by underline) plus \textbf{\_raw}.

\item {} 
List of cluster types, smooth paths. Each cluster type is a separate object. Object name composed of cluster type (colon replaced by underline) plus \textbf{\_smooth}.

\item {} 
All raw paths. They can be displayed as one object or separated in to Incoming, Object and Outgoing part. Object name: \textbf{all\_raw}, or \textbf{all\_raw\_in}, \textbf{all\_raw\_obj}, and \textbf{all\_raw\_out}.

\item {} 
All raw paths inlets arrows. Object name: \textbf{all\_raw\_paths\_io}.

\item {} 
All smooth paths. They can be displayed as one object or separated in to Incoming, Object and Outgoing part. Object name: \textbf{all\_smooth}, or \textbf{all\_smooth\_in}, \textbf{all\_smooth\_obj}, and \textbf{all\_smooth\_out}.

\item {} 
All raw paths inlets arrows. Object name: \textbf{all\_raw\_paths\_io}.

\item {} 
Raw paths displayed as separate objects or as one object with several states. Object name: \textbf{raw\_paths\_} plus number of path or \textbf{raw\_paths} if displayed as one object.

\item {} 
Smooth paths displayed as separate objects or as one object with several states. Object name: \textbf{smooth\_paths\_} plus number of path or \textbf{smooth\_paths} if displayed as one object.

\item {} 
Raw paths arrows displayed as separate objects or as one object with several states. Object name: \textbf{raw\_paths\_io\_} plus number of path or \textbf{raw\_paths\_io} if displayed as one object.

\item {} 
Smooth paths arrows displayed as separate objects or as one object with several states. Object name: \textbf{smooth\_paths\_io\_} plus number of path or \textbf{smooth\_paths\_io} if displayed as one object.

\end{itemize}


\subsubsection{Color schemes}
\label{valve/valve_manual:color-schemes}
Inlets clusters are colored automatically. Outliers are gray.

Incoming parts of paths are red, Outgoing parts are blue. Object parts in case of smooth paths are green and in case of raw paths are green if residue is precisely in the object area or yellow if is leaved object area but it is not in the Outgoing part yet.

Arrows are colored in accordance to paths colors.


\chapter{Configuration file options}
\label{valve/valve_config::doc}\label{valve/valve_config:configuration-file-options}
Valve Configuration file is a simple and plain text file. It is similar to INI files commonly used in one of the popular operating systems and is compliant with Python module \href{http://docs.python.org/2/library/configparser.html\#module-ConfigParser}{\code{ConfigParser}}.

Configuration file comprises of several \emph{sections}. They can be grouped in to three categories. Names of sections given in \textbf{bold} text.
\begin{enumerate}
\item {} \begin{description}
\item[{Global settings:}] \leavevmode\begin{itemize}
\item {} 
\textbf{global}

\end{itemize}

\end{description}

\item {} \begin{description}
\item[{Stages options:}] \leavevmode\begin{itemize}
\item {} 
\textbf{traceable\_residues}

\item {} 
\textbf{raw\_paths}

\item {} 
\textbf{separate\_paths}

\item {} 
\textbf{inlets\_clusterization}

\item {} 
\textbf{analysis}

\item {} 
\textbf{visualize}

\end{itemize}

\end{description}

\item {} \begin{description}
\item[{Methods options:}] \leavevmode\begin{itemize}
\item {} 
\textbf{smooth}

\item {} 
\textbf{clusterization}

\item {} 
\textbf{reclusteriation}

\end{itemize}

\end{description}

\end{enumerate}


\section{Section \textbf{global}}
\label{valve/valve_config:section-global}
This section allows settings of trajectory data and progress bar type.


\subsection{Available options}
\label{valve/valve_config:available-options}\begin{itemize}
\item {} 
\code{top} - Path to Amber topology file.

\item {} 
\code{nc} - Path to Amber NetCDF file.

\item {} \begin{description}
\item[{\code{pbar} - Progres bar type. Possible values:}] \leavevmode\begin{itemize}
\item {} 
\code{simple} - {[}Default, Recommended{]} Build in progress bar.

\end{itemize}

\end{description}

\end{itemize}


\subsection{Example}
\label{valve/valve_config:example}
\begin{Verbatim}[commandchars=\\\{\}]
[global]
top = path/to/topology/file.prmtop
nc = path/to/trajectory/file.nc
pbar = simple
\end{Verbatim}


\section{Common settings of stage sections}
\label{valve/valve_config:common-settings-of-stage-sections}

\subsection{Option \textbf{execute}}
\label{valve/valve_config:option-execute}
All stage sections have \code{execute} option which decides if the stage is executed or skipped. There are two possible values of \code{execute} option: \code{run}, and \code{skip}.

If \code{execute} is set to \code{run} the stage is executed and results of calculations can be optionally saved.

If \code{execute} is set to \code{skip} execution of the stage is skipped. Results of calculations saved previously can be optionally loaded.


\subsection{Option \textbf{save}}
\label{valve/valve_config:option-save}
This options allows to save a dump of calculated data on the disk. If \code{execute} is set to \code{run} and \code{save} is set to file name results are saved as gziped pickled dump.

In case of \textbf{analysis} and \textbf{visualize} sections this setting has slightly different function. Results of \textbf{analysis} section as saved to the file pointed by \textbf{save} as plain text comprising of tables and summaries. Stage \textbf{visualize}, on the other hand, uses \textbf{save} option to save PyMOL session.


\subsection{Option \textbf{load}}
\label{valve/valve_config:option-load}
This options allows to read previously saved dump of results. It is used only if \code{ececute} is set to \code{skip} and is valid only for \textbf{traceable\_residues}, \textbf{raw\_paths}, \textbf{separate\_paths}, and \textbf{inlets\_clusterization} stages.


\section{Stage \textbf{traceable\_residues}}
\label{valve/valve_config:stage-traceable-residues}

\subsection{Option \textbf{scope}}
\label{valve/valve_config:option-scope}
Definition of \emph{Scope} of interest. See also {\hyperref[valve/valve_manual:scope-definition]{\emph{Scope definition}}}.

\begin{notice}{note}{Note:}
This option is mandatory.
\end{notice}


\subsection{Option \textbf{scope\_convexhull}}
\label{valve/valve_config:option-scope-convexhull}
Flag to set if the \emph{Scope} is direct or convex hull definition.


\subsection{Option \textbf{object}}
\label{valve/valve_config:option-object}
Definition of \emph{Object} of interest. See also {\hyperref[valve/valve_manual:object-definition]{\emph{Object definition}}}.

\begin{notice}{note}{Note:}
This option is mandatory.
\end{notice}


\section{Stage \textbf{raw\_paths}}
\label{valve/valve_config:stage-raw-paths}
This stage also requires definition of the \emph{Scope} and \emph{Object}. If appropriate settings are not given, settings from the previous stage are used.


\subsection{Option \textbf{clear\_in\_object\_info}}
\label{valve/valve_config:option-clear-in-object-info}
If it is set to \code{True} information on occupation of \emph{Object} site by traceable residues calculated in the previous stage are cleared and have to be recalculated. This is useful if definition of \emph{Object} is changed.


\section{Stage \textbf{separate\_paths}}
\label{valve/valve_config:stage-separate-paths}

\subsection{Option \textbf{discard\_empty\_paths}}
\label{valve/valve_config:option-discard-empty-paths}
If set to \code{True} empty paths are discarded.


\subsection{Option \textbf{sort\_by\_id}}
\label{valve/valve_config:option-sort-by-id}
If set to \code{True} separate paths are sorted by ID.


\subsection{Option \textbf{apply\_smoothing}}
\label{valve/valve_config:option-apply-smoothing}
If set to \code{True} smooth paths are precalculated according to \textbf{smooth} setting.
This speed up access to smooth paths in later stages but makes dump data much bigger.


\subsection{Option \textbf{apply\_soft\_smoothing}}
\label{valve/valve_config:option-apply-soft-smoothing}
If set to \code{True} raw paths are replaced by smooth paths calculated according to \textbf{smooth} section.


\subsection{Option \textbf{discard\_short\_paths}}
\label{valve/valve_config:option-discard-short-paths}
This option allows to discard paths that are shorter then the threshold.


\section{Stage \textbf{inlets\_clusterization}}
\label{valve/valve_config:stage-inlets-clusterization}

\subsection{Option \textbf{recluster\_outliers}}
\label{valve/valve_config:option-recluster-outliers}
If set to \code{True} reclusterization of outliers is executed according to the method defined in \textbf{reclusterization} section.


\subsection{Option \textbf{detect\_outliers}}
\label{valve/valve_config:option-detect-outliers}
If set detection of outliers is executed. See {\hyperref[valve/valve_manual:clusterization-of-inlets]{\emph{Clusterization of inlets}}} for more details.


\subsection{Option \textbf{singletons\_outliers}}
\label{valve/valve_config:option-singletons-outliers}
Maximal size of cluster to be considered as outliers. If set to number \textgreater{} 0 clusters of that size are removed and their objects are moved to outliers. See {\hyperref[valve/valve_manual:clusterization-of-inlets]{\emph{Clusterization of inlets}}} for more details.


\section{Stage \textbf{analysis}}
\label{valve/valve_config:stage-analysis}

\subsection{Option \textbf{dump\_config}}
\label{valve/valve_config:option-dump-config}
If set to \code{True} configuration options, as seen by Valve, are added to the head of results.


\section{Stage \textbf{visualize}}
\label{valve/valve_config:stage-visualize}

\subsection{Option \textbf{simply\_smooths}}
\label{valve/valve_config:option-simply-smooths}
If set to float number simplification of smooth paths is applied.
Simplification removes points which do not (or almost do not) change the shape of smooth path. For more details see {\hyperref[aqueduct.geom.traces:simply-smooths-details]{\emph{Recursive Vector Linearization}}}.


\subsection{Option \textbf{all\_paths\_raw}}
\label{valve/valve_config:option-all-paths-raw}
If True produces one object in PyMOL that holds all paths visualized by raw coordinates.


\subsection{Option \textbf{all\_paths\_smooth}}
\label{valve/valve_config:option-all-paths-smooth}
If True produces one object in PyMOL that holds all paths visualized by smooth coordinates.


\subsection{Option \textbf{all\_paths\_split}}
\label{valve/valve_config:option-all-paths-split}
If is set True objects produced by \textbf{all\_paths\_raw} and \textbf{all\_paths\_smooth} are split into Incoming, Object, and Outgoing parts and visualized as three different objects.


\subsection{Options \textbf{all\_paths\_raw\_io} and \textbf{all\_paths\_smooth\_io}}
\label{valve/valve_config:options-all-paths-raw-io-and-all-paths-smooth-io}
If set True arrows pointing beginning and end of paths are displayed oriented accordingly to raw or smooth paths.


\subsection{Option \textbf{paths\_raw}}
\label{valve/valve_config:option-paths-raw}
If set True raw paths are displayed as separate objects or as one object with states corresponding to number of path.


\subsection{Option \textbf{paths\_raw}}
\label{valve/valve_config:id1}
If set True smooth paths are displayed as separate objects or as one object with states corresponding to number of path.


\subsection{Option \textbf{paths\_raw\_io}}
\label{valve/valve_config:option-paths-raw-io}
If set True arrows indicating beginning and and of paths, oriented accordingly to raw paths, are displayed as separate objects or as one object with states corresponding to number of paths.


\subsection{Option \textbf{paths\_smooth\_io}}
\label{valve/valve_config:option-paths-smooth-io}
If set True arrows indicating beginning and and of paths, oriented accordingly to smooth paths, are displayed as separate objects or as one object with states corresponding to number of paths.


\subsection{Option \textbf{paths\_states}}
\label{valve/valve_config:option-paths-states}
If True objects displayed by \textbf{paths\_raw}, \textbf{paths\_smooth}, \textbf{paths\_raw\_io}, and \textbf{paths\_smooth\_io} are displayed as one object with with states corresponding to number of paths. Otherwise they are displayed as separate objects.


\subsection{Option \textbf{ctypes\_raw}}
\label{valve/valve_config:option-ctypes-raw}
Displays raw paths in a similar manner as non split \textbf{all\_paths\_raw} but each cluster type is displayed in separate object.


\subsection{Option \textbf{ctypes\_smooth}}
\label{valve/valve_config:option-ctypes-smooth}
Displays smooth paths in a similar manner as non split \textbf{all\_paths\_smooth} but each cluster type is displayed in separate object.


\subsection{Option \textbf{show\_molecule}}
\label{valve/valve_config:option-show-molecule}
If is set to selection of some molecular object in the system, for example to \code{protein}, this object is displayed.

\begin{notice}{note}{Note:}
Possibly due to limitations of \code{MDAnalysis} only whole molecules can be displayed. If \textbf{show\_molecule} is set to \code{backbone} complete protein will be displayed any way. This may change in future version of \code{MDAnalysis} and or {\hyperref[aqueduct:module-aqueduct]{\code{aqueduct}}}.
\end{notice}


\subsection{Option \textbf{show\_molecule\_frames}}
\label{valve/valve_config:option-show-molecule-frames}
Allows to indicate which frames of object defined by \textbf{show\_molecule} should be displayed. It is possible to set several frames. In that case frames would be displayed as states.

\begin{notice}{note}{Note:}
If several frames are selected they are displayed as states which may interfere with other PyMOL objects displayed with several states.
\end{notice}

\begin{notice}{note}{Note:}
If several states are displayed protein tertiary structure data might be lost. This seems to be limitation of either \code{MDAnalysis} or PyMOL.
\end{notice}


\chapter{\emph{Valve} tutorial}
\label{valve/valve_tutorial:valve-tutorial}\label{valve/valve_tutorial::doc}
This is a tentative \emph{Valve} manual. Created for the sake of Aqueduct training we have today. Eventually, it will be rewritten to the official version.

This tutorial assumes {\hyperref[aqueduct:module-aqueduct]{\code{aqueduct}}} and \emph{Valve} is already installed - see {\hyperref[aqueduct_install::doc]{\emph{Aqueduct installation guide}}}. It is also assumed that user is acquainted with {\hyperref[valve/valve_manual::doc]{\emph{Valve manual}}} and \emph{Valve} {\hyperref[valve/valve_config::doc]{\emph{Configuration file options}}}.


\section{\emph{Valve} invocation}
\label{valve/valve_tutorial:valve-invocation}
Usually \emph{Valve} is run by:

\begin{Verbatim}[commandchars=\\\{\}]
\PYG{n}{valve}\PYG{o}{.}\PYG{n}{py}
\end{Verbatim}

Due to specific setup in our laboratory \emph{Valve} has to be run through simple wrapper script:

\begin{Verbatim}[commandchars=\\\{\}]
\PYG{n}{valve\PYGZus{}run}
\end{Verbatim}

Additionally, to speed up all calculations it is assumed that \emph{Valve} is run with \code{-{-}max-frame 1000} option:

\begin{Verbatim}[commandchars=\\\{\}]
valve\PYGZus{}run \PYGZhy{}\PYGZhy{}max\PYGZhy{}frame 1000
\end{Verbatim}

To check is \emph{Valve} is installed and works properly try to issue following commands:

\begin{Verbatim}[commandchars=\\\{\}]
\PYG{n}{valve\PYGZus{}run} \PYG{o}{\PYGZhy{}}\PYG{o}{\PYGZhy{}}\PYG{n}{help}
\PYG{n}{valve\PYGZus{}run} \PYG{o}{\PYGZhy{}}\PYG{o}{\PYGZhy{}}\PYG{n}{version}
\end{Verbatim}


\section{Test data}
\label{valve/valve_tutorial:test-data}
\textbf{Mouse!}

We will use 10ns Amber MD simulation data of sEH protein (PDBID \textbf{1cqz}). Necessary files can be downloaded \href{http://localhost:8001}{here}:
\begin{itemize}
\item {} 
Go to download server.

\item {} 
Go to \code{1cqz} directory.

\item {} 
Download all files and save them in sane location on your machine. Please note, that \code{.nc} file is ca. 3.5 GB so it may take a while to download it.

\end{itemize}


\section{Inspect your system}
\label{valve/valve_tutorial:inspect-your-system}
Before we start any calculations lets have a look at the protein of interest. Start \emph{PyMOL} and get \code{1cqz} PDB structure (for example by typing in \emph{PyMOL} command prompt \code{fetch 1cqa}).

To setup \emph{Valve} calculations we need to know active site of the protein. More precisely we need to know IDs or residues that are in the active site. This would allow us to create {\hyperref[valve/valve_manual:object-definition]{\emph{Object definition}}}.

But wait. Is it really the correct structure? How many chains there are? What is the numeration of residues?


\subsection{Create \emph{Object definition}}
\label{valve/valve_tutorial:create-object-definition}
Lets load another structure. Open file \code{first\_frame\_1cqz.pdb} downloaded from test data \href{http://localhost:8001}{repository}. It is a first frame of the MD simulation and it is en example of how the frame of MD looks like. In order to create {\hyperref[valve/valve_manual:object-definition]{\emph{Object definition}}} you have to discover following things:
\begin{enumerate}
\item {} 
What is the name of water residue?

\item {} 
What are numbers of residues in the active site?

\item {} 
What size the active site is?

\end{enumerate}

\begin{notice}{note}{Note:}
It is also good idea to open \code{.pdb} file in your favorite text editor and look at residue numbers and names.
\end{notice}


\subsection{Create \emph{Scope definition}}
\label{valve/valve_tutorial:create-scope-definition}
{\hyperref[valve/valve_manual:scope-definition]{\emph{Scope definition}}} is easy to create. We will use \emph{Convex hull} version so the scope definition could be simply \code{backbone}.


\section{Prepare config file}
\label{valve/valve_tutorial:prepare-config-file}
\emph{Valve} performs calculations according to the configuration (aka \emph{config}) file.

Lets start from dumping config file template to \code{config.txt} file. Open it in your favorite editor and fill all options.
If you have troubles look at {\hyperref[valve/valve_config::doc]{\emph{Configuration file options}}} (and {\hyperref[valve/valve_manual::doc]{\emph{Valve manual}}}).

Things to remember:
\begin{enumerate}
\item {} 
Provide correct paths to topology and trajectory data.

\item {} 
Enter correct {\hyperref[valve/valve_manual:object-definition]{\emph{Object}}} and {\hyperref[valve/valve_manual:scope-definition]{\emph{Scope}}} definitions.

\item {} 
Provide file name of result in analysis section, for example \code{results.txt} (for future reference).

\item {} 
Make sure visualization is switched on and \code{save} option points to session file name (\code{.pse})

\end{enumerate}


\section{Run \emph{Valve}}
\label{valve/valve_tutorial:run-valve}
Make sure all necessary data is in place. Open terminal, go to your working directory and type in:

\begin{Verbatim}[commandchars=\\\{\}]
valve\PYGZus{}run \PYGZhy{}\PYGZhy{}max\PYGZhy{}frame 1000 \PYGZhy{}c config.txt
\end{Verbatim}

Depending on your machine and current load it may take a while (matter of minutes) to complete all calculations.


\subsection{Visual inspection}
\label{valve/valve_tutorial:visual-inspection}
In the last stage \emph{PyMOL} should pop up and \emph{Valve} should start to feed it with visualization data. This would take a moment and if you set up \code{save} option a \emph{PyMOL} session would be saved. Once it is done \emph{Valve} quits and switches off \emph{PyMOL}. Now, you can restart it and read saved session.


\subsection{Analysis tables}
\label{valve/valve_tutorial:analysis-tables}
Open \code{results.txt} file and look at summaries and tables. See also {\hyperref[valve/valve_manual::doc]{\emph{Valve manual}}}.


\section{Feedback}
\label{valve/valve_tutorial:feedback}
Give us your opinion. Send your questions, inquires, anything to developer(s): \href{mailto:t.magdziarz@tunnelinggroup.pl}{Tomasz Magdziarz}.
This are couple of questions that might be useful to form your opinion.
\begin{enumerate}
\item {} 
What do you like in \emph{Valve} and \emph{Aqueduct}?

\item {} 
What do you do not like in \emph{Valve} or \emph{Aqueduct}?

\item {} 
What is missing?

\item {} 
Do you find it useful?

\end{enumerate}


\chapter{aqueduct}
\label{modules:aqueduct}\label{modules::doc}

\section{aqueduct package}
\label{aqueduct::doc}\label{aqueduct:aqueduct-package}

\subsection{Subpackages}
\label{aqueduct:subpackages}

\subsubsection{aqueduct.geom package}
\label{aqueduct.geom::doc}\label{aqueduct.geom:aqueduct-geom-package}

\paragraph{Submodules}
\label{aqueduct.geom:submodules}

\subparagraph{aqueduct.geom.cluster module}
\label{aqueduct.geom.cluster:module-aqueduct.geom.cluster}\label{aqueduct.geom.cluster::doc}\label{aqueduct.geom.cluster:aqueduct-geom-cluster-module}\index{aqueduct.geom.cluster (module)}
This module provides functions for clusterization.
Clusterization is done by \code{scikit-learn} module.
\index{MeanShiftBandwidth() (in module aqueduct.geom.cluster)}

\begin{fulllineitems}
\phantomsection\label{aqueduct.geom.cluster:aqueduct.geom.cluster.MeanShiftBandwidth}\pysiglinewithargsret{\bfcode{MeanShiftBandwidth}}{\emph{X}, \emph{**kwargs}}{}
\end{fulllineitems}

\index{PerformClustering (class in aqueduct.geom.cluster)}

\begin{fulllineitems}
\phantomsection\label{aqueduct.geom.cluster:aqueduct.geom.cluster.PerformClustering}\pysiglinewithargsret{\strong{class }\bfcode{PerformClustering}}{\emph{method}, \emph{**kwargs}}{}
Bases: \href{http://docs.python.org/2/library/functions.html\#object}{\code{object}}
\index{\_\_init\_\_() (PerformClustering method)}

\begin{fulllineitems}
\phantomsection\label{aqueduct.geom.cluster:aqueduct.geom.cluster.PerformClustering.__init__}\pysiglinewithargsret{\bfcode{\_\_init\_\_}}{\emph{method}, \emph{**kwargs}}{}
\end{fulllineitems}

\index{\_\_call\_\_() (PerformClustering method)}

\begin{fulllineitems}
\phantomsection\label{aqueduct.geom.cluster:aqueduct.geom.cluster.PerformClustering.__call__}\pysiglinewithargsret{\bfcode{\_\_call\_\_}}{\emph{coords}}{}
\end{fulllineitems}

\index{fit() (PerformClustering method)}

\begin{fulllineitems}
\phantomsection\label{aqueduct.geom.cluster:aqueduct.geom.cluster.PerformClustering.fit}\pysiglinewithargsret{\bfcode{fit}}{\emph{coords}}{}
\end{fulllineitems}

\index{centers() (PerformClustering method)}

\begin{fulllineitems}
\phantomsection\label{aqueduct.geom.cluster:aqueduct.geom.cluster.PerformClustering.centers}\pysiglinewithargsret{\bfcode{centers}}{}{}
\end{fulllineitems}


\end{fulllineitems}



\subparagraph{aqueduct.geom.convexhull module}
\label{aqueduct.geom.convexhull:aqueduct-geom-convexhull-module}\label{aqueduct.geom.convexhull::doc}\label{aqueduct.geom.convexhull:module-aqueduct.geom.convexhull}\index{aqueduct.geom.convexhull (module)}
Created on Dec 4, 2015

@author: tljm
\index{is\_point\_within\_convexhull() (in module aqueduct.geom.convexhull)}

\begin{fulllineitems}
\phantomsection\label{aqueduct.geom.convexhull:aqueduct.geom.convexhull.is_point_within_convexhull}\pysiglinewithargsret{\bfcode{is\_point\_within\_convexhull}}{\emph{point\_chull}}{}
\end{fulllineitems}



\subparagraph{aqueduct.geom.convexhull\_pyhull module}
\label{aqueduct.geom.convexhull_pyhull:aqueduct-geom-convexhull-pyhull-module}\label{aqueduct.geom.convexhull_pyhull::doc}\label{aqueduct.geom.convexhull_pyhull:module-aqueduct.geom.convexhull_pyhull}\index{aqueduct.geom.convexhull\_pyhull (module)}\index{\_unfold\_vertices\_list() (in module aqueduct.geom.convexhull\_pyhull)}

\begin{fulllineitems}
\phantomsection\label{aqueduct.geom.convexhull_pyhull:aqueduct.geom.convexhull_pyhull._unfold_vertices_list}\pysiglinewithargsret{\bfcode{\_unfold\_vertices\_list}}{\emph{vertices}}{}
\end{fulllineitems}

\index{\_vertices\_ids() (in module aqueduct.geom.convexhull\_pyhull)}

\begin{fulllineitems}
\phantomsection\label{aqueduct.geom.convexhull_pyhull:aqueduct.geom.convexhull_pyhull._vertices_ids}\pysiglinewithargsret{\bfcode{\_vertices\_ids}}{\emph{convexhull}}{}
\end{fulllineitems}

\index{\_vertices\_points() (in module aqueduct.geom.convexhull\_pyhull)}

\begin{fulllineitems}
\phantomsection\label{aqueduct.geom.convexhull_pyhull:aqueduct.geom.convexhull_pyhull._vertices_points}\pysiglinewithargsret{\bfcode{\_vertices\_points}}{\emph{convexhull}}{}
\end{fulllineitems}

\index{\_point\_within\_convexhull() (in module aqueduct.geom.convexhull\_pyhull)}

\begin{fulllineitems}
\phantomsection\label{aqueduct.geom.convexhull_pyhull:aqueduct.geom.convexhull_pyhull._point_within_convexhull}\pysiglinewithargsret{\bfcode{\_point\_within\_convexhull}}{\emph{convexhull}, \emph{point}}{}
\end{fulllineitems}

\index{\_facets() (in module aqueduct.geom.convexhull\_pyhull)}

\begin{fulllineitems}
\phantomsection\label{aqueduct.geom.convexhull_pyhull:aqueduct.geom.convexhull_pyhull._facets}\pysiglinewithargsret{\bfcode{\_facets}}{\emph{convexhull}}{}
\end{fulllineitems}



\subparagraph{aqueduct.geom.convexhull\_scipy module}
\label{aqueduct.geom.convexhull_scipy:aqueduct-geom-convexhull-scipy-module}\label{aqueduct.geom.convexhull_scipy::doc}\label{aqueduct.geom.convexhull_scipy:module-aqueduct.geom.convexhull_scipy}\index{aqueduct.geom.convexhull\_scipy (module)}\index{\_vertices\_ids() (in module aqueduct.geom.convexhull\_scipy)}

\begin{fulllineitems}
\phantomsection\label{aqueduct.geom.convexhull_scipy:aqueduct.geom.convexhull_scipy._vertices_ids}\pysiglinewithargsret{\bfcode{\_vertices\_ids}}{\emph{convexhull}}{}
\end{fulllineitems}

\index{\_vertices\_points() (in module aqueduct.geom.convexhull\_scipy)}

\begin{fulllineitems}
\phantomsection\label{aqueduct.geom.convexhull_scipy:aqueduct.geom.convexhull_scipy._vertices_points}\pysiglinewithargsret{\bfcode{\_vertices\_points}}{\emph{convexhull}}{}
\end{fulllineitems}

\index{\_point\_within\_convexhull() (in module aqueduct.geom.convexhull\_scipy)}

\begin{fulllineitems}
\phantomsection\label{aqueduct.geom.convexhull_scipy:aqueduct.geom.convexhull_scipy._point_within_convexhull}\pysiglinewithargsret{\bfcode{\_point\_within\_convexhull}}{\emph{convexhull}, \emph{point}}{}
\end{fulllineitems}

\index{\_facets() (in module aqueduct.geom.convexhull\_scipy)}

\begin{fulllineitems}
\phantomsection\label{aqueduct.geom.convexhull_scipy:aqueduct.geom.convexhull_scipy._facets}\pysiglinewithargsret{\bfcode{\_facets}}{\emph{convexhull}}{}
\end{fulllineitems}



\subparagraph{aqueduct.geom.master module}
\label{aqueduct.geom.master:module-aqueduct.geom.master}\label{aqueduct.geom.master::doc}\label{aqueduct.geom.master:aqueduct-geom-master-module}\index{aqueduct.geom.master (module)}\index{fit\_trace\_to\_points() (in module aqueduct.geom.master)}

\begin{fulllineitems}
\phantomsection\label{aqueduct.geom.master:aqueduct.geom.master.fit_trace_to_points}\pysiglinewithargsret{\bfcode{fit\_trace\_to\_points}}{\emph{trace}, \emph{points}}{}
\end{fulllineitems}

\index{strech\_zip() (in module aqueduct.geom.master)}

\begin{fulllineitems}
\phantomsection\label{aqueduct.geom.master:aqueduct.geom.master.strech_zip}\pysiglinewithargsret{\bfcode{strech\_zip}}{\emph{*args}}{}
\end{fulllineitems}

\index{compress\_zip() (in module aqueduct.geom.master)}

\begin{fulllineitems}
\phantomsection\label{aqueduct.geom.master:aqueduct.geom.master.compress_zip}\pysiglinewithargsret{\bfcode{compress\_zip}}{\emph{*args}}{}
\end{fulllineitems}

\index{zip\_zip() (in module aqueduct.geom.master)}

\begin{fulllineitems}
\phantomsection\label{aqueduct.geom.master:aqueduct.geom.master.zip_zip}\pysiglinewithargsret{\bfcode{zip\_zip}}{\emph{*args}, \emph{**kwargs}}{}
\end{fulllineitems}

\index{decide\_on\_type() (in module aqueduct.geom.master)}

\begin{fulllineitems}
\phantomsection\label{aqueduct.geom.master:aqueduct.geom.master.decide_on_type}\pysiglinewithargsret{\bfcode{decide\_on\_type}}{\emph{cont}, \emph{s2o\_treshold=0.5}}{}
\end{fulllineitems}

\index{simple\_types\_distribution() (in module aqueduct.geom.master)}

\begin{fulllineitems}
\phantomsection\label{aqueduct.geom.master:aqueduct.geom.master.simple_types_distribution}\pysiglinewithargsret{\bfcode{simple\_types\_distribution}}{\emph{types}}{}
\end{fulllineitems}

\index{get\_weights\_() (in module aqueduct.geom.master)}

\begin{fulllineitems}
\phantomsection\label{aqueduct.geom.master:aqueduct.geom.master.get_weights_}\pysiglinewithargsret{\bfcode{get\_weights\_}}{\emph{spaths}, \emph{smooth=None}}{}
\end{fulllineitems}

\index{get\_mean\_coord\_() (in module aqueduct.geom.master)}

\begin{fulllineitems}
\phantomsection\label{aqueduct.geom.master:aqueduct.geom.master.get_mean_coord_}\pysiglinewithargsret{\bfcode{get\_mean\_coord\_}}{\emph{coords}, \emph{l}}{}
\end{fulllineitems}

\index{concatenate() (in module aqueduct.geom.master)}

\begin{fulllineitems}
\phantomsection\label{aqueduct.geom.master:aqueduct.geom.master.concatenate}\pysiglinewithargsret{\bfcode{concatenate}}{\emph{*args}}{}
\end{fulllineitems}

\index{create\_master\_spath() (in module aqueduct.geom.master)}

\begin{fulllineitems}
\phantomsection\label{aqueduct.geom.master:aqueduct.geom.master.create_master_spath}\pysiglinewithargsret{\bfcode{create\_master\_spath}}{\emph{spaths}, \emph{smooth=None}, \emph{resid=0}, \emph{ctype=None}, \emph{bias\_long=5}, \emph{heartbeat=None}}{}
\end{fulllineitems}

\index{MasterTrace (class in aqueduct.geom.master)}

\begin{fulllineitems}
\phantomsection\label{aqueduct.geom.master:aqueduct.geom.master.MasterTrace}\pysigline{\strong{class }\bfcode{MasterTrace}}
Bases: \href{http://docs.python.org/2/library/functions.html\#object}{\code{object}}
\index{\_\_init\_\_() (MasterTrace method)}

\begin{fulllineitems}
\phantomsection\label{aqueduct.geom.master:aqueduct.geom.master.MasterTrace.__init__}\pysiglinewithargsret{\bfcode{\_\_init\_\_}}{}{}
\end{fulllineitems}


\end{fulllineitems}



\subparagraph{aqueduct.geom.pca module}
\label{aqueduct.geom.pca:aqueduct-geom-pca-module}\label{aqueduct.geom.pca:module-aqueduct.geom.pca}\label{aqueduct.geom.pca::doc}\index{aqueduct.geom.pca (module)}\index{Center (class in aqueduct.geom.pca)}

\begin{fulllineitems}
\phantomsection\label{aqueduct.geom.pca:aqueduct.geom.pca.Center}\pysiglinewithargsret{\strong{class }\bfcode{Center}}{\emph{X}}{}
Bases: \href{http://docs.python.org/2/library/functions.html\#object}{\code{object}}
\index{\_\_init\_\_() (Center method)}

\begin{fulllineitems}
\phantomsection\label{aqueduct.geom.pca:aqueduct.geom.pca.Center.__init__}\pysiglinewithargsret{\bfcode{\_\_init\_\_}}{\emph{X}}{}
\end{fulllineitems}

\index{\_\_call\_\_() (Center method)}

\begin{fulllineitems}
\phantomsection\label{aqueduct.geom.pca:aqueduct.geom.pca.Center.__call__}\pysiglinewithargsret{\bfcode{\_\_call\_\_}}{\emph{X}}{}
\end{fulllineitems}

\index{undo() (Center method)}

\begin{fulllineitems}
\phantomsection\label{aqueduct.geom.pca:aqueduct.geom.pca.Center.undo}\pysiglinewithargsret{\bfcode{undo}}{\emph{X}}{}
\end{fulllineitems}


\end{fulllineitems}

\index{Normalize (class in aqueduct.geom.pca)}

\begin{fulllineitems}
\phantomsection\label{aqueduct.geom.pca:aqueduct.geom.pca.Normalize}\pysiglinewithargsret{\strong{class }\bfcode{Normalize}}{\emph{X}}{}
Bases: \href{http://docs.python.org/2/library/functions.html\#object}{\code{object}}
\index{\_\_init\_\_() (Normalize method)}

\begin{fulllineitems}
\phantomsection\label{aqueduct.geom.pca:aqueduct.geom.pca.Normalize.__init__}\pysiglinewithargsret{\bfcode{\_\_init\_\_}}{\emph{X}}{}
\end{fulllineitems}

\index{\_\_call\_\_() (Normalize method)}

\begin{fulllineitems}
\phantomsection\label{aqueduct.geom.pca:aqueduct.geom.pca.Normalize.__call__}\pysiglinewithargsret{\bfcode{\_\_call\_\_}}{\emph{X}}{}
\end{fulllineitems}

\index{undo() (Normalize method)}

\begin{fulllineitems}
\phantomsection\label{aqueduct.geom.pca:aqueduct.geom.pca.Normalize.undo}\pysiglinewithargsret{\bfcode{undo}}{\emph{X}}{}
\end{fulllineitems}


\end{fulllineitems}

\index{Standartize (class in aqueduct.geom.pca)}

\begin{fulllineitems}
\phantomsection\label{aqueduct.geom.pca:aqueduct.geom.pca.Standartize}\pysiglinewithargsret{\strong{class }\bfcode{Standartize}}{\emph{X}}{}
Bases: {\hyperref[aqueduct.geom.pca:aqueduct.geom.pca.Center]{\code{aqueduct.geom.pca.Center}}}, {\hyperref[aqueduct.geom.pca:aqueduct.geom.pca.Normalize]{\code{aqueduct.geom.pca.Normalize}}}
\index{\_\_init\_\_() (Standartize method)}

\begin{fulllineitems}
\phantomsection\label{aqueduct.geom.pca:aqueduct.geom.pca.Standartize.__init__}\pysiglinewithargsret{\bfcode{\_\_init\_\_}}{\emph{X}}{}
\end{fulllineitems}

\index{\_\_call\_\_() (Standartize method)}

\begin{fulllineitems}
\phantomsection\label{aqueduct.geom.pca:aqueduct.geom.pca.Standartize.__call__}\pysiglinewithargsret{\bfcode{\_\_call\_\_}}{\emph{X}}{}
\end{fulllineitems}

\index{undo() (Standartize method)}

\begin{fulllineitems}
\phantomsection\label{aqueduct.geom.pca:aqueduct.geom.pca.Standartize.undo}\pysiglinewithargsret{\bfcode{undo}}{\emph{X}}{}
\end{fulllineitems}


\end{fulllineitems}

\index{PCA (class in aqueduct.geom.pca)}

\begin{fulllineitems}
\phantomsection\label{aqueduct.geom.pca:aqueduct.geom.pca.PCA}\pysiglinewithargsret{\strong{class }\bfcode{PCA}}{\emph{X}, \emph{prepro=None}}{}
Bases: \href{http://docs.python.org/2/library/functions.html\#object}{\code{object}}
\index{\_\_init\_\_() (PCA method)}

\begin{fulllineitems}
\phantomsection\label{aqueduct.geom.pca:aqueduct.geom.pca.PCA.__init__}\pysiglinewithargsret{\bfcode{\_\_init\_\_}}{\emph{X}, \emph{prepro=None}}{}
\end{fulllineitems}

\index{P (PCA attribute)}

\begin{fulllineitems}
\phantomsection\label{aqueduct.geom.pca:aqueduct.geom.pca.PCA.P}\pysigline{\bfcode{P}}
\end{fulllineitems}

\index{preprocess() (PCA method)}

\begin{fulllineitems}
\phantomsection\label{aqueduct.geom.pca:aqueduct.geom.pca.PCA.preprocess}\pysiglinewithargsret{\bfcode{preprocess}}{\emph{X}}{}
\end{fulllineitems}

\index{preprocess\_undo() (PCA method)}

\begin{fulllineitems}
\phantomsection\label{aqueduct.geom.pca:aqueduct.geom.pca.PCA.preprocess_undo}\pysiglinewithargsret{\bfcode{preprocess\_undo}}{\emph{X}}{}
\end{fulllineitems}

\index{\_\_call\_\_() (PCA method)}

\begin{fulllineitems}
\phantomsection\label{aqueduct.geom.pca:aqueduct.geom.pca.PCA.__call__}\pysiglinewithargsret{\bfcode{\_\_call\_\_}}{\emph{X}}{}
\end{fulllineitems}

\index{undo() (PCA method)}

\begin{fulllineitems}
\phantomsection\label{aqueduct.geom.pca:aqueduct.geom.pca.PCA.undo}\pysiglinewithargsret{\bfcode{undo}}{\emph{T}}{}
\end{fulllineitems}


\end{fulllineitems}



\subparagraph{aqueduct.geom.smooth module}
\label{aqueduct.geom.smooth:module-aqueduct.geom.smooth}\label{aqueduct.geom.smooth:aqueduct-geom-smooth-module}\label{aqueduct.geom.smooth::doc}\index{aqueduct.geom.smooth (module)}
Created on Dec 15, 2015

@author: tljm
\index{Smooth (class in aqueduct.geom.smooth)}

\begin{fulllineitems}
\phantomsection\label{aqueduct.geom.smooth:aqueduct.geom.smooth.Smooth}\pysiglinewithargsret{\strong{class }\bfcode{Smooth}}{\emph{recursive=None}, \emph{**kwargs}}{}
Bases: \href{http://docs.python.org/2/library/functions.html\#object}{\code{object}}
\index{\_\_init\_\_() (Smooth method)}

\begin{fulllineitems}
\phantomsection\label{aqueduct.geom.smooth:aqueduct.geom.smooth.Smooth.__init__}\pysiglinewithargsret{\bfcode{\_\_init\_\_}}{\emph{recursive=None}, \emph{**kwargs}}{}
\end{fulllineitems}

\index{smooth() (Smooth method)}

\begin{fulllineitems}
\phantomsection\label{aqueduct.geom.smooth:aqueduct.geom.smooth.Smooth.smooth}\pysiglinewithargsret{\bfcode{smooth}}{\emph{coords}}{}
\end{fulllineitems}

\index{\_\_call\_\_() (Smooth method)}

\begin{fulllineitems}
\phantomsection\label{aqueduct.geom.smooth:aqueduct.geom.smooth.Smooth.__call__}\pysiglinewithargsret{\bfcode{\_\_call\_\_}}{\emph{coords}}{}
\end{fulllineitems}


\end{fulllineitems}

\index{GeneralWindow (class in aqueduct.geom.smooth)}

\begin{fulllineitems}
\phantomsection\label{aqueduct.geom.smooth:aqueduct.geom.smooth.GeneralWindow}\pysigline{\strong{class }\bfcode{GeneralWindow}}~\index{max\_window\_at\_pos() (GeneralWindow static method)}

\begin{fulllineitems}
\phantomsection\label{aqueduct.geom.smooth:aqueduct.geom.smooth.GeneralWindow.max_window_at_pos}\pysiglinewithargsret{\strong{static }\bfcode{max\_window\_at\_pos}}{\emph{pos}, \emph{size}}{}
\end{fulllineitems}

\index{check\_bounds\_at\_max\_window\_at\_pos() (GeneralWindow method)}

\begin{fulllineitems}
\phantomsection\label{aqueduct.geom.smooth:aqueduct.geom.smooth.GeneralWindow.check_bounds_at_max_window_at_pos}\pysiglinewithargsret{\bfcode{check\_bounds\_at\_max\_window\_at\_pos}}{\emph{lb}, \emph{ub}, \emph{pos}, \emph{size}}{}
\end{fulllineitems}


\end{fulllineitems}

\index{WindowSmooth (class in aqueduct.geom.smooth)}

\begin{fulllineitems}
\phantomsection\label{aqueduct.geom.smooth:aqueduct.geom.smooth.WindowSmooth}\pysiglinewithargsret{\strong{class }\bfcode{WindowSmooth}}{\emph{window=5}, \emph{function=\textless{}function mean at 0x2af4e19fab18\textgreater{}}, \emph{**kwargs}}{}
Bases: {\hyperref[aqueduct.geom.smooth:aqueduct.geom.smooth.Smooth]{\code{aqueduct.geom.smooth.Smooth}}}, {\hyperref[aqueduct.geom.smooth:aqueduct.geom.smooth.GeneralWindow]{\code{aqueduct.geom.smooth.GeneralWindow}}}
\index{\_\_init\_\_() (WindowSmooth method)}

\begin{fulllineitems}
\phantomsection\label{aqueduct.geom.smooth:aqueduct.geom.smooth.WindowSmooth.__init__}\pysiglinewithargsret{\bfcode{\_\_init\_\_}}{\emph{window=5}, \emph{function=\textless{}function mean at 0x2af4e19fab18\textgreater{}}, \emph{**kwargs}}{}
\end{fulllineitems}

\index{smooth() (WindowSmooth method)}

\begin{fulllineitems}
\phantomsection\label{aqueduct.geom.smooth:aqueduct.geom.smooth.WindowSmooth.smooth}\pysiglinewithargsret{\bfcode{smooth}}{\emph{*args}, \emph{**kwargs}}{}
\end{fulllineitems}


\end{fulllineitems}

\index{DistanceWindowSmooth (class in aqueduct.geom.smooth)}

\begin{fulllineitems}
\phantomsection\label{aqueduct.geom.smooth:aqueduct.geom.smooth.DistanceWindowSmooth}\pysiglinewithargsret{\strong{class }\bfcode{DistanceWindowSmooth}}{\emph{window=5}, \emph{function=\textless{}function mean at 0x2af4e19fab18\textgreater{}}, \emph{**kwargs}}{}
Bases: {\hyperref[aqueduct.geom.smooth:aqueduct.geom.smooth.Smooth]{\code{aqueduct.geom.smooth.Smooth}}}, {\hyperref[aqueduct.geom.smooth:aqueduct.geom.smooth.GeneralWindow]{\code{aqueduct.geom.smooth.GeneralWindow}}}
\index{\_\_init\_\_() (DistanceWindowSmooth method)}

\begin{fulllineitems}
\phantomsection\label{aqueduct.geom.smooth:aqueduct.geom.smooth.DistanceWindowSmooth.__init__}\pysiglinewithargsret{\bfcode{\_\_init\_\_}}{\emph{window=5}, \emph{function=\textless{}function mean at 0x2af4e19fab18\textgreater{}}, \emph{**kwargs}}{}
\end{fulllineitems}

\index{smooth() (DistanceWindowSmooth method)}

\begin{fulllineitems}
\phantomsection\label{aqueduct.geom.smooth:aqueduct.geom.smooth.DistanceWindowSmooth.smooth}\pysiglinewithargsret{\bfcode{smooth}}{\emph{*args}, \emph{**kwargs}}{}
\end{fulllineitems}


\end{fulllineitems}

\index{ActiveWindowSmooth (class in aqueduct.geom.smooth)}

\begin{fulllineitems}
\phantomsection\label{aqueduct.geom.smooth:aqueduct.geom.smooth.ActiveWindowSmooth}\pysiglinewithargsret{\strong{class }\bfcode{ActiveWindowSmooth}}{\emph{window=5}, \emph{function=\textless{}function mean at 0x2af4e19fab18\textgreater{}}, \emph{**kwargs}}{}
Bases: {\hyperref[aqueduct.geom.smooth:aqueduct.geom.smooth.Smooth]{\code{aqueduct.geom.smooth.Smooth}}}, {\hyperref[aqueduct.geom.smooth:aqueduct.geom.smooth.GeneralWindow]{\code{aqueduct.geom.smooth.GeneralWindow}}}
\index{\_\_init\_\_() (ActiveWindowSmooth method)}

\begin{fulllineitems}
\phantomsection\label{aqueduct.geom.smooth:aqueduct.geom.smooth.ActiveWindowSmooth.__init__}\pysiglinewithargsret{\bfcode{\_\_init\_\_}}{\emph{window=5}, \emph{function=\textless{}function mean at 0x2af4e19fab18\textgreater{}}, \emph{**kwargs}}{}
\end{fulllineitems}

\index{smooth() (ActiveWindowSmooth method)}

\begin{fulllineitems}
\phantomsection\label{aqueduct.geom.smooth:aqueduct.geom.smooth.ActiveWindowSmooth.smooth}\pysiglinewithargsret{\bfcode{smooth}}{\emph{*args}, \emph{**kwargs}}{}
\end{fulllineitems}


\end{fulllineitems}

\index{MaxStepSmooth (class in aqueduct.geom.smooth)}

\begin{fulllineitems}
\phantomsection\label{aqueduct.geom.smooth:aqueduct.geom.smooth.MaxStepSmooth}\pysiglinewithargsret{\strong{class }\bfcode{MaxStepSmooth}}{\emph{step=1.0}, \emph{**kwargs}}{}
Bases: {\hyperref[aqueduct.geom.smooth:aqueduct.geom.smooth.Smooth]{\code{aqueduct.geom.smooth.Smooth}}}
\index{\_\_init\_\_() (MaxStepSmooth method)}

\begin{fulllineitems}
\phantomsection\label{aqueduct.geom.smooth:aqueduct.geom.smooth.MaxStepSmooth.__init__}\pysiglinewithargsret{\bfcode{\_\_init\_\_}}{\emph{step=1.0}, \emph{**kwargs}}{}
\end{fulllineitems}

\index{smooth() (MaxStepSmooth method)}

\begin{fulllineitems}
\phantomsection\label{aqueduct.geom.smooth:aqueduct.geom.smooth.MaxStepSmooth.smooth}\pysiglinewithargsret{\bfcode{smooth}}{\emph{*args}, \emph{**kwargs}}{}
\end{fulllineitems}


\end{fulllineitems}

\index{WindowOverMaxStepSmooth (class in aqueduct.geom.smooth)}

\begin{fulllineitems}
\phantomsection\label{aqueduct.geom.smooth:aqueduct.geom.smooth.WindowOverMaxStepSmooth}\pysiglinewithargsret{\strong{class }\bfcode{WindowOverMaxStepSmooth}}{\emph{**kwargs}}{}
Bases: {\hyperref[aqueduct.geom.smooth:aqueduct.geom.smooth.Smooth]{\code{aqueduct.geom.smooth.Smooth}}}
\index{\_\_init\_\_() (WindowOverMaxStepSmooth method)}

\begin{fulllineitems}
\phantomsection\label{aqueduct.geom.smooth:aqueduct.geom.smooth.WindowOverMaxStepSmooth.__init__}\pysiglinewithargsret{\bfcode{\_\_init\_\_}}{\emph{**kwargs}}{}
\end{fulllineitems}

\index{smooth() (WindowOverMaxStepSmooth method)}

\begin{fulllineitems}
\phantomsection\label{aqueduct.geom.smooth:aqueduct.geom.smooth.WindowOverMaxStepSmooth.smooth}\pysiglinewithargsret{\bfcode{smooth}}{\emph{coords}}{}
\end{fulllineitems}


\end{fulllineitems}

\index{ActiveWindowOverMaxStepSmooth (class in aqueduct.geom.smooth)}

\begin{fulllineitems}
\phantomsection\label{aqueduct.geom.smooth:aqueduct.geom.smooth.ActiveWindowOverMaxStepSmooth}\pysiglinewithargsret{\strong{class }\bfcode{ActiveWindowOverMaxStepSmooth}}{\emph{**kwargs}}{}
Bases: {\hyperref[aqueduct.geom.smooth:aqueduct.geom.smooth.Smooth]{\code{aqueduct.geom.smooth.Smooth}}}
\index{\_\_init\_\_() (ActiveWindowOverMaxStepSmooth method)}

\begin{fulllineitems}
\phantomsection\label{aqueduct.geom.smooth:aqueduct.geom.smooth.ActiveWindowOverMaxStepSmooth.__init__}\pysiglinewithargsret{\bfcode{\_\_init\_\_}}{\emph{**kwargs}}{}
\end{fulllineitems}

\index{smooth() (ActiveWindowOverMaxStepSmooth method)}

\begin{fulllineitems}
\phantomsection\label{aqueduct.geom.smooth:aqueduct.geom.smooth.ActiveWindowOverMaxStepSmooth.smooth}\pysiglinewithargsret{\bfcode{smooth}}{\emph{coords}}{}
\end{fulllineitems}


\end{fulllineitems}

\index{DistanceWindowOverMaxStepSmooth (class in aqueduct.geom.smooth)}

\begin{fulllineitems}
\phantomsection\label{aqueduct.geom.smooth:aqueduct.geom.smooth.DistanceWindowOverMaxStepSmooth}\pysiglinewithargsret{\strong{class }\bfcode{DistanceWindowOverMaxStepSmooth}}{\emph{**kwargs}}{}
Bases: {\hyperref[aqueduct.geom.smooth:aqueduct.geom.smooth.Smooth]{\code{aqueduct.geom.smooth.Smooth}}}
\index{\_\_init\_\_() (DistanceWindowOverMaxStepSmooth method)}

\begin{fulllineitems}
\phantomsection\label{aqueduct.geom.smooth:aqueduct.geom.smooth.DistanceWindowOverMaxStepSmooth.__init__}\pysiglinewithargsret{\bfcode{\_\_init\_\_}}{\emph{**kwargs}}{}
\end{fulllineitems}

\index{smooth() (DistanceWindowOverMaxStepSmooth method)}

\begin{fulllineitems}
\phantomsection\label{aqueduct.geom.smooth:aqueduct.geom.smooth.DistanceWindowOverMaxStepSmooth.smooth}\pysiglinewithargsret{\bfcode{smooth}}{\emph{coords}}{}
\end{fulllineitems}


\end{fulllineitems}



\subparagraph{aqueduct.geom.traces module}
\label{aqueduct.geom.traces:module-aqueduct.geom.traces}\label{aqueduct.geom.traces:aqueduct-geom-traces-module}\label{aqueduct.geom.traces::doc}\index{aqueduct.geom.traces (module)}\index{vector\_norm() (in module aqueduct.geom.traces)}

\begin{fulllineitems}
\phantomsection\label{aqueduct.geom.traces:aqueduct.geom.traces.vector_norm}\pysiglinewithargsret{\bfcode{vector\_norm}}{\emph{V}}{}
\end{fulllineitems}

\index{triangle\_angles() (in module aqueduct.geom.traces)}

\begin{fulllineitems}
\phantomsection\label{aqueduct.geom.traces:aqueduct.geom.traces.triangle_angles}\pysiglinewithargsret{\bfcode{triangle\_angles}}{\emph{A}, \emph{B}, \emph{C}}{}
\end{fulllineitems}

\index{triangle\_angles\_last() (in module aqueduct.geom.traces)}

\begin{fulllineitems}
\phantomsection\label{aqueduct.geom.traces:aqueduct.geom.traces.triangle_angles_last}\pysiglinewithargsret{\bfcode{triangle\_angles\_last}}{\emph{A}, \emph{B}, \emph{C}}{}
\end{fulllineitems}

\index{triangle\_height() (in module aqueduct.geom.traces)}

\begin{fulllineitems}
\phantomsection\label{aqueduct.geom.traces:aqueduct.geom.traces.triangle_height}\pysiglinewithargsret{\bfcode{triangle\_height}}{\emph{A}, \emph{B}, \emph{C}}{}
\end{fulllineitems}

\index{vectors\_angle() (in module aqueduct.geom.traces)}

\begin{fulllineitems}
\phantomsection\label{aqueduct.geom.traces:aqueduct.geom.traces.vectors_angle}\pysiglinewithargsret{\bfcode{vectors\_angle}}{\emph{A}, \emph{B}}{}
\end{fulllineitems}

\index{vectors\_angle\_alt() (in module aqueduct.geom.traces)}

\begin{fulllineitems}
\phantomsection\label{aqueduct.geom.traces:aqueduct.geom.traces.vectors_angle_alt}\pysiglinewithargsret{\bfcode{vectors\_angle\_alt}}{\emph{A}, \emph{B}}{}
\end{fulllineitems}

\index{vectors\_angle\_alt\_anorm() (in module aqueduct.geom.traces)}

\begin{fulllineitems}
\phantomsection\label{aqueduct.geom.traces:aqueduct.geom.traces.vectors_angle_alt_anorm}\pysiglinewithargsret{\bfcode{vectors\_angle\_alt\_anorm}}{\emph{A}, \emph{B}, \emph{A\_norm}}{}
\end{fulllineitems}

\index{vectors\_angle\_anorm() (in module aqueduct.geom.traces)}

\begin{fulllineitems}
\phantomsection\label{aqueduct.geom.traces:aqueduct.geom.traces.vectors_angle_anorm}\pysiglinewithargsret{\bfcode{vectors\_angle\_anorm}}{\emph{A}, \emph{B}, \emph{A\_norm}}{}
\end{fulllineitems}

\index{LinearizeOneWay (class in aqueduct.geom.traces)}

\begin{fulllineitems}
\phantomsection\label{aqueduct.geom.traces:aqueduct.geom.traces.LinearizeOneWay}\pysigline{\strong{class }\bfcode{LinearizeOneWay}}
Bases: \href{http://docs.python.org/2/library/functions.html\#object}{\code{object}}
\index{here() (LinearizeOneWay method)}

\begin{fulllineitems}
\phantomsection\label{aqueduct.geom.traces:aqueduct.geom.traces.LinearizeOneWay.here}\pysiglinewithargsret{\bfcode{here}}{\emph{coords}}{}
\end{fulllineitems}


\end{fulllineitems}

\index{LinearizeHobbit (class in aqueduct.geom.traces)}

\begin{fulllineitems}
\phantomsection\label{aqueduct.geom.traces:aqueduct.geom.traces.LinearizeHobbit}\pysigline{\strong{class }\bfcode{LinearizeHobbit}}
Bases: {\hyperref[aqueduct.geom.traces:aqueduct.geom.traces.LinearizeOneWay]{\code{aqueduct.geom.traces.LinearizeOneWay}}}
\index{and\_back\_again() (LinearizeHobbit method)}

\begin{fulllineitems}
\phantomsection\label{aqueduct.geom.traces:aqueduct.geom.traces.LinearizeHobbit.and_back_again}\pysiglinewithargsret{\bfcode{and\_back\_again}}{\emph{coords}}{}
\end{fulllineitems}

\index{\_\_call\_\_() (LinearizeHobbit method)}

\begin{fulllineitems}
\phantomsection\label{aqueduct.geom.traces:aqueduct.geom.traces.LinearizeHobbit.__call__}\pysiglinewithargsret{\bfcode{\_\_call\_\_}}{\emph{coords}}{}
\end{fulllineitems}


\end{fulllineitems}

\index{LinearizeRecursive (class in aqueduct.geom.traces)}

\begin{fulllineitems}
\phantomsection\label{aqueduct.geom.traces:aqueduct.geom.traces.LinearizeRecursive}\pysigline{\strong{class }\bfcode{LinearizeRecursive}}
Bases: \href{http://docs.python.org/2/library/functions.html\#object}{\code{object}}

Base class for linearization methods classes.

It implements recursive algorithm.
\index{here() (LinearizeRecursive method)}

\begin{fulllineitems}
\phantomsection\label{aqueduct.geom.traces:aqueduct.geom.traces.LinearizeRecursive.here}\pysiglinewithargsret{\bfcode{here}}{\emph{coords}, \emph{depth=0}}{}
Core of recursive linearization argorithm.

It checks if the first, the last and the middle point are linear acoording to the criterion. The middle point is selected a point that is in the middle of lenght of the paths made by input coordinates.

If these points are linear their indices are returned. Otherwise, coordinates are split into two parts. First part spans points from the first point to the middle point (inclusive) and the second parth spans points from the middle (inclusive) to the last point. Next, these two parts are submitted recursively to {\hyperref[aqueduct.geom.traces:aqueduct.geom.traces.LinearizeRecursive.here]{\code{here()}}}.
\begin{quote}

Results of these recursive calls are joined, redundant indecies are removed and sorted reslult is returned.
\end{quote}
\begin{quote}\begin{description}
\item[{Parameters}] \leavevmode\begin{itemize}
\item {} 
\textbf{coords} (\href{http://docs.scipy.org/doc/numpy/reference/generated/numpy.ndarray.html\#numpy.ndarray}{\emph{numpy.ndarray}}) -- Input coordinates.

\item {} 
\textbf{depth} (\href{http://docs.python.org/2/library/functions.html\#int}{\emph{int}}) -- Depth of recurence.

\end{itemize}

\item[{Returns}] \leavevmode
Indices of {\color{red}\bfseries{}:arg:{}`coords{}`} points that can be used instead of all points in visulatization.

\item[{Return type}] \leavevmode
list of int

\end{description}\end{quote}

\end{fulllineitems}

\index{\_\_call\_\_() (LinearizeRecursive method)}

\begin{fulllineitems}
\phantomsection\label{aqueduct.geom.traces:aqueduct.geom.traces.LinearizeRecursive.__call__}\pysiglinewithargsret{\bfcode{\_\_call\_\_}}{\emph{coords}}{}
\end{fulllineitems}


\end{fulllineitems}

\index{TrianlgeLinearize (class in aqueduct.geom.traces)}

\begin{fulllineitems}
\phantomsection\label{aqueduct.geom.traces:aqueduct.geom.traces.TrianlgeLinearize}\pysiglinewithargsret{\strong{class }\bfcode{TrianlgeLinearize}}{\emph{treshold}}{}
Bases: \href{http://docs.python.org/2/library/functions.html\#object}{\code{object}}
\index{\_\_init\_\_() (TrianlgeLinearize method)}

\begin{fulllineitems}
\phantomsection\label{aqueduct.geom.traces:aqueduct.geom.traces.TrianlgeLinearize.__init__}\pysiglinewithargsret{\bfcode{\_\_init\_\_}}{\emph{treshold}}{}
\end{fulllineitems}

\index{is\_linear() (TrianlgeLinearize method)}

\begin{fulllineitems}
\phantomsection\label{aqueduct.geom.traces:aqueduct.geom.traces.TrianlgeLinearize.is_linear}\pysiglinewithargsret{\bfcode{is\_linear}}{\emph{coords}, \emph{**kwargs}}{}
\end{fulllineitems}


\end{fulllineitems}

\index{VectorLinearize (class in aqueduct.geom.traces)}

\begin{fulllineitems}
\phantomsection\label{aqueduct.geom.traces:aqueduct.geom.traces.VectorLinearize}\pysiglinewithargsret{\strong{class }\bfcode{VectorLinearize}}{\emph{treshold}}{}
Bases: \href{http://docs.python.org/2/library/functions.html\#object}{\code{object}}

Base class for linearization methods classes.

It implements vectro linearization criterion.
\index{\_\_init\_\_() (VectorLinearize method)}

\begin{fulllineitems}
\phantomsection\label{aqueduct.geom.traces:aqueduct.geom.traces.VectorLinearize.__init__}\pysiglinewithargsret{\bfcode{\_\_init\_\_}}{\emph{treshold}}{}
\end{fulllineitems}

\index{is\_linear\_core() (VectorLinearize method)}

\begin{fulllineitems}
\phantomsection\label{aqueduct.geom.traces:aqueduct.geom.traces.VectorLinearize.is_linear_core}\pysiglinewithargsret{\bfcode{is\_linear\_core}}{\emph{coords}, \emph{depth=None}}{}
Method checks if input coordinates are linear acoording to the threshold and depth.

It begins with calculation of the threshold. If \emph{depth} is None it is set to 1. Current treshold is calculated with following simple equation:
\begin{gather}
\begin{split}treshold_{current} = treshold_{initial} * (2 - 0.9^{depth})\end{split}\notag
\end{gather}
Next, in a loop over all points but the first and the last the angle is calculated between two vectors. The first one made by the point and the firs point, and the second vector made by the last and the first point. If any of the calculated angles is bigger the the treshold methods returns False; otherwise method returns True.
\begin{quote}\begin{description}
\item[{Parameters}] \leavevmode\begin{itemize}
\item {} 
\textbf{coords} (\href{http://docs.scipy.org/doc/numpy/reference/generated/numpy.ndarray.html\#numpy.ndarray}{\emph{numpy.ndarray}}) -- Coodrdinates for which linearizetion criterion is checked.

\item {} 
\textbf{depth} (\href{http://docs.python.org/2/library/functions.html\#int}{\emph{int}}) -- Depth of recurence.

\end{itemize}

\item[{Returns}] \leavevmode
True if input coordinates are linear and False otherwise.

\item[{Return type}] \leavevmode
bool

\end{description}\end{quote}

\end{fulllineitems}

\index{is\_linear() (VectorLinearize method)}

\begin{fulllineitems}
\phantomsection\label{aqueduct.geom.traces:aqueduct.geom.traces.VectorLinearize.is_linear}\pysiglinewithargsret{\bfcode{is\_linear}}{\emph{coords}, \emph{depth=None}, \emph{**kwargs}}{}
For more detail see {\hyperref[aqueduct.geom.traces:aqueduct.geom.traces.VectorLinearize.is_linear_core]{\code{is\_linear\_core()}}} which is used as the criterion of linearity in this method.
\begin{quote}\begin{description}
\item[{Parameters}] \leavevmode\begin{itemize}
\item {} 
\textbf{coords} (\href{http://docs.scipy.org/doc/numpy/reference/generated/numpy.ndarray.html\#numpy.ndarray}{\emph{numpy.ndarray}}) -- Coodrdinates for which linearizetion criterion is checked.

\item {} 
\textbf{depth} (\href{http://docs.python.org/2/library/functions.html\#int}{\emph{int}}) -- Depth of recurence.

\end{itemize}

\item[{Returns}] \leavevmode
True if input coordinates are linear and False otherwise. Criterion is checked for coordinates in normal and reverse order.

\item[{Return type}] \leavevmode
bool

\end{description}\end{quote}

\end{fulllineitems}


\end{fulllineitems}

\index{LinearizeRecursiveVector (class in aqueduct.geom.traces)}

\begin{fulllineitems}
\phantomsection\label{aqueduct.geom.traces:aqueduct.geom.traces.LinearizeRecursiveVector}\pysiglinewithargsret{\strong{class }\bfcode{LinearizeRecursiveVector}}{\emph{treshold}}{}
Bases: {\hyperref[aqueduct.geom.traces:aqueduct.geom.traces.LinearizeRecursive]{\code{aqueduct.geom.traces.LinearizeRecursive}}}, {\hyperref[aqueduct.geom.traces:aqueduct.geom.traces.VectorLinearize]{\code{aqueduct.geom.traces.VectorLinearize}}}
\phantomsection\label{aqueduct.geom.traces:simply-smooths-details}
Class provides recursive linearization of coordinates with {\hyperref[aqueduct.geom.traces:aqueduct.geom.traces.LinearizeRecursive]{\code{LinearizeRecursive}}} algorithm and the criterion of linearity implemented by {\hyperref[aqueduct.geom.traces:aqueduct.geom.traces.VectorLinearize]{\code{VectorLinearize}}}.

\end{fulllineitems}

\index{diff() (in module aqueduct.geom.traces)}

\begin{fulllineitems}
\phantomsection\label{aqueduct.geom.traces:aqueduct.geom.traces.diff}\pysiglinewithargsret{\bfcode{diff}}{\emph{trace}}{}
\end{fulllineitems}

\index{tracepoints() (in module aqueduct.geom.traces)}

\begin{fulllineitems}
\phantomsection\label{aqueduct.geom.traces:aqueduct.geom.traces.tracepoints}\pysiglinewithargsret{\bfcode{tracepoints}}{\emph{start}, \emph{stop}, \emph{nr}}{}
\end{fulllineitems}

\index{midpoints() (in module aqueduct.geom.traces)}

\begin{fulllineitems}
\phantomsection\label{aqueduct.geom.traces:aqueduct.geom.traces.midpoints}\pysiglinewithargsret{\bfcode{midpoints}}{\emph{paths}}{}
\end{fulllineitems}

\index{length\_step\_std() (in module aqueduct.geom.traces)}

\begin{fulllineitems}
\phantomsection\label{aqueduct.geom.traces:aqueduct.geom.traces.length_step_std}\pysiglinewithargsret{\bfcode{length\_step\_std}}{\emph{trace}}{}
\end{fulllineitems}

\index{derrivative() (in module aqueduct.geom.traces)}

\begin{fulllineitems}
\phantomsection\label{aqueduct.geom.traces:aqueduct.geom.traces.derrivative}\pysiglinewithargsret{\bfcode{derrivative}}{\emph{values}}{}
\end{fulllineitems}



\paragraph{Module contents}
\label{aqueduct.geom:module-contents}\label{aqueduct.geom:module-aqueduct.geom}\index{aqueduct.geom (module)}

\subsubsection{aqueduct.traj package}
\label{aqueduct.traj:aqueduct-traj-package}\label{aqueduct.traj::doc}

\paragraph{Submodules}
\label{aqueduct.traj:submodules}

\subparagraph{aqueduct.traj.dumps module}
\label{aqueduct.traj.dumps:module-aqueduct.traj.dumps}\label{aqueduct.traj.dumps::doc}\label{aqueduct.traj.dumps:aqueduct-traj-dumps-module}\index{aqueduct.traj.dumps (module)}\index{TmpDumpWriterOfMDA (class in aqueduct.traj.dumps)}

\begin{fulllineitems}
\phantomsection\label{aqueduct.traj.dumps:aqueduct.traj.dumps.TmpDumpWriterOfMDA}\pysigline{\strong{class }\bfcode{TmpDumpWriterOfMDA}}
Bases: \href{http://docs.python.org/2/library/functions.html\#object}{\code{object}}
\index{\_\_init\_\_() (TmpDumpWriterOfMDA method)}

\begin{fulllineitems}
\phantomsection\label{aqueduct.traj.dumps:aqueduct.traj.dumps.TmpDumpWriterOfMDA.__init__}\pysiglinewithargsret{\bfcode{\_\_init\_\_}}{}{}
\end{fulllineitems}

\index{dump\_frames() (TmpDumpWriterOfMDA method)}

\begin{fulllineitems}
\phantomsection\label{aqueduct.traj.dumps:aqueduct.traj.dumps.TmpDumpWriterOfMDA.dump_frames}\pysiglinewithargsret{\bfcode{dump\_frames}}{\emph{reader}, \emph{frames}, \emph{selection='protein'}}{}
\end{fulllineitems}

\index{close() (TmpDumpWriterOfMDA method)}

\begin{fulllineitems}
\phantomsection\label{aqueduct.traj.dumps:aqueduct.traj.dumps.TmpDumpWriterOfMDA.close}\pysiglinewithargsret{\bfcode{close}}{}{}
\end{fulllineitems}

\index{\_\_del\_\_() (TmpDumpWriterOfMDA method)}

\begin{fulllineitems}
\phantomsection\label{aqueduct.traj.dumps:aqueduct.traj.dumps.TmpDumpWriterOfMDA.__del__}\pysiglinewithargsret{\bfcode{\_\_del\_\_}}{}{}
\end{fulllineitems}


\end{fulllineitems}



\subparagraph{aqueduct.traj.inlets module}
\label{aqueduct.traj.inlets:module-aqueduct.traj.inlets}\label{aqueduct.traj.inlets:aqueduct-traj-inlets-module}\label{aqueduct.traj.inlets::doc}\index{aqueduct.traj.inlets (module)}\index{ProtoInletTypeCodes (class in aqueduct.traj.inlets)}

\begin{fulllineitems}
\phantomsection\label{aqueduct.traj.inlets:aqueduct.traj.inlets.ProtoInletTypeCodes}\pysigline{\strong{class }\bfcode{ProtoInletTypeCodes}}~\index{surface (ProtoInletTypeCodes attribute)}

\begin{fulllineitems}
\phantomsection\label{aqueduct.traj.inlets:aqueduct.traj.inlets.ProtoInletTypeCodes.surface}\pysigline{\bfcode{surface}\strong{ = `surface'}}
\end{fulllineitems}

\index{internal (ProtoInletTypeCodes attribute)}

\begin{fulllineitems}
\phantomsection\label{aqueduct.traj.inlets:aqueduct.traj.inlets.ProtoInletTypeCodes.internal}\pysigline{\bfcode{internal}\strong{ = `internal'}}
\end{fulllineitems}

\index{incoming (ProtoInletTypeCodes attribute)}

\begin{fulllineitems}
\phantomsection\label{aqueduct.traj.inlets:aqueduct.traj.inlets.ProtoInletTypeCodes.incoming}\pysigline{\bfcode{incoming}\strong{ = `inin'}}
\end{fulllineitems}

\index{outgoing (ProtoInletTypeCodes attribute)}

\begin{fulllineitems}
\phantomsection\label{aqueduct.traj.inlets:aqueduct.traj.inlets.ProtoInletTypeCodes.outgoing}\pysigline{\bfcode{outgoing}\strong{ = `inout'}}
\end{fulllineitems}


\end{fulllineitems}

\index{InletTypeCodes (class in aqueduct.traj.inlets)}

\begin{fulllineitems}
\phantomsection\label{aqueduct.traj.inlets:aqueduct.traj.inlets.InletTypeCodes}\pysigline{\strong{class }\bfcode{InletTypeCodes}}
Bases: {\hyperref[aqueduct.traj.inlets:aqueduct.traj.inlets.ProtoInletTypeCodes]{\code{aqueduct.traj.inlets.ProtoInletTypeCodes}}}
\index{all\_surface (InletTypeCodes attribute)}

\begin{fulllineitems}
\phantomsection\label{aqueduct.traj.inlets:aqueduct.traj.inlets.InletTypeCodes.all_surface}\pysigline{\bfcode{all\_surface}\strong{ = {[}(`surface', `inin'), (`surface', `inout'){]}}}
\end{fulllineitems}

\index{all\_internal (InletTypeCodes attribute)}

\begin{fulllineitems}
\phantomsection\label{aqueduct.traj.inlets:aqueduct.traj.inlets.InletTypeCodes.all_internal}\pysigline{\bfcode{all\_internal}\strong{ = {[}(`internal', `inin'), (`internal', `inout'){]}}}
\end{fulllineitems}

\index{all\_incoming (InletTypeCodes attribute)}

\begin{fulllineitems}
\phantomsection\label{aqueduct.traj.inlets:aqueduct.traj.inlets.InletTypeCodes.all_incoming}\pysigline{\bfcode{all\_incoming}\strong{ = {[}(`surface', `inin'), (`internal', `inin'){]}}}
\end{fulllineitems}

\index{all\_outgoing (InletTypeCodes attribute)}

\begin{fulllineitems}
\phantomsection\label{aqueduct.traj.inlets:aqueduct.traj.inlets.InletTypeCodes.all_outgoing}\pysigline{\bfcode{all\_outgoing}\strong{ = {[}(`surface', `inout'), (`internal', `inout'){]}}}
\end{fulllineitems}

\index{surface\_incoming (InletTypeCodes attribute)}

\begin{fulllineitems}
\phantomsection\label{aqueduct.traj.inlets:aqueduct.traj.inlets.InletTypeCodes.surface_incoming}\pysigline{\bfcode{surface\_incoming}\strong{ = (`surface', `inin')}}
\end{fulllineitems}

\index{internal\_incoming (InletTypeCodes attribute)}

\begin{fulllineitems}
\phantomsection\label{aqueduct.traj.inlets:aqueduct.traj.inlets.InletTypeCodes.internal_incoming}\pysigline{\bfcode{internal\_incoming}\strong{ = (`internal', `inin')}}
\end{fulllineitems}

\index{internal\_outgoing (InletTypeCodes attribute)}

\begin{fulllineitems}
\phantomsection\label{aqueduct.traj.inlets:aqueduct.traj.inlets.InletTypeCodes.internal_outgoing}\pysigline{\bfcode{internal\_outgoing}\strong{ = (`internal', `inout')}}
\end{fulllineitems}

\index{surface\_outgoing (InletTypeCodes attribute)}

\begin{fulllineitems}
\phantomsection\label{aqueduct.traj.inlets:aqueduct.traj.inlets.InletTypeCodes.surface_outgoing}\pysigline{\bfcode{surface\_outgoing}\strong{ = (`surface', `inout')}}
\end{fulllineitems}

\index{itype (InletTypeCodes attribute)}

\begin{fulllineitems}
\phantomsection\label{aqueduct.traj.inlets:aqueduct.traj.inlets.InletTypeCodes.itype}\pysigline{\bfcode{itype}\strong{ = `internal'}}
\end{fulllineitems}


\end{fulllineitems}

\index{InletClusterGenericType (class in aqueduct.traj.inlets)}

\begin{fulllineitems}
\phantomsection\label{aqueduct.traj.inlets:aqueduct.traj.inlets.InletClusterGenericType}\pysiglinewithargsret{\strong{class }\bfcode{InletClusterGenericType}}{\emph{inp}, \emph{out}}{}
Bases: \href{http://docs.python.org/2/library/functions.html\#object}{\code{object}}
\index{\_\_init\_\_() (InletClusterGenericType method)}

\begin{fulllineitems}
\phantomsection\label{aqueduct.traj.inlets:aqueduct.traj.inlets.InletClusterGenericType.__init__}\pysiglinewithargsret{\bfcode{\_\_init\_\_}}{\emph{inp}, \emph{out}}{}
\end{fulllineitems}

\index{input (InletClusterGenericType attribute)}

\begin{fulllineitems}
\phantomsection\label{aqueduct.traj.inlets:aqueduct.traj.inlets.InletClusterGenericType.input}\pysigline{\bfcode{input}}
\end{fulllineitems}

\index{output (InletClusterGenericType attribute)}

\begin{fulllineitems}
\phantomsection\label{aqueduct.traj.inlets:aqueduct.traj.inlets.InletClusterGenericType.output}\pysigline{\bfcode{output}}
\end{fulllineitems}

\index{cluster2str() (InletClusterGenericType method)}

\begin{fulllineitems}
\phantomsection\label{aqueduct.traj.inlets:aqueduct.traj.inlets.InletClusterGenericType.cluster2str}\pysiglinewithargsret{\bfcode{cluster2str}}{\emph{cl}}{}
\end{fulllineitems}

\index{\_\_getitem\_\_() (InletClusterGenericType method)}

\begin{fulllineitems}
\phantomsection\label{aqueduct.traj.inlets:aqueduct.traj.inlets.InletClusterGenericType.__getitem__}\pysiglinewithargsret{\bfcode{\_\_getitem\_\_}}{\emph{item}}{}
\end{fulllineitems}

\index{\_\_len\_\_() (InletClusterGenericType method)}

\begin{fulllineitems}
\phantomsection\label{aqueduct.traj.inlets:aqueduct.traj.inlets.InletClusterGenericType.__len__}\pysiglinewithargsret{\bfcode{\_\_len\_\_}}{}{}
\end{fulllineitems}

\index{\_\_str\_\_() (InletClusterGenericType method)}

\begin{fulllineitems}
\phantomsection\label{aqueduct.traj.inlets:aqueduct.traj.inlets.InletClusterGenericType.__str__}\pysiglinewithargsret{\bfcode{\_\_str\_\_}}{}{}
\end{fulllineitems}

\index{\_\_repr\_\_() (InletClusterGenericType method)}

\begin{fulllineitems}
\phantomsection\label{aqueduct.traj.inlets:aqueduct.traj.inlets.InletClusterGenericType.__repr__}\pysiglinewithargsret{\bfcode{\_\_repr\_\_}}{}{}
\end{fulllineitems}

\index{make\_val() (InletClusterGenericType method)}

\begin{fulllineitems}
\phantomsection\label{aqueduct.traj.inlets:aqueduct.traj.inlets.InletClusterGenericType.make_val}\pysiglinewithargsret{\bfcode{make\_val}}{\emph{base}}{}
\end{fulllineitems}

\index{\_\_cmp\_\_() (InletClusterGenericType method)}

\begin{fulllineitems}
\phantomsection\label{aqueduct.traj.inlets:aqueduct.traj.inlets.InletClusterGenericType.__cmp__}\pysiglinewithargsret{\bfcode{\_\_cmp\_\_}}{\emph{other}}{}
\end{fulllineitems}

\index{\_\_hash\_\_() (InletClusterGenericType method)}

\begin{fulllineitems}
\phantomsection\label{aqueduct.traj.inlets:aqueduct.traj.inlets.InletClusterGenericType.__hash__}\pysiglinewithargsret{\bfcode{\_\_hash\_\_}}{}{}
\end{fulllineitems}


\end{fulllineitems}

\index{InletClusterExtendedType (class in aqueduct.traj.inlets)}

\begin{fulllineitems}
\phantomsection\label{aqueduct.traj.inlets:aqueduct.traj.inlets.InletClusterExtendedType}\pysiglinewithargsret{\strong{class }\bfcode{InletClusterExtendedType}}{\emph{surfin}, \emph{interin}, \emph{interout}, \emph{surfout}}{}
Bases: {\hyperref[aqueduct.traj.inlets:aqueduct.traj.inlets.InletClusterGenericType]{\code{aqueduct.traj.inlets.InletClusterGenericType}}}
\index{\_\_init\_\_() (InletClusterExtendedType method)}

\begin{fulllineitems}
\phantomsection\label{aqueduct.traj.inlets:aqueduct.traj.inlets.InletClusterExtendedType.__init__}\pysiglinewithargsret{\bfcode{\_\_init\_\_}}{\emph{surfin}, \emph{interin}, \emph{interout}, \emph{surfout}}{}
\end{fulllineitems}

\index{generic (InletClusterExtendedType attribute)}

\begin{fulllineitems}
\phantomsection\label{aqueduct.traj.inlets:aqueduct.traj.inlets.InletClusterExtendedType.generic}\pysigline{\bfcode{generic}}
\end{fulllineitems}


\end{fulllineitems}

\index{Inlet (class in aqueduct.traj.inlets)}

\begin{fulllineitems}
\phantomsection\label{aqueduct.traj.inlets:aqueduct.traj.inlets.Inlet}\pysigline{\strong{class }\bfcode{Inlet}}
Bases: \code{tuple}

Inlet(coords, type, reference)
\index{\_\_getnewargs\_\_() (Inlet method)}

\begin{fulllineitems}
\phantomsection\label{aqueduct.traj.inlets:aqueduct.traj.inlets.Inlet.__getnewargs__}\pysiglinewithargsret{\bfcode{\_\_getnewargs\_\_}}{}{}
Return self as a plain tuple.  Used by copy and pickle.

\end{fulllineitems}

\index{\_\_getstate\_\_() (Inlet method)}

\begin{fulllineitems}
\phantomsection\label{aqueduct.traj.inlets:aqueduct.traj.inlets.Inlet.__getstate__}\pysiglinewithargsret{\bfcode{\_\_getstate\_\_}}{}{}
Exclude the OrderedDict from pickling

\end{fulllineitems}

\index{\_\_new\_\_() (Inlet static method)}

\begin{fulllineitems}
\phantomsection\label{aqueduct.traj.inlets:aqueduct.traj.inlets.Inlet.__new__}\pysiglinewithargsret{\strong{static }\bfcode{\_\_new\_\_}}{\emph{\_cls}, \emph{coords}, \emph{type}, \emph{reference}}{}
Create new instance of Inlet(coords, type, reference)

\end{fulllineitems}

\index{\_\_repr\_\_() (Inlet method)}

\begin{fulllineitems}
\phantomsection\label{aqueduct.traj.inlets:aqueduct.traj.inlets.Inlet.__repr__}\pysiglinewithargsret{\bfcode{\_\_repr\_\_}}{}{}
Return a nicely formatted representation string

\end{fulllineitems}

\index{\_\_slots\_\_ (Inlet attribute)}

\begin{fulllineitems}
\phantomsection\label{aqueduct.traj.inlets:aqueduct.traj.inlets.Inlet.__slots__}\pysigline{\bfcode{\_\_slots\_\_}\strong{ = ()}}
\end{fulllineitems}

\index{\_asdict() (Inlet method)}

\begin{fulllineitems}
\phantomsection\label{aqueduct.traj.inlets:aqueduct.traj.inlets.Inlet._asdict}\pysiglinewithargsret{\bfcode{\_asdict}}{}{}
Return a new OrderedDict which maps field names to their values

\end{fulllineitems}

\index{\_fields (Inlet attribute)}

\begin{fulllineitems}
\phantomsection\label{aqueduct.traj.inlets:aqueduct.traj.inlets.Inlet._fields}\pysigline{\bfcode{\_fields}\strong{ = (`coords', `type', `reference')}}
\end{fulllineitems}

\index{\_make() (aqueduct.traj.inlets.Inlet class method)}

\begin{fulllineitems}
\phantomsection\label{aqueduct.traj.inlets:aqueduct.traj.inlets.Inlet._make}\pysiglinewithargsret{\strong{classmethod }\bfcode{\_make}}{\emph{iterable}, \emph{new=\textless{}built-in method \_\_new\_\_ of type object at 0x9192c0\textgreater{}}, \emph{len=\textless{}built-in function len\textgreater{}}}{}
Make a new Inlet object from a sequence or iterable

\end{fulllineitems}

\index{\_replace() (Inlet method)}

\begin{fulllineitems}
\phantomsection\label{aqueduct.traj.inlets:aqueduct.traj.inlets.Inlet._replace}\pysiglinewithargsret{\bfcode{\_replace}}{\emph{\_self}, \emph{**kwds}}{}
Return a new Inlet object replacing specified fields with new values

\end{fulllineitems}

\index{coords (Inlet attribute)}

\begin{fulllineitems}
\phantomsection\label{aqueduct.traj.inlets:aqueduct.traj.inlets.Inlet.coords}\pysigline{\bfcode{coords}}
Alias for field number 0

\end{fulllineitems}

\index{reference (Inlet attribute)}

\begin{fulllineitems}
\phantomsection\label{aqueduct.traj.inlets:aqueduct.traj.inlets.Inlet.reference}\pysigline{\bfcode{reference}}
Alias for field number 2

\end{fulllineitems}

\index{type (Inlet attribute)}

\begin{fulllineitems}
\phantomsection\label{aqueduct.traj.inlets:aqueduct.traj.inlets.Inlet.type}\pysigline{\bfcode{type}}
Alias for field number 1

\end{fulllineitems}


\end{fulllineitems}

\index{Inlets (class in aqueduct.traj.inlets)}

\begin{fulllineitems}
\phantomsection\label{aqueduct.traj.inlets:aqueduct.traj.inlets.Inlets}\pysiglinewithargsret{\strong{class }\bfcode{Inlets}}{\emph{spaths, onlytype={[}(`surface', `inin'), (`surface', `inout'){]}}}{}
Bases: \href{http://docs.python.org/2/library/functions.html\#object}{\code{object}}
\index{\_\_init\_\_() (Inlets method)}

\begin{fulllineitems}
\phantomsection\label{aqueduct.traj.inlets:aqueduct.traj.inlets.Inlets.__init__}\pysiglinewithargsret{\bfcode{\_\_init\_\_}}{\emph{spaths, onlytype={[}(`surface', `inin'), (`surface', `inout'){]}}}{}
\end{fulllineitems}

\index{extend\_inlets() (Inlets method)}

\begin{fulllineitems}
\phantomsection\label{aqueduct.traj.inlets:aqueduct.traj.inlets.Inlets.extend_inlets}\pysiglinewithargsret{\bfcode{extend\_inlets}}{\emph{spath}, \emph{onlytype=None}}{}
\end{fulllineitems}

\index{add\_cluster\_annotations() (Inlets method)}

\begin{fulllineitems}
\phantomsection\label{aqueduct.traj.inlets:aqueduct.traj.inlets.Inlets.add_cluster_annotations}\pysiglinewithargsret{\bfcode{add\_cluster\_annotations}}{\emph{clusters}}{}
\end{fulllineitems}

\index{size (Inlets attribute)}

\begin{fulllineitems}
\phantomsection\label{aqueduct.traj.inlets:aqueduct.traj.inlets.Inlets.size}\pysigline{\bfcode{size}}
\end{fulllineitems}

\index{coords (Inlets attribute)}

\begin{fulllineitems}
\phantomsection\label{aqueduct.traj.inlets:aqueduct.traj.inlets.Inlets.coords}\pysigline{\bfcode{coords}}
\end{fulllineitems}

\index{types (Inlets attribute)}

\begin{fulllineitems}
\phantomsection\label{aqueduct.traj.inlets:aqueduct.traj.inlets.Inlets.types}\pysigline{\bfcode{types}}
\end{fulllineitems}

\index{refs (Inlets attribute)}

\begin{fulllineitems}
\phantomsection\label{aqueduct.traj.inlets:aqueduct.traj.inlets.Inlets.refs}\pysigline{\bfcode{refs}}
\end{fulllineitems}

\index{perform\_clustering() (Inlets method)}

\begin{fulllineitems}
\phantomsection\label{aqueduct.traj.inlets:aqueduct.traj.inlets.Inlets.perform_clustering}\pysiglinewithargsret{\bfcode{perform\_clustering}}{\emph{method}}{}
\end{fulllineitems}

\index{recluster\_outliers() (Inlets method)}

\begin{fulllineitems}
\phantomsection\label{aqueduct.traj.inlets:aqueduct.traj.inlets.Inlets.recluster_outliers}\pysiglinewithargsret{\bfcode{recluster\_outliers}}{\emph{method}}{}
\end{fulllineitems}

\index{small\_clusters\_to\_outliers() (Inlets method)}

\begin{fulllineitems}
\phantomsection\label{aqueduct.traj.inlets:aqueduct.traj.inlets.Inlets.small_clusters_to_outliers}\pysiglinewithargsret{\bfcode{small\_clusters\_to\_outliers}}{\emph{maxsize}}{}
\end{fulllineitems}

\index{renumber\_clusters() (Inlets method)}

\begin{fulllineitems}
\phantomsection\label{aqueduct.traj.inlets:aqueduct.traj.inlets.Inlets.renumber_clusters}\pysiglinewithargsret{\bfcode{renumber\_clusters}}{}{}
\end{fulllineitems}

\index{sort\_clusters() (Inlets method)}

\begin{fulllineitems}
\phantomsection\label{aqueduct.traj.inlets:aqueduct.traj.inlets.Inlets.sort_clusters}\pysiglinewithargsret{\bfcode{sort\_clusters}}{}{}
\end{fulllineitems}

\index{clusters\_list (Inlets attribute)}

\begin{fulllineitems}
\phantomsection\label{aqueduct.traj.inlets:aqueduct.traj.inlets.Inlets.clusters_list}\pysigline{\bfcode{clusters\_list}}
\end{fulllineitems}

\index{clusters\_centers (Inlets attribute)}

\begin{fulllineitems}
\phantomsection\label{aqueduct.traj.inlets:aqueduct.traj.inlets.Inlets.clusters_centers}\pysigline{\bfcode{clusters\_centers}}
\end{fulllineitems}

\index{clusters\_size (Inlets attribute)}

\begin{fulllineitems}
\phantomsection\label{aqueduct.traj.inlets:aqueduct.traj.inlets.Inlets.clusters_size}\pysigline{\bfcode{clusters\_size}}
\end{fulllineitems}

\index{clusters\_std (Inlets attribute)}

\begin{fulllineitems}
\phantomsection\label{aqueduct.traj.inlets:aqueduct.traj.inlets.Inlets.clusters_std}\pysigline{\bfcode{clusters\_std}}
\end{fulllineitems}

\index{spaths2ctypes() (Inlets method)}

\begin{fulllineitems}
\phantomsection\label{aqueduct.traj.inlets:aqueduct.traj.inlets.Inlets.spaths2ctypes}\pysiglinewithargsret{\bfcode{spaths2ctypes}}{\emph{*args}, \emph{**kwargs}}{}
\end{fulllineitems}

\index{lim\_to() (Inlets method)}

\begin{fulllineitems}
\phantomsection\label{aqueduct.traj.inlets:aqueduct.traj.inlets.Inlets.lim_to}\pysiglinewithargsret{\bfcode{lim\_to}}{\emph{what}, \emph{towhat}}{}
\end{fulllineitems}

\index{lim2spaths() (Inlets method)}

\begin{fulllineitems}
\phantomsection\label{aqueduct.traj.inlets:aqueduct.traj.inlets.Inlets.lim2spaths}\pysiglinewithargsret{\bfcode{lim2spaths}}{\emph{spaths}}{}
\end{fulllineitems}

\index{lim2types() (Inlets method)}

\begin{fulllineitems}
\phantomsection\label{aqueduct.traj.inlets:aqueduct.traj.inlets.Inlets.lim2types}\pysiglinewithargsret{\bfcode{lim2types}}{\emph{types}}{}
\end{fulllineitems}

\index{lim2clusters() (Inlets method)}

\begin{fulllineitems}
\phantomsection\label{aqueduct.traj.inlets:aqueduct.traj.inlets.Inlets.lim2clusters}\pysiglinewithargsret{\bfcode{lim2clusters}}{\emph{clusters}}{}
\end{fulllineitems}

\index{limspaths2() (Inlets method)}

\begin{fulllineitems}
\phantomsection\label{aqueduct.traj.inlets:aqueduct.traj.inlets.Inlets.limspaths2}\pysiglinewithargsret{\bfcode{limspaths2}}{\emph{*args}, \emph{**kwargs}}{}
\end{fulllineitems}


\end{fulllineitems}



\subparagraph{aqueduct.traj.paths module}
\label{aqueduct.traj.paths:aqueduct-traj-paths-module}\label{aqueduct.traj.paths:module-aqueduct.traj.paths}\label{aqueduct.traj.paths::doc}\index{aqueduct.traj.paths (module)}
Created on Dec 10, 2015

@author: tljm
\index{union() (in module aqueduct.traj.paths)}

\begin{fulllineitems}
\phantomsection\label{aqueduct.traj.paths:aqueduct.traj.paths.union}\pysiglinewithargsret{\bfcode{union}}{\emph{a}, \emph{b}}{}
\end{fulllineitems}

\index{glue() (in module aqueduct.traj.paths)}

\begin{fulllineitems}
\phantomsection\label{aqueduct.traj.paths:aqueduct.traj.paths.glue}\pysiglinewithargsret{\bfcode{glue}}{\emph{a}, \emph{b}}{}
\end{fulllineitems}

\index{xor() (in module aqueduct.traj.paths)}

\begin{fulllineitems}
\phantomsection\label{aqueduct.traj.paths:aqueduct.traj.paths.xor}\pysiglinewithargsret{\bfcode{xor}}{\emph{*args}, \emph{**kwargs}}{}
\end{fulllineitems}

\index{left() (in module aqueduct.traj.paths)}

\begin{fulllineitems}
\phantomsection\label{aqueduct.traj.paths:aqueduct.traj.paths.left}\pysiglinewithargsret{\bfcode{left}}{\emph{a}, \emph{b}}{}
\end{fulllineitems}

\index{right() (in module aqueduct.traj.paths)}

\begin{fulllineitems}
\phantomsection\label{aqueduct.traj.paths:aqueduct.traj.paths.right}\pysiglinewithargsret{\bfcode{right}}{\emph{a}, \emph{b}}{}
\end{fulllineitems}

\index{PathTypesCodes (class in aqueduct.traj.paths)}

\begin{fulllineitems}
\phantomsection\label{aqueduct.traj.paths:aqueduct.traj.paths.PathTypesCodes}\pysigline{\strong{class }\bfcode{PathTypesCodes}}~\index{path\_in\_code (PathTypesCodes attribute)}

\begin{fulllineitems}
\phantomsection\label{aqueduct.traj.paths:aqueduct.traj.paths.PathTypesCodes.path_in_code}\pysigline{\bfcode{path\_in\_code}\strong{ = `i'}}
\end{fulllineitems}

\index{path\_object\_code (PathTypesCodes attribute)}

\begin{fulllineitems}
\phantomsection\label{aqueduct.traj.paths:aqueduct.traj.paths.PathTypesCodes.path_object_code}\pysigline{\bfcode{path\_object\_code}\strong{ = `c'}}
\end{fulllineitems}

\index{path\_out\_code (PathTypesCodes attribute)}

\begin{fulllineitems}
\phantomsection\label{aqueduct.traj.paths:aqueduct.traj.paths.PathTypesCodes.path_out_code}\pysigline{\bfcode{path\_out\_code}\strong{ = `o'}}
\end{fulllineitems}


\end{fulllineitems}

\index{GenericPathTypeCodes (class in aqueduct.traj.paths)}

\begin{fulllineitems}
\phantomsection\label{aqueduct.traj.paths:aqueduct.traj.paths.GenericPathTypeCodes}\pysigline{\strong{class }\bfcode{GenericPathTypeCodes}}~\index{object\_name (GenericPathTypeCodes attribute)}

\begin{fulllineitems}
\phantomsection\label{aqueduct.traj.paths:aqueduct.traj.paths.GenericPathTypeCodes.object_name}\pysigline{\bfcode{object\_name}\strong{ = `c'}}
\end{fulllineitems}

\index{scope\_name (GenericPathTypeCodes attribute)}

\begin{fulllineitems}
\phantomsection\label{aqueduct.traj.paths:aqueduct.traj.paths.GenericPathTypeCodes.scope_name}\pysigline{\bfcode{scope\_name}\strong{ = `s'}}
\end{fulllineitems}

\index{out\_name (GenericPathTypeCodes attribute)}

\begin{fulllineitems}
\phantomsection\label{aqueduct.traj.paths:aqueduct.traj.paths.GenericPathTypeCodes.out_name}\pysigline{\bfcode{out\_name}\strong{ = `n'}}
\end{fulllineitems}


\end{fulllineitems}

\index{GenericPaths (class in aqueduct.traj.paths)}

\begin{fulllineitems}
\phantomsection\label{aqueduct.traj.paths:aqueduct.traj.paths.GenericPaths}\pysiglinewithargsret{\strong{class }\bfcode{GenericPaths}}{\emph{id}, \emph{min\_pf=None}, \emph{max\_pf=None}}{}
Bases: \href{http://docs.python.org/2/library/functions.html\#object}{\code{object}}, {\hyperref[aqueduct.traj.paths:aqueduct.traj.paths.GenericPathTypeCodes]{\code{aqueduct.traj.paths.GenericPathTypeCodes}}}
\index{\_\_init\_\_() (GenericPaths method)}

\begin{fulllineitems}
\phantomsection\label{aqueduct.traj.paths:aqueduct.traj.paths.GenericPaths.__init__}\pysiglinewithargsret{\bfcode{\_\_init\_\_}}{\emph{id}, \emph{min\_pf=None}, \emph{max\_pf=None}}{}
\end{fulllineitems}

\index{add\_coord() (GenericPaths method)}

\begin{fulllineitems}
\phantomsection\label{aqueduct.traj.paths:aqueduct.traj.paths.GenericPaths.add_coord}\pysiglinewithargsret{\bfcode{add\_coord}}{\emph{coord}}{}
\end{fulllineitems}

\index{add\_object() (GenericPaths method)}

\begin{fulllineitems}
\phantomsection\label{aqueduct.traj.paths:aqueduct.traj.paths.GenericPaths.add_object}\pysiglinewithargsret{\bfcode{add\_object}}{\emph{frame}}{}
\end{fulllineitems}

\index{add\_scope() (GenericPaths method)}

\begin{fulllineitems}
\phantomsection\label{aqueduct.traj.paths:aqueduct.traj.paths.GenericPaths.add_scope}\pysiglinewithargsret{\bfcode{add\_scope}}{\emph{frame}}{}
\end{fulllineitems}

\index{add\_type() (GenericPaths method)}

\begin{fulllineitems}
\phantomsection\label{aqueduct.traj.paths:aqueduct.traj.paths.GenericPaths.add_type}\pysiglinewithargsret{\bfcode{add\_type}}{\emph{frame}, \emph{ftype}}{}
\end{fulllineitems}

\index{max\_frame (GenericPaths attribute)}

\begin{fulllineitems}
\phantomsection\label{aqueduct.traj.paths:aqueduct.traj.paths.GenericPaths.max_frame}\pysigline{\bfcode{max\_frame}}
\end{fulllineitems}

\index{min\_frame (GenericPaths attribute)}

\begin{fulllineitems}
\phantomsection\label{aqueduct.traj.paths:aqueduct.traj.paths.GenericPaths.min_frame}\pysigline{\bfcode{min\_frame}}
\end{fulllineitems}

\index{get\_paths\_in() (GenericPaths method)}

\begin{fulllineitems}
\phantomsection\label{aqueduct.traj.paths:aqueduct.traj.paths.GenericPaths.get_paths_in}\pysiglinewithargsret{\bfcode{get\_paths\_in}}{}{}
\end{fulllineitems}

\index{get\_paths\_out() (GenericPaths method)}

\begin{fulllineitems}
\phantomsection\label{aqueduct.traj.paths:aqueduct.traj.paths.GenericPaths.get_paths_out}\pysiglinewithargsret{\bfcode{get\_paths\_out}}{}{}
\end{fulllineitems}

\index{get\_paths\_for\_frames\_range() (GenericPaths method)}

\begin{fulllineitems}
\phantomsection\label{aqueduct.traj.paths:aqueduct.traj.paths.GenericPaths.get_paths_for_frames_range}\pysiglinewithargsret{\bfcode{get\_paths\_for\_frames\_range}}{\emph{*args}, \emph{**kwargs}}{}
\end{fulllineitems}

\index{find\_paths() (GenericPaths method)}

\begin{fulllineitems}
\phantomsection\label{aqueduct.traj.paths:aqueduct.traj.paths.GenericPaths.find_paths}\pysiglinewithargsret{\bfcode{find\_paths}}{\emph{fullonly=False}}{}
\end{fulllineitems}

\index{find\_paths\_coords() (GenericPaths method)}

\begin{fulllineitems}
\phantomsection\label{aqueduct.traj.paths:aqueduct.traj.paths.GenericPaths.find_paths_coords}\pysiglinewithargsret{\bfcode{find\_paths\_coords}}{\emph{fullonly=False}}{}
\end{fulllineitems}

\index{find\_paths\_types() (GenericPaths method)}

\begin{fulllineitems}
\phantomsection\label{aqueduct.traj.paths:aqueduct.traj.paths.GenericPaths.find_paths_types}\pysiglinewithargsret{\bfcode{find\_paths\_types}}{\emph{fullonly=False}}{}
\end{fulllineitems}

\index{find\_paths\_coords\_types() (GenericPaths method)}

\begin{fulllineitems}
\phantomsection\label{aqueduct.traj.paths:aqueduct.traj.paths.GenericPaths.find_paths_coords_types}\pysiglinewithargsret{\bfcode{find\_paths\_coords\_types}}{\emph{fullonly=False}}{}
\end{fulllineitems}

\index{get\_single\_path\_coords() (GenericPaths method)}

\begin{fulllineitems}
\phantomsection\label{aqueduct.traj.paths:aqueduct.traj.paths.GenericPaths.get_single_path_coords}\pysiglinewithargsret{\bfcode{get\_single\_path\_coords}}{\emph{spath}}{}
\end{fulllineitems}

\index{get\_single\_path\_types() (GenericPaths method)}

\begin{fulllineitems}
\phantomsection\label{aqueduct.traj.paths:aqueduct.traj.paths.GenericPaths.get_single_path_types}\pysiglinewithargsret{\bfcode{get\_single\_path\_types}}{\emph{spath}}{}
\end{fulllineitems}


\end{fulllineitems}

\index{SinglePathID (class in aqueduct.traj.paths)}

\begin{fulllineitems}
\phantomsection\label{aqueduct.traj.paths:aqueduct.traj.paths.SinglePathID}\pysiglinewithargsret{\strong{class }\bfcode{SinglePathID}}{\emph{id=None}, \emph{nr=None}}{}
Bases: \href{http://docs.python.org/2/library/functions.html\#object}{\code{object}}
\index{\_\_init\_\_() (SinglePathID method)}

\begin{fulllineitems}
\phantomsection\label{aqueduct.traj.paths:aqueduct.traj.paths.SinglePathID.__init__}\pysiglinewithargsret{\bfcode{\_\_init\_\_}}{\emph{id=None}, \emph{nr=None}}{}
\end{fulllineitems}

\index{\_\_str\_\_() (SinglePathID method)}

\begin{fulllineitems}
\phantomsection\label{aqueduct.traj.paths:aqueduct.traj.paths.SinglePathID.__str__}\pysiglinewithargsret{\bfcode{\_\_str\_\_}}{}{}
\end{fulllineitems}


\end{fulllineitems}

\index{yield\_single\_paths() (in module aqueduct.traj.paths)}

\begin{fulllineitems}
\phantomsection\label{aqueduct.traj.paths:aqueduct.traj.paths.yield_single_paths}\pysiglinewithargsret{\bfcode{yield\_single\_paths}}{\emph{gps}, \emph{fullonly=False}, \emph{progress=False}}{}
\end{fulllineitems}

\index{SinglePath (class in aqueduct.traj.paths)}

\begin{fulllineitems}
\phantomsection\label{aqueduct.traj.paths:aqueduct.traj.paths.SinglePath}\pysiglinewithargsret{\strong{class }\bfcode{SinglePath}}{\emph{id}, \emph{paths}, \emph{coords}, \emph{types}}{}
Bases: \href{http://docs.python.org/2/library/functions.html\#object}{\code{object}}, {\hyperref[aqueduct.traj.paths:aqueduct.traj.paths.PathTypesCodes]{\code{aqueduct.traj.paths.PathTypesCodes}}}, {\hyperref[aqueduct.traj.inlets:aqueduct.traj.inlets.InletTypeCodes]{\code{aqueduct.traj.inlets.InletTypeCodes}}}
\index{empty\_coords (SinglePath attribute)}

\begin{fulllineitems}
\phantomsection\label{aqueduct.traj.paths:aqueduct.traj.paths.SinglePath.empty_coords}\pysigline{\bfcode{empty\_coords}\strong{ = array({[}{]}, shape=(0, 3), dtype=float64)}}
\end{fulllineitems}

\index{\_\_init\_\_() (SinglePath method)}

\begin{fulllineitems}
\phantomsection\label{aqueduct.traj.paths:aqueduct.traj.paths.SinglePath.__init__}\pysiglinewithargsret{\bfcode{\_\_init\_\_}}{\emph{id}, \emph{paths}, \emph{coords}, \emph{types}}{}
\end{fulllineitems}

\index{coords\_first\_in (SinglePath attribute)}

\begin{fulllineitems}
\phantomsection\label{aqueduct.traj.paths:aqueduct.traj.paths.SinglePath.coords_first_in}\pysigline{\bfcode{coords\_first\_in}}
\end{fulllineitems}

\index{coords\_last\_out (SinglePath attribute)}

\begin{fulllineitems}
\phantomsection\label{aqueduct.traj.paths:aqueduct.traj.paths.SinglePath.coords_last_out}\pysigline{\bfcode{coords\_last\_out}}
\end{fulllineitems}

\index{coords\_filo (SinglePath attribute)}

\begin{fulllineitems}
\phantomsection\label{aqueduct.traj.paths:aqueduct.traj.paths.SinglePath.coords_filo}\pysigline{\bfcode{coords\_filo}}
\end{fulllineitems}

\index{get\_inlets() (SinglePath method)}

\begin{fulllineitems}
\phantomsection\label{aqueduct.traj.paths:aqueduct.traj.paths.SinglePath.get_inlets}\pysiglinewithargsret{\bfcode{get\_inlets}}{}{}
\end{fulllineitems}

\index{coords (SinglePath attribute)}

\begin{fulllineitems}
\phantomsection\label{aqueduct.traj.paths:aqueduct.traj.paths.SinglePath.coords}\pysigline{\bfcode{coords}}
\end{fulllineitems}

\index{coords\_cont (SinglePath attribute)}

\begin{fulllineitems}
\phantomsection\label{aqueduct.traj.paths:aqueduct.traj.paths.SinglePath.coords_cont}\pysigline{\bfcode{coords\_cont}}
\end{fulllineitems}

\index{paths (SinglePath attribute)}

\begin{fulllineitems}
\phantomsection\label{aqueduct.traj.paths:aqueduct.traj.paths.SinglePath.paths}\pysigline{\bfcode{paths}}
\end{fulllineitems}

\index{paths\_cont (SinglePath attribute)}

\begin{fulllineitems}
\phantomsection\label{aqueduct.traj.paths:aqueduct.traj.paths.SinglePath.paths_cont}\pysigline{\bfcode{paths\_cont}}
\end{fulllineitems}

\index{types (SinglePath attribute)}

\begin{fulllineitems}
\phantomsection\label{aqueduct.traj.paths:aqueduct.traj.paths.SinglePath.types}\pysigline{\bfcode{types}}
\end{fulllineitems}

\index{types\_cont (SinglePath attribute)}

\begin{fulllineitems}
\phantomsection\label{aqueduct.traj.paths:aqueduct.traj.paths.SinglePath.types_cont}\pysigline{\bfcode{types\_cont}}
\end{fulllineitems}

\index{gtypes (SinglePath attribute)}

\begin{fulllineitems}
\phantomsection\label{aqueduct.traj.paths:aqueduct.traj.paths.SinglePath.gtypes}\pysigline{\bfcode{gtypes}}
\end{fulllineitems}

\index{gtypes\_cont (SinglePath attribute)}

\begin{fulllineitems}
\phantomsection\label{aqueduct.traj.paths:aqueduct.traj.paths.SinglePath.gtypes_cont}\pysigline{\bfcode{gtypes\_cont}}
\end{fulllineitems}

\index{etypes (SinglePath attribute)}

\begin{fulllineitems}
\phantomsection\label{aqueduct.traj.paths:aqueduct.traj.paths.SinglePath.etypes}\pysigline{\bfcode{etypes}}
\end{fulllineitems}

\index{etypes\_cont (SinglePath attribute)}

\begin{fulllineitems}
\phantomsection\label{aqueduct.traj.paths:aqueduct.traj.paths.SinglePath.etypes_cont}\pysigline{\bfcode{etypes\_cont}}
\end{fulllineitems}

\index{size (SinglePath attribute)}

\begin{fulllineitems}
\phantomsection\label{aqueduct.traj.paths:aqueduct.traj.paths.SinglePath.size}\pysigline{\bfcode{size}}
\end{fulllineitems}

\index{begins (SinglePath attribute)}

\begin{fulllineitems}
\phantomsection\label{aqueduct.traj.paths:aqueduct.traj.paths.SinglePath.begins}\pysigline{\bfcode{begins}}
\end{fulllineitems}

\index{ends (SinglePath attribute)}

\begin{fulllineitems}
\phantomsection\label{aqueduct.traj.paths:aqueduct.traj.paths.SinglePath.ends}\pysigline{\bfcode{ends}}
\end{fulllineitems}

\index{has\_in (SinglePath attribute)}

\begin{fulllineitems}
\phantomsection\label{aqueduct.traj.paths:aqueduct.traj.paths.SinglePath.has_in}\pysigline{\bfcode{has\_in}}
\end{fulllineitems}

\index{has\_object (SinglePath attribute)}

\begin{fulllineitems}
\phantomsection\label{aqueduct.traj.paths:aqueduct.traj.paths.SinglePath.has_object}\pysigline{\bfcode{has\_object}}
\end{fulllineitems}

\index{has\_out (SinglePath attribute)}

\begin{fulllineitems}
\phantomsection\label{aqueduct.traj.paths:aqueduct.traj.paths.SinglePath.has_out}\pysigline{\bfcode{has\_out}}
\end{fulllineitems}

\index{get\_coords() (SinglePath method)}

\begin{fulllineitems}
\phantomsection\label{aqueduct.traj.paths:aqueduct.traj.paths.SinglePath.get_coords}\pysiglinewithargsret{\bfcode{get\_coords}}{\emph{*args}, \emph{**kwargs}}{}
\end{fulllineitems}

\index{get\_coords\_cont() (SinglePath method)}

\begin{fulllineitems}
\phantomsection\label{aqueduct.traj.paths:aqueduct.traj.paths.SinglePath.get_coords_cont}\pysiglinewithargsret{\bfcode{get\_coords\_cont}}{\emph{smooth=None}}{}
\end{fulllineitems}

\index{\_make\_smooth\_coords() (SinglePath method)}

\begin{fulllineitems}
\phantomsection\label{aqueduct.traj.paths:aqueduct.traj.paths.SinglePath._make_smooth_coords}\pysiglinewithargsret{\bfcode{\_make\_smooth\_coords}}{\emph{*args}, \emph{**kwargs}}{}
\end{fulllineitems}

\index{apply\_smoothing() (SinglePath method)}

\begin{fulllineitems}
\phantomsection\label{aqueduct.traj.paths:aqueduct.traj.paths.SinglePath.apply_smoothing}\pysiglinewithargsret{\bfcode{apply\_smoothing}}{\emph{smooth}}{}
\end{fulllineitems}

\index{get\_distance\_cont() (SinglePath method)}

\begin{fulllineitems}
\phantomsection\label{aqueduct.traj.paths:aqueduct.traj.paths.SinglePath.get_distance_cont}\pysiglinewithargsret{\bfcode{get\_distance\_cont}}{\emph{smooth=None}, \emph{normalize=False}}{}
\end{fulllineitems}

\index{get\_distance\_rev\_cont() (SinglePath method)}

\begin{fulllineitems}
\phantomsection\label{aqueduct.traj.paths:aqueduct.traj.paths.SinglePath.get_distance_rev_cont}\pysiglinewithargsret{\bfcode{get\_distance\_rev\_cont}}{\emph{*args}, \emph{**kwargs}}{}
\end{fulllineitems}

\index{get\_distance\_both\_cont() (SinglePath method)}

\begin{fulllineitems}
\phantomsection\label{aqueduct.traj.paths:aqueduct.traj.paths.SinglePath.get_distance_both_cont}\pysiglinewithargsret{\bfcode{get\_distance\_both\_cont}}{\emph{*args}, \emph{**kwargs}}{}
\end{fulllineitems}

\index{get\_velocity\_cont() (SinglePath method)}

\begin{fulllineitems}
\phantomsection\label{aqueduct.traj.paths:aqueduct.traj.paths.SinglePath.get_velocity_cont}\pysiglinewithargsret{\bfcode{get\_velocity\_cont}}{\emph{*args}, \emph{**kwargs}}{}
\end{fulllineitems}

\index{get\_acceleration\_cont() (SinglePath method)}

\begin{fulllineitems}
\phantomsection\label{aqueduct.traj.paths:aqueduct.traj.paths.SinglePath.get_acceleration_cont}\pysiglinewithargsret{\bfcode{get\_acceleration\_cont}}{\emph{*args}, \emph{**kwargs}}{}
\end{fulllineitems}


\end{fulllineitems}

\index{MasterPath (class in aqueduct.traj.paths)}

\begin{fulllineitems}
\phantomsection\label{aqueduct.traj.paths:aqueduct.traj.paths.MasterPath}\pysiglinewithargsret{\strong{class }\bfcode{MasterPath}}{\emph{sp}}{}
Bases: {\hyperref[aqueduct.traj.paths:aqueduct.traj.paths.SinglePath]{\code{aqueduct.traj.paths.SinglePath}}}
\index{\_\_init\_\_() (MasterPath method)}

\begin{fulllineitems}
\phantomsection\label{aqueduct.traj.paths:aqueduct.traj.paths.MasterPath.__init__}\pysiglinewithargsret{\bfcode{\_\_init\_\_}}{\emph{sp}}{}
\end{fulllineitems}

\index{add\_width() (MasterPath method)}

\begin{fulllineitems}
\phantomsection\label{aqueduct.traj.paths:aqueduct.traj.paths.MasterPath.add_width}\pysiglinewithargsret{\bfcode{add\_width}}{\emph{width}}{}
\end{fulllineitems}


\end{fulllineitems}



\subparagraph{aqueduct.traj.reader module}
\label{aqueduct.traj.reader:aqueduct-traj-reader-module}\label{aqueduct.traj.reader::doc}\label{aqueduct.traj.reader:module-aqueduct.traj.reader}\index{aqueduct.traj.reader (module)}
Created on Nov 19, 2015

@author: tljm
\index{Reader (class in aqueduct.traj.reader)}

\begin{fulllineitems}
\phantomsection\label{aqueduct.traj.reader:aqueduct.traj.reader.Reader}\pysiglinewithargsret{\strong{class }\bfcode{Reader}}{\emph{topology}, \emph{trajectory}}{}
Bases: \href{http://docs.python.org/2/library/functions.html\#object}{\code{object}}
\index{\_\_init\_\_() (Reader method)}

\begin{fulllineitems}
\phantomsection\label{aqueduct.traj.reader:aqueduct.traj.reader.Reader.__init__}\pysiglinewithargsret{\bfcode{\_\_init\_\_}}{\emph{topology}, \emph{trajectory}}{}
\end{fulllineitems}

\index{open\_trajectory() (Reader method)}

\begin{fulllineitems}
\phantomsection\label{aqueduct.traj.reader:aqueduct.traj.reader.Reader.open_trajectory}\pysiglinewithargsret{\bfcode{open\_trajectory}}{}{}
\end{fulllineitems}

\index{number\_of\_frames (Reader attribute)}

\begin{fulllineitems}
\phantomsection\label{aqueduct.traj.reader:aqueduct.traj.reader.Reader.number_of_frames}\pysigline{\bfcode{number\_of\_frames}}
\end{fulllineitems}

\index{set\_current\_frame() (Reader method)}

\begin{fulllineitems}
\phantomsection\label{aqueduct.traj.reader:aqueduct.traj.reader.Reader.set_current_frame}\pysiglinewithargsret{\bfcode{set\_current\_frame}}{\emph{frame}}{}
\end{fulllineitems}

\index{next\_frame() (Reader method)}

\begin{fulllineitems}
\phantomsection\label{aqueduct.traj.reader:aqueduct.traj.reader.Reader.next_frame}\pysiglinewithargsret{\bfcode{next\_frame}}{}{}
\end{fulllineitems}

\index{iterate\_over\_frames() (Reader method)}

\begin{fulllineitems}
\phantomsection\label{aqueduct.traj.reader:aqueduct.traj.reader.Reader.iterate_over_frames}\pysiglinewithargsret{\bfcode{iterate\_over\_frames}}{}{}
\end{fulllineitems}

\index{parse\_selection() (Reader method)}

\begin{fulllineitems}
\phantomsection\label{aqueduct.traj.reader:aqueduct.traj.reader.Reader.parse_selection}\pysiglinewithargsret{\bfcode{parse\_selection}}{\emph{selection}}{}
\end{fulllineitems}

\index{select\_resnum() (Reader method)}

\begin{fulllineitems}
\phantomsection\label{aqueduct.traj.reader:aqueduct.traj.reader.Reader.select_resnum}\pysiglinewithargsret{\bfcode{select\_resnum}}{\emph{resnum}}{}
\end{fulllineitems}

\index{select\_multiple\_resnum() (Reader method)}

\begin{fulllineitems}
\phantomsection\label{aqueduct.traj.reader:aqueduct.traj.reader.Reader.select_multiple_resnum}\pysiglinewithargsret{\bfcode{select\_multiple\_resnum}}{\emph{resnum\_list}}{}
\end{fulllineitems}


\end{fulllineitems}

\index{ReadViaMDA (class in aqueduct.traj.reader)}

\begin{fulllineitems}
\phantomsection\label{aqueduct.traj.reader:aqueduct.traj.reader.ReadViaMDA}\pysiglinewithargsret{\strong{class }\bfcode{ReadViaMDA}}{\emph{topology}, \emph{trajectory}}{}
Bases: {\hyperref[aqueduct.traj.reader:aqueduct.traj.reader.Reader]{\code{aqueduct.traj.reader.Reader}}}
\index{number\_of\_frames (ReadViaMDA attribute)}

\begin{fulllineitems}
\phantomsection\label{aqueduct.traj.reader:aqueduct.traj.reader.ReadViaMDA.number_of_frames}\pysigline{\bfcode{number\_of\_frames}}
\end{fulllineitems}

\index{set\_current\_frame() (ReadViaMDA method)}

\begin{fulllineitems}
\phantomsection\label{aqueduct.traj.reader:aqueduct.traj.reader.ReadViaMDA.set_current_frame}\pysiglinewithargsret{\bfcode{set\_current\_frame}}{\emph{frame}}{}
\end{fulllineitems}

\index{next\_frame() (ReadViaMDA method)}

\begin{fulllineitems}
\phantomsection\label{aqueduct.traj.reader:aqueduct.traj.reader.ReadViaMDA.next_frame}\pysiglinewithargsret{\bfcode{next\_frame}}{}{}
\end{fulllineitems}

\index{parse\_selection() (ReadViaMDA method)}

\begin{fulllineitems}
\phantomsection\label{aqueduct.traj.reader:aqueduct.traj.reader.ReadViaMDA.parse_selection}\pysiglinewithargsret{\bfcode{parse\_selection}}{\emph{selection}}{}
\end{fulllineitems}

\index{select\_resnum() (ReadViaMDA method)}

\begin{fulllineitems}
\phantomsection\label{aqueduct.traj.reader:aqueduct.traj.reader.ReadViaMDA.select_resnum}\pysiglinewithargsret{\bfcode{select\_resnum}}{\emph{resnum}}{}
\end{fulllineitems}

\index{select\_multiple\_resnum() (ReadViaMDA method)}

\begin{fulllineitems}
\phantomsection\label{aqueduct.traj.reader:aqueduct.traj.reader.ReadViaMDA.select_multiple_resnum}\pysiglinewithargsret{\bfcode{select\_multiple\_resnum}}{\emph{resnum\_list}}{}
\end{fulllineitems}

\index{\_\_enter\_\_() (ReadViaMDA method)}

\begin{fulllineitems}
\phantomsection\label{aqueduct.traj.reader:aqueduct.traj.reader.ReadViaMDA.__enter__}\pysiglinewithargsret{\bfcode{\_\_enter\_\_}}{}{}
\end{fulllineitems}

\index{\_\_exit\_\_() (ReadViaMDA method)}

\begin{fulllineitems}
\phantomsection\label{aqueduct.traj.reader:aqueduct.traj.reader.ReadViaMDA.__exit__}\pysiglinewithargsret{\bfcode{\_\_exit\_\_}}{\emph{typ}, \emph{value}, \emph{traceback}}{}
\end{fulllineitems}


\end{fulllineitems}

\index{ReadAmberNetCDFviaMDA (class in aqueduct.traj.reader)}

\begin{fulllineitems}
\phantomsection\label{aqueduct.traj.reader:aqueduct.traj.reader.ReadAmberNetCDFviaMDA}\pysiglinewithargsret{\strong{class }\bfcode{ReadAmberNetCDFviaMDA}}{\emph{topology}, \emph{trajectory}}{}
Bases: {\hyperref[aqueduct.traj.reader:aqueduct.traj.reader.ReadViaMDA]{\code{aqueduct.traj.reader.ReadViaMDA}}}
\index{open\_trajectory() (ReadAmberNetCDFviaMDA method)}

\begin{fulllineitems}
\phantomsection\label{aqueduct.traj.reader:aqueduct.traj.reader.ReadAmberNetCDFviaMDA.open_trajectory}\pysiglinewithargsret{\bfcode{open\_trajectory}}{}{}
\end{fulllineitems}


\end{fulllineitems}

\index{ReadDCDviaMDA (class in aqueduct.traj.reader)}

\begin{fulllineitems}
\phantomsection\label{aqueduct.traj.reader:aqueduct.traj.reader.ReadDCDviaMDA}\pysiglinewithargsret{\strong{class }\bfcode{ReadDCDviaMDA}}{\emph{topology}, \emph{trajectory}}{}
Bases: {\hyperref[aqueduct.traj.reader:aqueduct.traj.reader.ReadViaMDA]{\code{aqueduct.traj.reader.ReadViaMDA}}}
\index{open\_trajectory() (ReadDCDviaMDA method)}

\begin{fulllineitems}
\phantomsection\label{aqueduct.traj.reader:aqueduct.traj.reader.ReadDCDviaMDA.open_trajectory}\pysiglinewithargsret{\bfcode{open\_trajectory}}{}{}
\end{fulllineitems}


\end{fulllineitems}



\subparagraph{aqueduct.traj.selections module}
\label{aqueduct.traj.selections:aqueduct-traj-selections-module}\label{aqueduct.traj.selections::doc}\label{aqueduct.traj.selections:module-aqueduct.traj.selections}\index{aqueduct.traj.selections (module)}\index{Selection (class in aqueduct.traj.selections)}

\begin{fulllineitems}
\phantomsection\label{aqueduct.traj.selections:aqueduct.traj.selections.Selection}\pysigline{\strong{class }\bfcode{Selection}}
Bases: \href{http://docs.python.org/2/library/functions.html\#object}{\code{object}}

def \_\_init\_\_(self,selection,selection\_string=None):
\begin{quote}

self.selection\_object = selection
self.selection\_string = selection\_string
\end{quote}
\index{center\_of\_mass() (Selection method)}

\begin{fulllineitems}
\phantomsection\label{aqueduct.traj.selections:aqueduct.traj.selections.Selection.center_of_mass}\pysiglinewithargsret{\bfcode{center\_of\_mass}}{}{}
\end{fulllineitems}

\index{iterate\_over\_residues() (Selection method)}

\begin{fulllineitems}
\phantomsection\label{aqueduct.traj.selections:aqueduct.traj.selections.Selection.iterate_over_residues}\pysiglinewithargsret{\bfcode{iterate\_over\_residues}}{}{}
\end{fulllineitems}

\index{unique\_resids() (Selection method)}

\begin{fulllineitems}
\phantomsection\label{aqueduct.traj.selections:aqueduct.traj.selections.Selection.unique_resids}\pysiglinewithargsret{\bfcode{unique\_resids}}{}{}
\end{fulllineitems}

\index{unique\_resids\_number() (Selection method)}

\begin{fulllineitems}
\phantomsection\label{aqueduct.traj.selections:aqueduct.traj.selections.Selection.unique_resids_number}\pysiglinewithargsret{\bfcode{unique\_resids\_number}}{}{}
\end{fulllineitems}

\index{atom\_positions() (Selection method)}

\begin{fulllineitems}
\phantomsection\label{aqueduct.traj.selections:aqueduct.traj.selections.Selection.atom_positions}\pysiglinewithargsret{\bfcode{atom\_positions}}{}{}
\end{fulllineitems}

\index{center\_of\_mass\_of\_residues() (Selection method)}

\begin{fulllineitems}
\phantomsection\label{aqueduct.traj.selections:aqueduct.traj.selections.Selection.center_of_mass_of_residues}\pysiglinewithargsret{\bfcode{center\_of\_mass\_of\_residues}}{}{}
\end{fulllineitems}

\index{get\_convexhull\_of\_atom\_positions() (Selection method)}

\begin{fulllineitems}
\phantomsection\label{aqueduct.traj.selections:aqueduct.traj.selections.Selection.get_convexhull_of_atom_positions}\pysiglinewithargsret{\bfcode{get\_convexhull\_of\_atom\_positions}}{}{}
\end{fulllineitems}

\index{uniquify() (Selection method)}

\begin{fulllineitems}
\phantomsection\label{aqueduct.traj.selections:aqueduct.traj.selections.Selection.uniquify}\pysiglinewithargsret{\bfcode{uniquify}}{}{}
\end{fulllineitems}

\index{\_\_add\_\_() (Selection method)}

\begin{fulllineitems}
\phantomsection\label{aqueduct.traj.selections:aqueduct.traj.selections.Selection.__add__}\pysiglinewithargsret{\bfcode{\_\_add\_\_}}{\emph{other}}{}
\end{fulllineitems}

\index{first\_resid() (Selection method)}

\begin{fulllineitems}
\phantomsection\label{aqueduct.traj.selections:aqueduct.traj.selections.Selection.first_resid}\pysiglinewithargsret{\bfcode{first\_resid}}{}{}
\end{fulllineitems}


\end{fulllineitems}

\index{SelectionMDA (class in aqueduct.traj.selections)}

\begin{fulllineitems}
\phantomsection\label{aqueduct.traj.selections:aqueduct.traj.selections.SelectionMDA}\pysiglinewithargsret{\strong{class }\bfcode{SelectionMDA}}{\emph{atoms}}{}
Bases: \code{MDAnalysis.core.AtomGroup.AtomGroup}, {\hyperref[aqueduct.traj.selections:aqueduct.traj.selections.Selection]{\code{aqueduct.traj.selections.Selection}}}
\index{iterate\_over\_residues() (SelectionMDA method)}

\begin{fulllineitems}
\phantomsection\label{aqueduct.traj.selections:aqueduct.traj.selections.SelectionMDA.iterate_over_residues}\pysiglinewithargsret{\bfcode{iterate\_over\_residues}}{}{}
\end{fulllineitems}

\index{unique\_resids() (SelectionMDA method)}

\begin{fulllineitems}
\phantomsection\label{aqueduct.traj.selections:aqueduct.traj.selections.SelectionMDA.unique_resids}\pysiglinewithargsret{\bfcode{unique\_resids}}{\emph{ikwid=False}}{}
\end{fulllineitems}

\index{atom\_positions() (SelectionMDA method)}

\begin{fulllineitems}
\phantomsection\label{aqueduct.traj.selections:aqueduct.traj.selections.SelectionMDA.atom_positions}\pysiglinewithargsret{\bfcode{atom\_positions}}{}{}
\end{fulllineitems}

\index{\_\_add\_\_() (SelectionMDA method)}

\begin{fulllineitems}
\phantomsection\label{aqueduct.traj.selections:aqueduct.traj.selections.SelectionMDA.__add__}\pysiglinewithargsret{\bfcode{\_\_add\_\_}}{\emph{other}}{}
\end{fulllineitems}

\index{uniquify() (SelectionMDA method)}

\begin{fulllineitems}
\phantomsection\label{aqueduct.traj.selections:aqueduct.traj.selections.SelectionMDA.uniquify}\pysiglinewithargsret{\bfcode{uniquify}}{}{}
\end{fulllineitems}


\end{fulllineitems}

\index{CompactSelectionMDA (class in aqueduct.traj.selections)}

\begin{fulllineitems}
\phantomsection\label{aqueduct.traj.selections:aqueduct.traj.selections.CompactSelectionMDA}\pysiglinewithargsret{\strong{class }\bfcode{CompactSelectionMDA}}{\emph{sMDA}}{}
Bases: \href{http://docs.python.org/2/library/functions.html\#object}{\code{object}}
\index{\_\_init\_\_() (CompactSelectionMDA method)}

\begin{fulllineitems}
\phantomsection\label{aqueduct.traj.selections:aqueduct.traj.selections.CompactSelectionMDA.__init__}\pysiglinewithargsret{\bfcode{\_\_init\_\_}}{\emph{sMDA}}{}
\end{fulllineitems}

\index{toSelectionMDA() (CompactSelectionMDA method)}

\begin{fulllineitems}
\phantomsection\label{aqueduct.traj.selections:aqueduct.traj.selections.CompactSelectionMDA.toSelectionMDA}\pysiglinewithargsret{\bfcode{toSelectionMDA}}{\emph{reader}}{}
\end{fulllineitems}


\end{fulllineitems}



\paragraph{Module contents}
\label{aqueduct.traj:module-aqueduct.traj}\label{aqueduct.traj:module-contents}\index{aqueduct.traj (module)}

\subsubsection{aqueduct.utils package}
\label{aqueduct.utils:aqueduct-utils-package}\label{aqueduct.utils::doc}

\paragraph{Submodules}
\label{aqueduct.utils:submodules}

\subparagraph{aqueduct.utils.helpers module}
\label{aqueduct.utils.helpers:module-aqueduct.utils.helpers}\label{aqueduct.utils.helpers:aqueduct-utils-helpers-module}\label{aqueduct.utils.helpers::doc}\index{aqueduct.utils.helpers (module)}
Collection of helpers - functions and decorators.
\index{combine() (in module aqueduct.utils.helpers)}

\begin{fulllineitems}
\phantomsection\label{aqueduct.utils.helpers:aqueduct.utils.helpers.combine}\pysiglinewithargsret{\bfcode{combine}}{\emph{seqin}}{}
This is an alien function. It is not extensively used.

Directly taken form \href{http://aspn.activestate.com/ASPN/Cookbook/Python/Recipe/302478/index\_txt}{http://aspn.activestate.com/ASPN/Cookbook/Python/Recipe/302478/index\_txt}

Returns a list of all combinations of argument sequences.
For example, following call:

\begin{Verbatim}[commandchars=\\\{\}]
\PYG{n}{combine}\PYG{p}{(}\PYG{p}{(}\PYG{p}{(}\PYG{l+m+mi}{1}\PYG{p}{,}\PYG{l+m+mi}{2}\PYG{p}{)}\PYG{p}{,}\PYG{p}{(}\PYG{l+m+mi}{3}\PYG{p}{,}\PYG{l+m+mi}{4}\PYG{p}{)}\PYG{p}{)}\PYG{p}{)}
\end{Verbatim}

gives following list of combinations:

\begin{Verbatim}[commandchars=\\\{\}]
\PYG{p}{[}\PYG{p}{[}\PYG{l+m+mi}{1}\PYG{p}{,} \PYG{l+m+mi}{3}\PYG{p}{]}\PYG{p}{,} \PYG{p}{[}\PYG{l+m+mi}{1}\PYG{p}{,} \PYG{l+m+mi}{4}\PYG{p}{]}\PYG{p}{,} \PYG{p}{[}\PYG{l+m+mi}{2}\PYG{p}{,} \PYG{l+m+mi}{3}\PYG{p}{]}\PYG{p}{,} \PYG{p}{[}\PYG{l+m+mi}{2}\PYG{p}{,} \PYG{l+m+mi}{4}\PYG{p}{]}\PYG{p}{]}
\end{Verbatim}
\begin{quote}\begin{description}
\item[{Parameters}] \leavevmode
\textbf{sequin} (\href{http://docs.python.org/2/library/functions.html\#tuple}{\emph{tuple}}) -- Tuple of sequences to combine.

\item[{Returns}] \leavevmode
All possible combinations of all input sequences.

\item[{Return type}] \leavevmode
list of lists

\end{description}\end{quote}

\end{fulllineitems}

\index{is\_number() (in module aqueduct.utils.helpers)}

\begin{fulllineitems}
\phantomsection\label{aqueduct.utils.helpers:aqueduct.utils.helpers.is_number}\pysiglinewithargsret{\bfcode{is\_number}}{\emph{s}}{}
\end{fulllineitems}

\index{lind() (in module aqueduct.utils.helpers)}

\begin{fulllineitems}
\phantomsection\label{aqueduct.utils.helpers:aqueduct.utils.helpers.lind}\pysiglinewithargsret{\bfcode{lind}}{\emph{l}, \emph{ind}}{}
Indexes lists using lists of integers as identificators.
For example:

\begin{Verbatim}[commandchars=\\\{\}]
\PYG{n}{lind}\PYG{p}{(}\PYG{p}{[}\PYG{l+s}{\PYGZsq{}}\PYG{l+s}{a}\PYG{l+s}{\PYGZsq{}}\PYG{p}{,}\PYG{l+s}{\PYGZsq{}}\PYG{l+s}{b}\PYG{l+s}{\PYGZsq{}}\PYG{p}{,}\PYG{l+s}{\PYGZsq{}}\PYG{l+s}{c}\PYG{l+s}{\PYGZsq{}}\PYG{p}{,}\PYG{l+s}{\PYGZsq{}}\PYG{l+s}{d}\PYG{l+s}{\PYGZsq{}}\PYG{p}{,}\PYG{l+s}{\PYGZsq{}}\PYG{l+s}{e}\PYG{l+s}{\PYGZsq{}}\PYG{p}{]}\PYG{p}{,}\PYG{p}{[}\PYG{l+m+mi}{1}\PYG{p}{,}\PYG{l+m+mi}{4}\PYG{p}{,}\PYG{l+m+mi}{2}\PYG{p}{]}\PYG{p}{)}
\end{Verbatim}

returns:

\begin{Verbatim}[commandchars=\\\{\}]
\PYG{p}{[}\PYG{l+s}{\PYGZsq{}}\PYG{l+s}{b}\PYG{l+s}{\PYGZsq{}}\PYG{p}{,} \PYG{l+s}{\PYGZsq{}}\PYG{l+s}{e}\PYG{l+s}{\PYGZsq{}}\PYG{p}{,} \PYG{l+s}{\PYGZsq{}}\PYG{l+s}{c}\PYG{l+s}{\PYGZsq{}}\PYG{p}{]}
\end{Verbatim}
\begin{quote}\begin{description}
\item[{Parameters}] \leavevmode\begin{itemize}
\item {} 
\textbf{l} (\href{http://docs.python.org/2/library/functions.html\#list}{\emph{list}}) -- List to be indexed.

\item {} 
\textbf{ind} (\href{http://docs.python.org/2/library/functions.html\#list}{\emph{list}}) -- Intiger indexes.

\end{itemize}

\item[{Returns}] \leavevmode
Reindexed list.

\item[{Return type}] \leavevmode
list

\end{description}\end{quote}

\end{fulllineitems}

\index{Auto (class in aqueduct.utils.helpers)}

\begin{fulllineitems}
\phantomsection\label{aqueduct.utils.helpers:aqueduct.utils.helpers.Auto}\pysigline{\strong{class }\bfcode{Auto}}
Auto type definition.
The class is used as an alternative value for options (if particular option supports it).
If options (or variables/parameters etc.) have value of {\hyperref[aqueduct.utils.helpers:aqueduct.utils.helpers.Auto]{\code{Auto}}} it means that an automatic
process for parametrization shoud be performed.

For example, if the input parameter is set to {\hyperref[aqueduct.utils.helpers:aqueduct.utils.helpers.Auto]{\code{Auto}}} it is supposed that its value is calculated
on the basis of input data or other parapmeters.
\index{\_\_repr\_\_() (Auto method)}

\begin{fulllineitems}
\phantomsection\label{aqueduct.utils.helpers:aqueduct.utils.helpers.Auto.__repr__}\pysiglinewithargsret{\bfcode{\_\_repr\_\_}}{}{}~\begin{quote}\begin{description}
\item[{Returns}] \leavevmode
String \code{Auto}.

\item[{Return type}] \leavevmode
str

\end{description}\end{quote}

\end{fulllineitems}

\index{\_\_str\_\_() (Auto method)}

\begin{fulllineitems}
\phantomsection\label{aqueduct.utils.helpers:aqueduct.utils.helpers.Auto.__str__}\pysiglinewithargsret{\bfcode{\_\_str\_\_}}{}{}
Calls {\hyperref[aqueduct.utils.helpers:aqueduct.utils.helpers.Auto.__repr__]{\code{\_\_repr\_\_()}}}.

\end{fulllineitems}


\end{fulllineitems}

\index{create\_tmpfile() (in module aqueduct.utils.helpers)}

\begin{fulllineitems}
\phantomsection\label{aqueduct.utils.helpers:aqueduct.utils.helpers.create_tmpfile}\pysiglinewithargsret{\bfcode{create\_tmpfile}}{\emph{ext=None}}{}
Creates temporary file. File is created, closed and its file name is returned.

\begin{notice}{note}{Note:}
It is responsability of the caller to delete the file.
\end{notice}
\begin{quote}\begin{description}
\item[{Parameters}] \leavevmode
\textbf{ext} (\href{http://docs.python.org/2/library/functions.html\#str}{\emph{str}}) -- Optional extension of the file.

\item[{Returns}] \leavevmode
File name of created temporary file.

\item[{Return type}] \leavevmode
str

\end{description}\end{quote}

\end{fulllineitems}

\index{range2int() (in module aqueduct.utils.helpers)}

\begin{fulllineitems}
\phantomsection\label{aqueduct.utils.helpers:aqueduct.utils.helpers.range2int}\pysiglinewithargsret{\bfcode{range2int}}{\emph{r}, \emph{uniq=True}}{}
Transforms a string range in to a list of integers.

For example, a following string:

\begin{Verbatim}[commandchars=\\\{\}]
\PYG{l+s}{\PYGZsq{}}\PYG{l+s}{0:2 4:5 7 9}\PYG{l+s}{\PYGZsq{}}
\end{Verbatim}

is transformed into:

\begin{Verbatim}[commandchars=\\\{\}]
\PYG{p}{[}\PYG{l+m+mi}{0}\PYG{p}{,}\PYG{l+m+mi}{1}\PYG{p}{,}\PYG{l+m+mi}{2}\PYG{p}{,}\PYG{l+m+mi}{4}\PYG{p}{,}\PYG{l+m+mi}{5}\PYG{p}{,}\PYG{l+m+mi}{7}\PYG{p}{,}\PYG{l+m+mi}{9}\PYG{p}{]}
\end{Verbatim}
\begin{quote}\begin{description}
\item[{Parameters}] \leavevmode\begin{itemize}
\item {} 
\textbf{r} (\href{http://docs.python.org/2/library/functions.html\#str}{\emph{str}}) -- String of input range.

\item {} 
\textbf{uniq} (\href{http://docs.python.org/2/library/functions.html\#bool}{\emph{bool}}) -- Optional parameter, if set to \emph{True} only unique and sorted integers are returned.

\end{itemize}

\item[{Returns}] \leavevmode
List of integers.

\item[{Return type}] \leavevmode
list of int

\end{description}\end{quote}

\end{fulllineitems}

\index{int2range() (in module aqueduct.utils.helpers)}

\begin{fulllineitems}
\phantomsection\label{aqueduct.utils.helpers:aqueduct.utils.helpers.int2range}\pysiglinewithargsret{\bfcode{int2range}}{\emph{l}}{}
Transforms a list of integers in to a string of ranges.

For example, a following list:

\begin{Verbatim}[commandchars=\\\{\}]
\PYG{p}{[}\PYG{l+m+mi}{0}\PYG{p}{,}\PYG{l+m+mi}{1}\PYG{p}{,}\PYG{l+m+mi}{2}\PYG{p}{,}\PYG{l+m+mi}{4}\PYG{p}{,}\PYG{l+m+mi}{5}\PYG{p}{,}\PYG{l+m+mi}{7}\PYG{p}{,}\PYG{l+m+mi}{9}\PYG{p}{]}
\end{Verbatim}

is transformed into:

\begin{Verbatim}[commandchars=\\\{\}]
0:2 4:5 7 9
\end{Verbatim}
\begin{quote}\begin{description}
\item[{Parameters}] \leavevmode
\textbf{l} (\href{http://docs.python.org/2/library/functions.html\#list}{\emph{list}}) -- input list of int

\item[{Returns}] \leavevmode
String of ranges.

\item[{Return type}] \leavevmode
str

\end{description}\end{quote}

\end{fulllineitems}

\index{is\_iterable() (in module aqueduct.utils.helpers)}

\begin{fulllineitems}
\phantomsection\label{aqueduct.utils.helpers:aqueduct.utils.helpers.is_iterable}\pysiglinewithargsret{\bfcode{is\_iterable}}{\emph{l}}{}
Checks if provided object is iterable.
Returns True is it is iterable, otherwise returns False.
\begin{quote}\begin{description}
\item[{Parameters}] \leavevmode
\textbf{l} (\href{http://docs.python.org/2/library/functions.html\#list}{\emph{list}}) -- input object

\item[{Returns}] \leavevmode
True if submited object is iterable otherwise returns False.

\item[{Return type}] \leavevmode
bool

\end{description}\end{quote}

\begin{notice}{warning}{Warning:}
Current implementation cannot be used with generators!
\end{notice}

\begin{notice}{note}{Todo}

Current implementation is primitive and HAVE TO be replaced.
\end{notice}

\end{fulllineitems}

\index{sortify() (in module aqueduct.utils.helpers)}

\begin{fulllineitems}
\phantomsection\label{aqueduct.utils.helpers:aqueduct.utils.helpers.sortify}\pysiglinewithargsret{\bfcode{sortify}}{\emph{gen}}{}
Decorator to convert functions' outputs into a sorted list. If the output is iterable it is converted in to a list
of apropriate lenght. If the output is not iterable it is converted in to a list of lenght 1.

Written on the basis of {\hyperref[aqueduct.utils.helpers:aqueduct.utils.helpers.listify]{\code{listify()}}}.
\begin{quote}\begin{description}
\item[{Returns}] \leavevmode
Output of decorated function converted to a sorted list.

\item[{Return type}] \leavevmode
list

\end{description}\end{quote}

\end{fulllineitems}

\index{listify() (in module aqueduct.utils.helpers)}

\begin{fulllineitems}
\phantomsection\label{aqueduct.utils.helpers:aqueduct.utils.helpers.listify}\pysiglinewithargsret{\bfcode{listify}}{\emph{gen}}{}
Decorator to convert functions' outputs into a list. If the output is iterable it is converted in to a list
of apropriate lenght. If the output is not iterable it is converted in to a list of lenght 1.

This function was copied from:

\href{http://argandgahandapandpa.wordpress.com/2009/03/29/python-generator-to-list-decorator/}{http://argandgahandapandpa.wordpress.com/2009/03/29/python-generator-to-list-decorator/}

and further improved by \href{mailto:tljm@wp.pl}{tljm@wp.pl}.
\begin{quote}\begin{description}
\item[{Returns}] \leavevmode
Output of decorated function converted to a list.

\item[{Return type}] \leavevmode
list

\end{description}\end{quote}

\end{fulllineitems}

\index{tupleify() (in module aqueduct.utils.helpers)}

\begin{fulllineitems}
\phantomsection\label{aqueduct.utils.helpers:aqueduct.utils.helpers.tupleify}\pysiglinewithargsret{\bfcode{tupleify}}{\emph{gen}}{}
Decorator to convert functions' outputs into a tuple. If the output is iterable it is converted in to a tuple
of apropriate lenght. If the output is not iterable it is converted in to a tuple of lenght 1.

Written on the basis of {\hyperref[aqueduct.utils.helpers:aqueduct.utils.helpers.listify]{\code{listify()}}}.
\begin{quote}\begin{description}
\item[{Returns}] \leavevmode
Output of decorated function converted to a tuple.

\item[{Return type}] \leavevmode
tuple

\end{description}\end{quote}

\end{fulllineitems}

\index{arrayify() (in module aqueduct.utils.helpers)}

\begin{fulllineitems}
\phantomsection\label{aqueduct.utils.helpers:aqueduct.utils.helpers.arrayify}\pysiglinewithargsret{\bfcode{arrayify}}{\emph{gen}}{}
Decorator to convert functions' outputs into a 2D numpy array. If the output is iterable it is converted in to a a 2D numpy array
of apropriate shape. If the output is not iterable it is converted in to a a 2D numpy array of shape 1x1.

Written on the basis of {\hyperref[aqueduct.utils.helpers:aqueduct.utils.helpers.listify]{\code{listify()}}}.
\begin{quote}\begin{description}
\item[{Returns}] \leavevmode
Output of decorated function converted to a 2D numpy array.

\item[{Return type}] \leavevmode
numpy.ndarray

\end{description}\end{quote}

\end{fulllineitems}

\index{arrayify1() (in module aqueduct.utils.helpers)}

\begin{fulllineitems}
\phantomsection\label{aqueduct.utils.helpers:aqueduct.utils.helpers.arrayify1}\pysiglinewithargsret{\bfcode{arrayify1}}{\emph{gen}}{}
Decorator to convert functions' outputs into a 1D numpy array. If the output is iterable it is converted in to a a 2D numpy array
of apropriate shape. If the output is not iterable it is converted in to a a 2D numpy array of shape 1x1.

Written on the basis of {\hyperref[aqueduct.utils.helpers:aqueduct.utils.helpers.listify]{\code{listify()}}}.
\begin{quote}\begin{description}
\item[{Returns}] \leavevmode
Output of decorated function converted to a 1D numpy array.

\item[{Return type}] \leavevmode
numpy.ndarray

\end{description}\end{quote}

\end{fulllineitems}

\index{list\_blocks\_to\_slices() (in module aqueduct.utils.helpers)}

\begin{fulllineitems}
\phantomsection\label{aqueduct.utils.helpers:aqueduct.utils.helpers.list_blocks_to_slices}\pysiglinewithargsret{\bfcode{list\_blocks\_to\_slices}}{\emph{l}}{}
Slices list in to block according to its elements identity. Resulting slices correspond to blocks of
identical elements.
\begin{quote}\begin{description}
\item[{Parameters}] \leavevmode
\textbf{l} (\href{http://docs.python.org/2/library/functions.html\#list}{\emph{list}}) -- List of any objects.

\item[{Returns}] \leavevmode
Generator of slices.

\item[{Return type}] \leavevmode
generator

\end{description}\end{quote}

\end{fulllineitems}

\index{what2what() (in module aqueduct.utils.helpers)}

\begin{fulllineitems}
\phantomsection\label{aqueduct.utils.helpers:aqueduct.utils.helpers.what2what}\pysiglinewithargsret{\bfcode{what2what}}{\emph{what}, \emph{towhat}}{}~\begin{quote}\begin{description}
\item[{Parameters}] \leavevmode\begin{itemize}
\item {} 
\textbf{what} (\href{http://docs.python.org/2/library/functions.html\#list}{\emph{list}}) -- Input list for which indices of elements present in \code{towhat} are returned.

\item {} 
\textbf{towhat} (\href{http://docs.python.org/2/library/functions.html\#list}{\emph{list}}) -- List of elements which input list is indexed to.

\end{itemize}

\item[{Returns}] \leavevmode
Indices of \code{what} list that are present in \code{towhat} list.

\item[{Return type}] \leavevmode
tuple

\end{description}\end{quote}

\end{fulllineitems}

\index{make\_iterable() (in module aqueduct.utils.helpers)}

\begin{fulllineitems}
\phantomsection\label{aqueduct.utils.helpers:aqueduct.utils.helpers.make_iterable}\pysiglinewithargsret{\bfcode{make\_iterable}}{\emph{something}}{}
If input object is not interable returns it as one element list. Otherwise returns the object.
\begin{quote}\begin{description}
\item[{Parameters}] \leavevmode
\textbf{something} (\href{http://docs.python.org/2/library/functions.html\#object}{\emph{object}}) -- Input object.

\item[{Returns}] \leavevmode
Iterable object.

\item[{Return type}] \leavevmode
iterable or list

\end{description}\end{quote}

\end{fulllineitems}



\subparagraph{aqueduct.utils.log module}
\label{aqueduct.utils.log::doc}\label{aqueduct.utils.log:module-aqueduct.utils.log}\label{aqueduct.utils.log:aqueduct-utils-log-module}\index{aqueduct.utils.log (module)}
Module comprises convieniences functions and definitios for logging
purposes including progress bar helpers.
\index{gregorian\_year\_in\_days (in module aqueduct.utils.log)}

\begin{fulllineitems}
\phantomsection\label{aqueduct.utils.log:aqueduct.utils.log.gregorian_year_in_days}\pysigline{\bfcode{gregorian\_year\_in\_days}\strong{ = 365.2425}}
Length of Gregorian year in days. Average value. Source: \href{https://en.wikipedia.org/wiki/Year}{https://en.wikipedia.org/wiki/Year}

\end{fulllineitems}

\index{smart\_time\_string() (in module aqueduct.utils.log)}

\begin{fulllineitems}
\phantomsection\label{aqueduct.utils.log:aqueduct.utils.log.smart_time_string}\pysiglinewithargsret{\bfcode{smart\_time\_string}}{\emph{s}, \emph{rl=0}, \emph{t=1.1}, \emph{maximal\_length=None}, \emph{maximal\_units=5}}{}
Function transforms time in seconds to nicely formatted string of
length defined by \code{maximal\_length}. Depending on number of seconds time is represented with
one or more of the following units:

\begin{tabulary}{\linewidth}{|L|L|}
\hline
\textsf{\relax 
Unit name
} & \textsf{\relax 
Unit abbreviation
}\\
\hline
seconds
 & 
s
\\
\hline
minutes
 & 
m
\\
\hline
hours
 & 
h
\\
\hline
days
 & 
d
\\
\hline
years
 & 
y
\\
\hline\end{tabulary}


Maximal number of units used in time string can be set with \code{maximal\_units}.
\begin{quote}\begin{description}
\item[{Parameters}] \leavevmode\begin{itemize}
\item {} 
\textbf{s} (\href{http://docs.python.org/2/library/functions.html\#int}{\emph{int}}) -- Input time in seconds.

\item {} 
\textbf{rl} (\href{http://docs.python.org/2/library/functions.html\#int}{\emph{int}}) -- Number of units already used for representing time.

\item {} 
\textbf{t} (\href{http://docs.python.org/2/library/functions.html\#float}{\emph{float}}) -- Exces above standard number of current time units.

\item {} 
\textbf{maximal\_length} (\href{http://docs.python.org/2/library/functions.html\#int}{\emph{int}}) -- Maximal length of the output string. Must be greater then 0.

\item {} 
\textbf{maximal\_units} (\href{http://docs.python.org/2/library/functions.html\#int}{\emph{int}}) -- Maximal number of units used in the output string. Must be greater then 0 and lower then 6.

\end{itemize}

\item[{Returns}] \leavevmode
string of nicely formated time

\item[{Return type}] \leavevmode
str

\end{description}\end{quote}

\end{fulllineitems}

\index{gsep() (in module aqueduct.utils.log)}

\begin{fulllineitems}
\phantomsection\label{aqueduct.utils.log:aqueduct.utils.log.gsep}\pysiglinewithargsret{\bfcode{gsep}}{\emph{sep='-`}, \emph{times=72}, \emph{length=None}}{}
Generic separator.
\begin{quote}\begin{description}
\item[{Parameters}] \leavevmode\begin{itemize}
\item {} 
\textbf{sep} (\href{http://docs.python.org/2/library/functions.html\#str}{\emph{str}}) -- Element(s) of separator.

\item {} 
\textbf{times} (\href{http://docs.python.org/2/library/functions.html\#int}{\emph{int}}) -- Number of times \code{sep} is printed.

\item {} 
\textbf{length} (\href{http://docs.python.org/2/library/functions.html\#int}{\emph{int}}) -- Optional maximal length of output.

\end{itemize}

\item[{Returns}] \leavevmode
String separator.

\item[{Return type}] \leavevmode
str

\end{description}\end{quote}

\end{fulllineitems}

\index{tsep() (in module aqueduct.utils.log)}

\begin{fulllineitems}
\phantomsection\label{aqueduct.utils.log:aqueduct.utils.log.tsep}\pysiglinewithargsret{\bfcode{tsep}}{\emph{line}}{}~\begin{quote}\begin{description}
\item[{Parameters}] \leavevmode
\textbf{line} (\href{http://docs.python.org/2/library/functions.html\#str}{\emph{str}}) -- Input line.

\item[{Returns}] \leavevmode
Returns default {\hyperref[aqueduct.utils.log:aqueduct.utils.log.gsep]{\code{gsep()}}} of length of \code{line}.

\end{description}\end{quote}

\end{fulllineitems}

\index{underline() (in module aqueduct.utils.log)}

\begin{fulllineitems}
\phantomsection\label{aqueduct.utils.log:aqueduct.utils.log.underline}\pysiglinewithargsret{\bfcode{underline}}{\emph{line}}{}~\begin{quote}\begin{description}
\item[{Parameters}] \leavevmode
\textbf{line} (\href{http://docs.python.org/2/library/functions.html\#str}{\emph{str}}) -- Input line.

\item[{Returns}] \leavevmode
String made by concatenation of \code{line}, \code{os.linesep}, and output of {\hyperref[aqueduct.utils.log:aqueduct.utils.log.tsep]{\code{tsep()}}} called with \code{line}.

\item[{Return type}] \leavevmode
str

\end{description}\end{quote}

\end{fulllineitems}

\index{thead() (in module aqueduct.utils.log)}

\begin{fulllineitems}
\phantomsection\label{aqueduct.utils.log:aqueduct.utils.log.thead}\pysiglinewithargsret{\bfcode{thead}}{\emph{line}}{}~\begin{quote}\begin{description}
\item[{Parameters}] \leavevmode
\textbf{line} (\href{http://docs.python.org/2/library/functions.html\#str}{\emph{str}}) -- Input line.

\item[{Returns}] \leavevmode
String made by concatenation of output of {\hyperref[aqueduct.utils.log:aqueduct.utils.log.tsep]{\code{tsep()}}} called with \code{line}, \code{line}, \code{os.linesep}, and again output of {\hyperref[aqueduct.utils.log:aqueduct.utils.log.tsep]{\code{tsep()}}} called with \code{line}.

\item[{Return type}] \leavevmode
str

\end{description}\end{quote}

\end{fulllineitems}

\index{SimpleProgressBar (class in aqueduct.utils.log)}

\begin{fulllineitems}
\phantomsection\label{aqueduct.utils.log:aqueduct.utils.log.SimpleProgressBar}\pysiglinewithargsret{\strong{class }\bfcode{SimpleProgressBar}}{\emph{mess=None}, \emph{maxval=None}}{}
Bases: \href{http://docs.python.org/2/library/functions.html\#object}{\code{object}}

Simple progress bar displaying progress with percent indicator, progress bar and ETA.
Progress is measured by iterations.
\begin{quote}\begin{description}
\item[{Variables}] \leavevmode\begin{itemize}
\item {} 
\textbf{rotate} (\href{http://docs.python.org/2/library/functions.html\#str}{\emph{str}}) -- String comprising characters with frames of a rotating toy.

\item {} 
\textbf{barlenght} (\href{http://docs.python.org/2/library/functions.html\#int}{\emph{int}}) -- Length of progress bar.

\item {} 
\textbf{maxval} (\href{http://docs.python.org/2/library/functions.html\#int}{\emph{int}}) -- maximal number of iterations

\item {} 
\textbf{current} (\href{http://docs.python.org/2/library/functions.html\#int}{\emph{int}}) -- current number of iterations

\item {} 
\textbf{overrun\_notice} (\href{http://docs.python.org/2/library/functions.html\#bool}{\emph{bool}}) -- if True, overrun above {\color{red}\bfseries{}:ivar:{}`maxval{}`} iterations causes insert of newline

\item {} 
\textbf{overrun} (\href{http://docs.python.org/2/library/functions.html\#bool}{\emph{bool}}) -- flag of overrun

\item {} 
\textbf{begin} (\href{http://docs.python.org/2/library/functions.html\#int}{\emph{int}}) -- time in seconds at the initialization of the {\hyperref[aqueduct.utils.log:aqueduct.utils.log.SimpleProgressBar]{\code{SimpleProgressBar}}} class.

\item {} 
\textbf{tcurrent} (\href{http://docs.python.org/2/library/functions.html\#int}{\emph{int}}) -- time in seconds of current iteration

\end{itemize}

\end{description}\end{quote}
\index{rotate (SimpleProgressBar attribute)}

\begin{fulllineitems}
\phantomsection\label{aqueduct.utils.log:aqueduct.utils.log.SimpleProgressBar.rotate}\pysigline{\bfcode{rotate}\strong{ = `\textbackslash{}\textbackslash{}\textbar{}/-`}}
\end{fulllineitems}

\index{barlenght (SimpleProgressBar attribute)}

\begin{fulllineitems}
\phantomsection\label{aqueduct.utils.log:aqueduct.utils.log.SimpleProgressBar.barlenght}\pysigline{\bfcode{barlenght}\strong{ = 24}}
\end{fulllineitems}

\index{\_\_init\_\_() (SimpleProgressBar method)}

\begin{fulllineitems}
\phantomsection\label{aqueduct.utils.log:aqueduct.utils.log.SimpleProgressBar.__init__}\pysiglinewithargsret{\bfcode{\_\_init\_\_}}{\emph{mess=None}, \emph{maxval=None}}{}~\begin{quote}\begin{description}
\item[{Parameters}] \leavevmode\begin{itemize}
\item {} 
\textbf{maxval} (\href{http://docs.python.org/2/library/functions.html\#int}{\emph{int}}) -- Maximal number of iterations stored to \code{maxval}.

\item {} 
\textbf{mess} (\href{http://docs.python.org/2/library/functions.html\#str}{\emph{str}}) -- Optional message displayed at progress bar initialization.

\end{itemize}

\end{description}\end{quote}

\end{fulllineitems}

\index{bar() (SimpleProgressBar method)}

\begin{fulllineitems}
\phantomsection\label{aqueduct.utils.log:aqueduct.utils.log.SimpleProgressBar.bar}\pysiglinewithargsret{\bfcode{bar}}{}{}
\end{fulllineitems}

\index{ETA() (SimpleProgressBar method)}

\begin{fulllineitems}
\phantomsection\label{aqueduct.utils.log:aqueduct.utils.log.SimpleProgressBar.ETA}\pysiglinewithargsret{\bfcode{ETA}}{}{}
Returns ETA calculated on the basis of current number of iterations
\code{current} and current time \code{tcurrent}. If number of
iterations is 0 returns \code{?}.
Time is formated wiht {\hyperref[aqueduct.utils.log:aqueduct.utils.log.smart_time_string]{\code{smart\_time\_string()}}}.
\begin{quote}\begin{description}
\item[{Returns}] \leavevmode
ETA as string.

\item[{Return type}] \leavevmode
str

\end{description}\end{quote}

\end{fulllineitems}

\index{percent() (SimpleProgressBar method)}

\begin{fulllineitems}
\phantomsection\label{aqueduct.utils.log:aqueduct.utils.log.SimpleProgressBar.percent}\pysiglinewithargsret{\bfcode{percent}}{}{}
Returns float number of precent progress calculated in the basis
of current number of iterations \code{current}. Should return
number between 0 and 100.
\begin{quote}\begin{description}
\item[{Returns}] \leavevmode
percent progress number

\item[{Return type}] \leavevmode
float

\end{description}\end{quote}

\end{fulllineitems}

\index{show() (SimpleProgressBar method)}

\begin{fulllineitems}
\phantomsection\label{aqueduct.utils.log:aqueduct.utils.log.SimpleProgressBar.show}\pysiglinewithargsret{\bfcode{show}}{}{}
Shows current progress.

If value returned by {\hyperref[aqueduct.utils.log:aqueduct.utils.log.SimpleProgressBar.percent]{\code{percent()}}} is =\textless{} 100 then progres is printed as
percent indicator leaded by ETA calculated by {\hyperref[aqueduct.utils.log:aqueduct.utils.log.SimpleProgressBar.ETA]{\code{ETA()}}}.

If value returned by {\hyperref[aqueduct.utils.log:aqueduct.utils.log.SimpleProgressBar.percent]{\code{percent()}}} is \textgreater{} 100 then progress is printed as
number of iterations and total time.

Progress bar is writen to standard error.

\end{fulllineitems}

\index{heartbeat() (SimpleProgressBar method)}

\begin{fulllineitems}
\phantomsection\label{aqueduct.utils.log:aqueduct.utils.log.SimpleProgressBar.heartbeat}\pysiglinewithargsret{\bfcode{heartbeat}}{}{}
\end{fulllineitems}

\index{update() (SimpleProgressBar method)}

\begin{fulllineitems}
\phantomsection\label{aqueduct.utils.log:aqueduct.utils.log.SimpleProgressBar.update}\pysiglinewithargsret{\bfcode{update}}{\emph{step}}{}
Updates number of current iterations \code{current} by one if \code{step} is \textgreater{} 0.
Otherwise number of current iterations is not updated.
In boths cases time of current iteration \code{tcurrent} is updated and
{\hyperref[aqueduct.utils.log:aqueduct.utils.log.SimpleProgressBar.show]{\code{show()}}} is called.
\begin{quote}\begin{description}
\item[{Parameters}] \leavevmode
\textbf{step} (\href{http://docs.python.org/2/library/functions.html\#int}{\emph{int}}) -- update step

\end{description}\end{quote}

\end{fulllineitems}

\index{ttime() (SimpleProgressBar method)}

\begin{fulllineitems}
\phantomsection\label{aqueduct.utils.log:aqueduct.utils.log.SimpleProgressBar.ttime}\pysiglinewithargsret{\bfcode{ttime}}{}{}
Calculates and returns total time string formated with {\hyperref[aqueduct.utils.log:aqueduct.utils.log.smart_time_string]{\code{smart\_time\_string()}}}.
\begin{quote}\begin{description}
\item[{Returns}] \leavevmode
string of total time

\item[{Return type}] \leavevmode
str

\end{description}\end{quote}

\end{fulllineitems}

\index{finish() (SimpleProgressBar method)}

\begin{fulllineitems}
\phantomsection\label{aqueduct.utils.log:aqueduct.utils.log.SimpleProgressBar.finish}\pysiglinewithargsret{\bfcode{finish}}{}{}
Finishes progress bar. First, {\hyperref[aqueduct.utils.log:aqueduct.utils.log.SimpleProgressBar.update]{\code{update()}}} is called with \code{step} = 0. Next message of total time
is writen to standard error.

\end{fulllineitems}


\end{fulllineitems}

\index{pbar (class in aqueduct.utils.log)}

\begin{fulllineitems}
\phantomsection\label{aqueduct.utils.log:aqueduct.utils.log.pbar}\pysiglinewithargsret{\strong{class }\bfcode{pbar}}{\emph{maxval=100}, \emph{kind='simple'}}{}
Bases: \href{http://docs.python.org/2/library/functions.html\#object}{\code{object}}

Progress bar wrapper class.
It can use several types of progress bars, including {\hyperref[aqueduct.utils.log:aqueduct.utils.log.SimpleProgressBar]{\code{SimpleProgressBar}}}. Additionaly, it can
handle progress bars with following packages (must be installed separately):
\begin{itemize}
\item {} 
progressbar

\item {} 
tqdm

\item {} 
pyprind

\end{itemize}
\begin{quote}\begin{description}
\item[{Variables}] \leavevmode\begin{itemize}
\item {} 
\textbf{\_\_maxval} (\href{http://docs.python.org/2/library/functions.html\#int}{\emph{int}}) -- maximal number of iterations

\item {} 
\textbf{\_\_kind} (\href{http://docs.python.org/2/library/functions.html\#str}{\emph{str}}) -- type of progress bar

\item {} 
\textbf{\_\_curval} (\href{http://docs.python.org/2/library/functions.html\#int}{\emph{int}}) -- current number of iterations

\item {} 
\textbf{\_\_pbar} -- progress bar child object

\end{itemize}

\end{description}\end{quote}

\begin{notice}{warning}{Warning:}
{\hyperref[aqueduct.utils.log:aqueduct.utils.log.SimpleProgressBar]{\code{SimpleProgressBar}}} is the only one kind of progress bar recommended.
\end{notice}
\index{\_\_init\_\_() (pbar method)}

\begin{fulllineitems}
\phantomsection\label{aqueduct.utils.log:aqueduct.utils.log.pbar.__init__}\pysiglinewithargsret{\bfcode{\_\_init\_\_}}{\emph{maxval=100}, \emph{kind='simple'}}{}~\begin{quote}\begin{description}
\item[{Parameters}] \leavevmode\begin{itemize}
\item {} 
\textbf{maxval} (\href{http://docs.python.org/2/library/functions.html\#int}{\emph{int}}) -- maximal number of iterations stored to {\color{red}\bfseries{}:ivar:{}`\_\_maxval{}`} and passed child progress bar object

\item {} 
\textbf{kind} (\href{http://docs.python.org/2/library/functions.html\#str}{\emph{str}}) -- type of progress bar, available types: simple, progressbar, tqdm, pyprind

\end{itemize}

\end{description}\end{quote}

\end{fulllineitems}

\index{update() (pbar method)}

\begin{fulllineitems}
\phantomsection\label{aqueduct.utils.log:aqueduct.utils.log.pbar.update}\pysiglinewithargsret{\bfcode{update}}{\emph{val}}{}
Updates progress bar with value of \code{val} parameter. Exact behavior depends
on the type of progress bar.
\begin{quote}\begin{description}
\item[{Parameters}] \leavevmode
\textbf{val} (\href{http://docs.python.org/2/library/functions.html\#int}{\emph{int}}) -- value used to update progress bar

\end{description}\end{quote}

\end{fulllineitems}

\index{finish() (pbar method)}

\begin{fulllineitems}
\phantomsection\label{aqueduct.utils.log:aqueduct.utils.log.pbar.finish}\pysiglinewithargsret{\bfcode{finish}}{}{}
Finishes progress bar. It cals appropriate, if exist, method of child progress bar object.
is writen to standard error.

\end{fulllineitems}


\end{fulllineitems}

\index{get\_str\_timestamp() (in module aqueduct.utils.log)}

\begin{fulllineitems}
\phantomsection\label{aqueduct.utils.log:aqueduct.utils.log.get_str_timestamp}\pysiglinewithargsret{\bfcode{get\_str\_timestamp}}{}{}
\end{fulllineitems}

\index{fbm (class in aqueduct.utils.log)}

\begin{fulllineitems}
\phantomsection\label{aqueduct.utils.log:aqueduct.utils.log.fbm}\pysiglinewithargsret{\strong{class }\bfcode{fbm}}{\emph{info}}{}
Bases: \href{http://docs.python.org/2/library/functions.html\#object}{\code{object}}
\index{\_\_init\_\_() (fbm method)}

\begin{fulllineitems}
\phantomsection\label{aqueduct.utils.log:aqueduct.utils.log.fbm.__init__}\pysiglinewithargsret{\bfcode{\_\_init\_\_}}{\emph{info}}{}
\end{fulllineitems}

\index{\_\_enter\_\_() (fbm method)}

\begin{fulllineitems}
\phantomsection\label{aqueduct.utils.log:aqueduct.utils.log.fbm.__enter__}\pysiglinewithargsret{\bfcode{\_\_enter\_\_}}{}{}
\end{fulllineitems}

\index{\_\_exit\_\_() (fbm method)}

\begin{fulllineitems}
\phantomsection\label{aqueduct.utils.log:aqueduct.utils.log.fbm.__exit__}\pysiglinewithargsret{\bfcode{\_\_exit\_\_}}{\emph{typ}, \emph{value}, \emph{traceback}}{}
\end{fulllineitems}

\index{\_\_call\_\_() (fbm method)}

\begin{fulllineitems}
\phantomsection\label{aqueduct.utils.log:aqueduct.utils.log.fbm.__call__}\pysiglinewithargsret{\bfcode{\_\_call\_\_}}{\emph{info}}{}
\end{fulllineitems}


\end{fulllineitems}

\index{message() (in module aqueduct.utils.log)}

\begin{fulllineitems}
\phantomsection\label{aqueduct.utils.log:aqueduct.utils.log.message}\pysiglinewithargsret{\bfcode{message}}{\emph{mess}, \emph{cont=False}}{}
Prints message to standard error.
\begin{quote}\begin{description}
\item[{Parameters}] \leavevmode\begin{itemize}
\item {} 
\textbf{mess} (\href{http://docs.python.org/2/library/functions.html\#str}{\emph{str}}) -- message to print

\item {} 
\textbf{cont} (\href{http://docs.python.org/2/library/functions.html\#bool}{\emph{bool}}) -- if set True no new line is printed

\end{itemize}

\end{description}\end{quote}

\end{fulllineitems}



\subparagraph{aqueduct.utils.multip module}
\label{aqueduct.utils.multip:module-aqueduct.utils.multip}\label{aqueduct.utils.multip::doc}\label{aqueduct.utils.multip:aqueduct-utils-multip-module}\index{aqueduct.utils.multip (module)}
Created on Feb 3, 2016

@author: tljm


\paragraph{Module contents}
\label{aqueduct.utils:module-aqueduct.utils}\label{aqueduct.utils:module-contents}\index{aqueduct.utils (module)}

\subsubsection{aqueduct.visual package}
\label{aqueduct.visual:aqueduct-visual-package}\label{aqueduct.visual::doc}

\paragraph{Submodules}
\label{aqueduct.visual:submodules}

\subparagraph{aqueduct.visual.pymol\_connector module}
\label{aqueduct.visual.pymol_connector:module-aqueduct.visual.pymol_connector}\label{aqueduct.visual.pymol_connector::doc}\label{aqueduct.visual.pymol_connector:aqueduct-visual-pymol-connector-module}\index{aqueduct.visual.pymol\_connector (module)}\index{BasicPymolCGO (class in aqueduct.visual.pymol\_connector)}

\begin{fulllineitems}
\phantomsection\label{aqueduct.visual.pymol_connector:aqueduct.visual.pymol_connector.BasicPymolCGO}\pysigline{\strong{class }\bfcode{BasicPymolCGO}}
Bases: \href{http://docs.python.org/2/library/functions.html\#object}{\code{object}}
\index{cgo\_entity\_begin (BasicPymolCGO attribute)}

\begin{fulllineitems}
\phantomsection\label{aqueduct.visual.pymol_connector:aqueduct.visual.pymol_connector.BasicPymolCGO.cgo_entity_begin}\pysigline{\bfcode{cgo\_entity\_begin}\strong{ = {[}{]}}}
\end{fulllineitems}

\index{cgo\_entity\_end (BasicPymolCGO attribute)}

\begin{fulllineitems}
\phantomsection\label{aqueduct.visual.pymol_connector:aqueduct.visual.pymol_connector.BasicPymolCGO.cgo_entity_end}\pysigline{\bfcode{cgo\_entity\_end}\strong{ = {[}{]}}}
\end{fulllineitems}

\index{\_\_init\_\_() (BasicPymolCGO method)}

\begin{fulllineitems}
\phantomsection\label{aqueduct.visual.pymol_connector:aqueduct.visual.pymol_connector.BasicPymolCGO.__init__}\pysiglinewithargsret{\bfcode{\_\_init\_\_}}{}{}
\end{fulllineitems}

\index{clean() (BasicPymolCGO method)}

\begin{fulllineitems}
\phantomsection\label{aqueduct.visual.pymol_connector:aqueduct.visual.pymol_connector.BasicPymolCGO.clean}\pysiglinewithargsret{\bfcode{clean}}{}{}
\end{fulllineitems}

\index{new() (BasicPymolCGO method)}

\begin{fulllineitems}
\phantomsection\label{aqueduct.visual.pymol_connector:aqueduct.visual.pymol_connector.BasicPymolCGO.new}\pysiglinewithargsret{\bfcode{new}}{}{}
\end{fulllineitems}

\index{get() (BasicPymolCGO method)}

\begin{fulllineitems}
\phantomsection\label{aqueduct.visual.pymol_connector:aqueduct.visual.pymol_connector.BasicPymolCGO.get}\pysiglinewithargsret{\bfcode{get}}{}{}
\end{fulllineitems}


\end{fulllineitems}

\index{BasicPymolCGOLines (class in aqueduct.visual.pymol\_connector)}

\begin{fulllineitems}
\phantomsection\label{aqueduct.visual.pymol_connector:aqueduct.visual.pymol_connector.BasicPymolCGOLines}\pysigline{\strong{class }\bfcode{BasicPymolCGOLines}}
Bases: {\hyperref[aqueduct.visual.pymol_connector:aqueduct.visual.pymol_connector.BasicPymolCGO]{\code{aqueduct.visual.pymol\_connector.BasicPymolCGO}}}
\index{cgo\_entity\_begin (BasicPymolCGOLines attribute)}

\begin{fulllineitems}
\phantomsection\label{aqueduct.visual.pymol_connector:aqueduct.visual.pymol_connector.BasicPymolCGOLines.cgo_entity_begin}\pysigline{\bfcode{cgo\_entity\_begin}\strong{ = {[}2.0, 1.0{]}}}
\end{fulllineitems}

\index{cgo\_entity\_end (BasicPymolCGOLines attribute)}

\begin{fulllineitems}
\phantomsection\label{aqueduct.visual.pymol_connector:aqueduct.visual.pymol_connector.BasicPymolCGOLines.cgo_entity_end}\pysigline{\bfcode{cgo\_entity\_end}\strong{ = {[}3.0{]}}}
\end{fulllineitems}

\index{add() (BasicPymolCGOLines method)}

\begin{fulllineitems}
\phantomsection\label{aqueduct.visual.pymol_connector:aqueduct.visual.pymol_connector.BasicPymolCGOLines.add}\pysiglinewithargsret{\bfcode{add}}{\emph{coords=None}, \emph{color=None}}{}
\end{fulllineitems}


\end{fulllineitems}

\index{BasicPymolCGOSpheres (class in aqueduct.visual.pymol\_connector)}

\begin{fulllineitems}
\phantomsection\label{aqueduct.visual.pymol_connector:aqueduct.visual.pymol_connector.BasicPymolCGOSpheres}\pysigline{\strong{class }\bfcode{BasicPymolCGOSpheres}}
Bases: {\hyperref[aqueduct.visual.pymol_connector:aqueduct.visual.pymol_connector.BasicPymolCGO]{\code{aqueduct.visual.pymol\_connector.BasicPymolCGO}}}
\index{cgo\_entity\_begin (BasicPymolCGOSpheres attribute)}

\begin{fulllineitems}
\phantomsection\label{aqueduct.visual.pymol_connector:aqueduct.visual.pymol_connector.BasicPymolCGOSpheres.cgo_entity_begin}\pysigline{\bfcode{cgo\_entity\_begin}\strong{ = {[}{]}}}
\end{fulllineitems}

\index{cgo\_entity\_end (BasicPymolCGOSpheres attribute)}

\begin{fulllineitems}
\phantomsection\label{aqueduct.visual.pymol_connector:aqueduct.visual.pymol_connector.BasicPymolCGOSpheres.cgo_entity_end}\pysigline{\bfcode{cgo\_entity\_end}\strong{ = {[}{]}}}
\end{fulllineitems}

\index{add() (BasicPymolCGOSpheres method)}

\begin{fulllineitems}
\phantomsection\label{aqueduct.visual.pymol_connector:aqueduct.visual.pymol_connector.BasicPymolCGOSpheres.add}\pysiglinewithargsret{\bfcode{add}}{\emph{coords=None}, \emph{radius=None}, \emph{color=None}}{}
\end{fulllineitems}


\end{fulllineitems}

\index{BasicPymolCGOPointers (class in aqueduct.visual.pymol\_connector)}

\begin{fulllineitems}
\phantomsection\label{aqueduct.visual.pymol_connector:aqueduct.visual.pymol_connector.BasicPymolCGOPointers}\pysigline{\strong{class }\bfcode{BasicPymolCGOPointers}}
Bases: {\hyperref[aqueduct.visual.pymol_connector:aqueduct.visual.pymol_connector.BasicPymolCGO]{\code{aqueduct.visual.pymol\_connector.BasicPymolCGO}}}
\index{cgo\_entity\_begin (BasicPymolCGOPointers attribute)}

\begin{fulllineitems}
\phantomsection\label{aqueduct.visual.pymol_connector:aqueduct.visual.pymol_connector.BasicPymolCGOPointers.cgo_entity_begin}\pysigline{\bfcode{cgo\_entity\_begin}\strong{ = {[}{]}}}
\end{fulllineitems}

\index{cgo\_entity\_end (BasicPymolCGOPointers attribute)}

\begin{fulllineitems}
\phantomsection\label{aqueduct.visual.pymol_connector:aqueduct.visual.pymol_connector.BasicPymolCGOPointers.cgo_entity_end}\pysigline{\bfcode{cgo\_entity\_end}\strong{ = {[}{]}}}
\end{fulllineitems}

\index{add\_cone() (BasicPymolCGOPointers method)}

\begin{fulllineitems}
\phantomsection\label{aqueduct.visual.pymol_connector:aqueduct.visual.pymol_connector.BasicPymolCGOPointers.add_cone}\pysiglinewithargsret{\bfcode{add\_cone}}{\emph{coords1=None}, \emph{coords2=None}, \emph{radius1=None}, \emph{radius2=None}, \emph{color1=None}, \emph{color2=None}}{}
\end{fulllineitems}

\index{add\_pointer() (BasicPymolCGOPointers method)}

\begin{fulllineitems}
\phantomsection\label{aqueduct.visual.pymol_connector:aqueduct.visual.pymol_connector.BasicPymolCGOPointers.add_pointer}\pysiglinewithargsret{\bfcode{add\_pointer}}{\emph{point=None}, \emph{direction=None}, \emph{length=None}, \emph{color=None}, \emph{reverse=False}}{}
\end{fulllineitems}


\end{fulllineitems}

\index{ConnectToPymol (class in aqueduct.visual.pymol\_connector)}

\begin{fulllineitems}
\phantomsection\label{aqueduct.visual.pymol_connector:aqueduct.visual.pymol_connector.ConnectToPymol}\pysigline{\strong{class }\bfcode{ConnectToPymol}}
Bases: \href{http://docs.python.org/2/library/functions.html\#object}{\code{object}}
\index{cgo\_line\_width (ConnectToPymol attribute)}

\begin{fulllineitems}
\phantomsection\label{aqueduct.visual.pymol_connector:aqueduct.visual.pymol_connector.ConnectToPymol.cgo_line_width}\pysigline{\bfcode{cgo\_line\_width}\strong{ = 2.0}}
\end{fulllineitems}

\index{init\_pymol() (ConnectToPymol static method)}

\begin{fulllineitems}
\phantomsection\label{aqueduct.visual.pymol_connector:aqueduct.visual.pymol_connector.ConnectToPymol.init_pymol}\pysiglinewithargsret{\strong{static }\bfcode{init\_pymol}}{}{}
\end{fulllineitems}

\index{add\_cgo\_object() (ConnectToPymol static method)}

\begin{fulllineitems}
\phantomsection\label{aqueduct.visual.pymol_connector:aqueduct.visual.pymol_connector.ConnectToPymol.add_cgo_object}\pysiglinewithargsret{\strong{static }\bfcode{add\_cgo\_object}}{\emph{name}, \emph{cgo\_object}, \emph{state=None}}{}
\end{fulllineitems}

\index{del\_cgo\_object() (ConnectToPymol static method)}

\begin{fulllineitems}
\phantomsection\label{aqueduct.visual.pymol_connector:aqueduct.visual.pymol_connector.ConnectToPymol.del_cgo_object}\pysiglinewithargsret{\strong{static }\bfcode{del\_cgo\_object}}{\emph{name}, \emph{state=None}}{}
\end{fulllineitems}

\index{load\_pdb() (ConnectToPymol static method)}

\begin{fulllineitems}
\phantomsection\label{aqueduct.visual.pymol_connector:aqueduct.visual.pymol_connector.ConnectToPymol.load_pdb}\pysiglinewithargsret{\strong{static }\bfcode{load\_pdb}}{\emph{name}, \emph{filename}, \emph{state=None}}{}
\end{fulllineitems}


\end{fulllineitems}

\index{SinglePathPlotter (class in aqueduct.visual.pymol\_connector)}

\begin{fulllineitems}
\phantomsection\label{aqueduct.visual.pymol_connector:aqueduct.visual.pymol_connector.SinglePathPlotter}\pysiglinewithargsret{\strong{class }\bfcode{SinglePathPlotter}}{\emph{linearize=None}}{}
Bases: \href{http://docs.python.org/2/library/functions.html\#object}{\code{object}}
\index{\_\_init\_\_() (SinglePathPlotter method)}

\begin{fulllineitems}
\phantomsection\label{aqueduct.visual.pymol_connector:aqueduct.visual.pymol_connector.SinglePathPlotter.__init__}\pysiglinewithargsret{\bfcode{\_\_init\_\_}}{\emph{linearize=None}}{}
\end{fulllineitems}

\index{add\_single\_path\_continous\_trace() (SinglePathPlotter method)}

\begin{fulllineitems}
\phantomsection\label{aqueduct.visual.pymol_connector:aqueduct.visual.pymol_connector.SinglePathPlotter.add_single_path_continous_trace}\pysiglinewithargsret{\bfcode{add\_single\_path\_continous\_trace}}{\emph{spath}, \emph{smooth=None}, \emph{plot\_in=True}, \emph{plot\_object=True}, \emph{plot\_out=True}, \emph{**kwargs}}{}
\end{fulllineitems}

\index{paths\_trace() (SinglePathPlotter method)}

\begin{fulllineitems}
\phantomsection\label{aqueduct.visual.pymol_connector:aqueduct.visual.pymol_connector.SinglePathPlotter.paths_trace}\pysiglinewithargsret{\bfcode{paths\_trace}}{\emph{spaths}, \emph{smooth=None}, \emph{name='paths'}, \emph{state=None}, \emph{**kwargs}}{}
\end{fulllineitems}

\index{paths\_inlets() (SinglePathPlotter method)}

\begin{fulllineitems}
\phantomsection\label{aqueduct.visual.pymol_connector:aqueduct.visual.pymol_connector.SinglePathPlotter.paths_inlets}\pysiglinewithargsret{\bfcode{paths\_inlets}}{\emph{spaths}, \emph{smooth=None}, \emph{color=None}, \emph{plot\_in=True}, \emph{plot\_out=True}, \emph{name='in-out-let'}, \emph{state=None}, \emph{**kwargs}}{}
\end{fulllineitems}

\index{scatter() (SinglePathPlotter method)}

\begin{fulllineitems}
\phantomsection\label{aqueduct.visual.pymol_connector:aqueduct.visual.pymol_connector.SinglePathPlotter.scatter}\pysiglinewithargsret{\bfcode{scatter}}{\emph{coords}, \emph{radius=0.4}, \emph{color='r'}, \emph{name='scatter'}, \emph{state=None}}{}
\end{fulllineitems}


\end{fulllineitems}



\subparagraph{aqueduct.visual.quickplot module}
\label{aqueduct.visual.quickplot:module-aqueduct.visual.quickplot}\label{aqueduct.visual.quickplot::doc}\label{aqueduct.visual.quickplot:aqueduct-visual-quickplot-module}\index{aqueduct.visual.quickplot (module)}\index{cc() (in module aqueduct.visual.quickplot)}

\begin{fulllineitems}
\phantomsection\label{aqueduct.visual.quickplot:aqueduct.visual.quickplot.cc}\pysiglinewithargsret{\bfcode{cc}}{\emph{c}, \emph{alpha=1.0}}{}
\end{fulllineitems}

\index{get\_cmap() (in module aqueduct.visual.quickplot)}

\begin{fulllineitems}
\phantomsection\label{aqueduct.visual.quickplot:aqueduct.visual.quickplot.get_cmap}\pysiglinewithargsret{\bfcode{get\_cmap}}{\emph{name}, \emph{size=None}}{}
\end{fulllineitems}

\index{ColorMapDistMap (class in aqueduct.visual.quickplot)}

\begin{fulllineitems}
\phantomsection\label{aqueduct.visual.quickplot:aqueduct.visual.quickplot.ColorMapDistMap}\pysiglinewithargsret{\strong{class }\bfcode{ColorMapDistMap}}{\emph{name='hsv'}, \emph{size=None}}{}
Bases: \href{http://docs.python.org/2/library/functions.html\#object}{\code{object}}
\index{default\_cm\_size (ColorMapDistMap attribute)}

\begin{fulllineitems}
\phantomsection\label{aqueduct.visual.quickplot:aqueduct.visual.quickplot.ColorMapDistMap.default_cm_size}\pysigline{\bfcode{default\_cm\_size}\strong{ = 256}}
\end{fulllineitems}

\index{grey (ColorMapDistMap attribute)}

\begin{fulllineitems}
\phantomsection\label{aqueduct.visual.quickplot:aqueduct.visual.quickplot.ColorMapDistMap.grey}\pysigline{\bfcode{grey}\strong{ = (0.5, 0.5, 0.5, 1)}}
\end{fulllineitems}

\index{\_\_init\_\_() (ColorMapDistMap method)}

\begin{fulllineitems}
\phantomsection\label{aqueduct.visual.quickplot:aqueduct.visual.quickplot.ColorMapDistMap.__init__}\pysiglinewithargsret{\bfcode{\_\_init\_\_}}{\emph{name='hsv'}, \emph{size=None}}{}
\end{fulllineitems}

\index{\_\_call\_\_() (ColorMapDistMap method)}

\begin{fulllineitems}
\phantomsection\label{aqueduct.visual.quickplot:aqueduct.visual.quickplot.ColorMapDistMap.__call__}\pysiglinewithargsret{\bfcode{\_\_call\_\_}}{\emph{node}}{}
\end{fulllineitems}


\end{fulllineitems}

\index{f\_like() (in module aqueduct.visual.quickplot)}

\begin{fulllineitems}
\phantomsection\label{aqueduct.visual.quickplot:aqueduct.visual.quickplot.f_like}\pysiglinewithargsret{\bfcode{f\_like}}{\emph{n}}{}
\end{fulllineitems}

\index{yield\_spath\_len\_and\_smooth\_diff\_in\_types\_slices() (in module aqueduct.visual.quickplot)}

\begin{fulllineitems}
\phantomsection\label{aqueduct.visual.quickplot:aqueduct.visual.quickplot.yield_spath_len_and_smooth_diff_in_types_slices}\pysiglinewithargsret{\bfcode{yield\_spath\_len\_and\_smooth\_diff\_in\_types\_slices}}{\emph{sp}, \emph{smooth=None}, \emph{smooth\_len=None}, \emph{smooth\_diff=None}, \emph{types='etypes'}}{}
\end{fulllineitems}

\index{color\_codes() (in module aqueduct.visual.quickplot)}

\begin{fulllineitems}
\phantomsection\label{aqueduct.visual.quickplot:aqueduct.visual.quickplot.color_codes}\pysiglinewithargsret{\bfcode{color\_codes}}{\emph{code}, \emph{custom\_codes=None}}{}
\end{fulllineitems}

\index{plot\_colorful\_lines() (in module aqueduct.visual.quickplot)}

\begin{fulllineitems}
\phantomsection\label{aqueduct.visual.quickplot:aqueduct.visual.quickplot.plot_colorful_lines}\pysiglinewithargsret{\bfcode{plot\_colorful\_lines}}{\emph{x}, \emph{y}, \emph{c}, \emph{**kwargs}}{}
\end{fulllineitems}

\index{spaths\_spectra() (in module aqueduct.visual.quickplot)}

\begin{fulllineitems}
\phantomsection\label{aqueduct.visual.quickplot:aqueduct.visual.quickplot.spaths_spectra}\pysiglinewithargsret{\bfcode{spaths\_spectra}}{\emph{spaths}, \emph{**kwargs}}{}
\end{fulllineitems}

\index{plot\_spath\_spectrum() (in module aqueduct.visual.quickplot)}

\begin{fulllineitems}
\phantomsection\label{aqueduct.visual.quickplot:aqueduct.visual.quickplot.plot_spath_spectrum}\pysiglinewithargsret{\bfcode{plot\_spath\_spectrum}}{\emph{sp}, \emph{**kwargs}}{}
\end{fulllineitems}

\index{spath\_spectrum() (in module aqueduct.visual.quickplot)}

\begin{fulllineitems}
\phantomsection\label{aqueduct.visual.quickplot:aqueduct.visual.quickplot.spath_spectrum}\pysiglinewithargsret{\bfcode{spath\_spectrum}}{\emph{sp}, \emph{**kwargs}}{}
\end{fulllineitems}

\index{showit() (in module aqueduct.visual.quickplot)}

\begin{fulllineitems}
\phantomsection\label{aqueduct.visual.quickplot:aqueduct.visual.quickplot.showit}\pysiglinewithargsret{\bfcode{showit}}{\emph{gen}}{}
\end{fulllineitems}

\index{get\_ax3d() (in module aqueduct.visual.quickplot)}

\begin{fulllineitems}
\phantomsection\label{aqueduct.visual.quickplot:aqueduct.visual.quickplot.get_ax3d}\pysiglinewithargsret{\bfcode{get\_ax3d}}{\emph{fig}, \emph{sub=111}}{}
\end{fulllineitems}

\index{SimpleTracePlotter (class in aqueduct.visual.quickplot)}

\begin{fulllineitems}
\phantomsection\label{aqueduct.visual.quickplot:aqueduct.visual.quickplot.SimpleTracePlotter}\pysigline{\strong{class }\bfcode{SimpleTracePlotter}}
Bases: \href{http://docs.python.org/2/library/functions.html\#object}{\code{object}}
\index{plot\_line() (SimpleTracePlotter method)}

\begin{fulllineitems}
\phantomsection\label{aqueduct.visual.quickplot:aqueduct.visual.quickplot.SimpleTracePlotter.plot_line}\pysiglinewithargsret{\bfcode{plot\_line}}{\emph{coords}, \emph{color}, \emph{**kwargs}}{}
\end{fulllineitems}

\index{single\_trace() (SimpleTracePlotter method)}

\begin{fulllineitems}
\phantomsection\label{aqueduct.visual.quickplot:aqueduct.visual.quickplot.SimpleTracePlotter.single_trace}\pysiglinewithargsret{\bfcode{single\_trace}}{\emph{coords}, \emph{color='r'}, \emph{**kwargs}}{}
\end{fulllineitems}

\index{path\_trace() (SimpleTracePlotter method)}

\begin{fulllineitems}
\phantomsection\label{aqueduct.visual.quickplot:aqueduct.visual.quickplot.SimpleTracePlotter.path_trace}\pysiglinewithargsret{\bfcode{path\_trace}}{\emph{path}, \emph{color=(`r'}, \emph{`g'}, \emph{`b')}, \emph{plot\_in=True}, \emph{plot\_object=True}, \emph{plot\_out=True}, \emph{**kwargs}}{}
\end{fulllineitems}


\end{fulllineitems}

\index{SimpleProteinPlotter (class in aqueduct.visual.quickplot)}

\begin{fulllineitems}
\phantomsection\label{aqueduct.visual.quickplot:aqueduct.visual.quickplot.SimpleProteinPlotter}\pysigline{\strong{class }\bfcode{SimpleProteinPlotter}}
Bases: {\hyperref[aqueduct.visual.quickplot:aqueduct.visual.quickplot.SimpleTracePlotter]{\code{aqueduct.visual.quickplot.SimpleTracePlotter}}}
\index{protein\_trace() (SimpleProteinPlotter method)}

\begin{fulllineitems}
\phantomsection\label{aqueduct.visual.quickplot:aqueduct.visual.quickplot.SimpleProteinPlotter.protein_trace}\pysiglinewithargsret{\bfcode{protein\_trace}}{\emph{protein}, \emph{smooth=None}, \emph{color=(`c'}, \emph{`m'}, \emph{`y')}, \emph{**kwargs}}{}
\end{fulllineitems}


\end{fulllineitems}

\index{SimplePathPlotter (class in aqueduct.visual.quickplot)}

\begin{fulllineitems}
\phantomsection\label{aqueduct.visual.quickplot:aqueduct.visual.quickplot.SimplePathPlotter}\pysigline{\strong{class }\bfcode{SimplePathPlotter}}
Bases: {\hyperref[aqueduct.visual.quickplot:aqueduct.visual.quickplot.SimpleTracePlotter]{\code{aqueduct.visual.quickplot.SimpleTracePlotter}}}
\index{single\_path\_traces() (SimplePathPlotter method)}

\begin{fulllineitems}
\phantomsection\label{aqueduct.visual.quickplot:aqueduct.visual.quickplot.SimplePathPlotter.single_path_traces}\pysiglinewithargsret{\bfcode{single\_path\_traces}}{\emph{spaths}, \emph{smooth=None}, \emph{color=(`r'}, \emph{`g'}, \emph{`b')}, \emph{**kwargs}}{}
\end{fulllineitems}


\end{fulllineitems}

\index{MPLTracePlotter (class in aqueduct.visual.quickplot)}

\begin{fulllineitems}
\phantomsection\label{aqueduct.visual.quickplot:aqueduct.visual.quickplot.MPLTracePlotter}\pysigline{\strong{class }\bfcode{MPLTracePlotter}}
Bases: {\hyperref[aqueduct.visual.quickplot:aqueduct.visual.quickplot.SimplePathPlotter]{\code{aqueduct.visual.quickplot.SimplePathPlotter}}}, {\hyperref[aqueduct.visual.quickplot:aqueduct.visual.quickplot.SimpleProteinPlotter]{\code{aqueduct.visual.quickplot.SimpleProteinPlotter}}}
\index{init\_ax() (MPLTracePlotter method)}

\begin{fulllineitems}
\phantomsection\label{aqueduct.visual.quickplot:aqueduct.visual.quickplot.MPLTracePlotter.init_ax}\pysiglinewithargsret{\bfcode{init\_ax}}{\emph{*args}, \emph{**kwargs}}{}
\end{fulllineitems}

\index{plot\_line() (MPLTracePlotter method)}

\begin{fulllineitems}
\phantomsection\label{aqueduct.visual.quickplot:aqueduct.visual.quickplot.MPLTracePlotter.plot_line}\pysiglinewithargsret{\bfcode{plot\_line}}{\emph{*args}, \emph{**kwargs}}{}
\end{fulllineitems}

\index{scatter() (MPLTracePlotter method)}

\begin{fulllineitems}
\phantomsection\label{aqueduct.visual.quickplot:aqueduct.visual.quickplot.MPLTracePlotter.scatter}\pysiglinewithargsret{\bfcode{scatter}}{\emph{*args}, \emph{**kwargs}}{}
\end{fulllineitems}


\end{fulllineitems}



\paragraph{Module contents}
\label{aqueduct.visual:module-aqueduct.visual}\label{aqueduct.visual:module-contents}\index{aqueduct.visual (module)}

\subsection{Module contents}
\label{aqueduct:module-aqueduct}\label{aqueduct:module-contents}\index{aqueduct (module)}
Aqueduct - a collection of tools to trace residues in MD simulation.
\index{version() (in module aqueduct)}

\begin{fulllineitems}
\phantomsection\label{aqueduct:aqueduct.version}\pysiglinewithargsret{\bfcode{version}}{}{}
Returns {\hyperref[aqueduct:module-aqueduct]{\code{aqueduct}}} version number.
\begin{quote}\begin{description}
\item[{Returns}] \leavevmode
3 element tuple of int numbers

\item[{Return type}] \leavevmode
tuple

\end{description}\end{quote}

\end{fulllineitems}

\index{version\_nice() (in module aqueduct)}

\begin{fulllineitems}
\phantomsection\label{aqueduct:aqueduct.version_nice}\pysiglinewithargsret{\bfcode{version\_nice}}{}{}
Returns {\hyperref[aqueduct:module-aqueduct]{\code{aqueduct}}} version number as nicely formated string.
\begin{quote}\begin{description}
\item[{Returns}] \leavevmode
string composed on the basis of the number returned by {\hyperref[aqueduct:aqueduct.version]{\code{version()}}}.

\item[{Return type}] \leavevmode
str

\end{description}\end{quote}

\end{fulllineitems}

\index{greetings() (in module aqueduct)}

\begin{fulllineitems}
\phantomsection\label{aqueduct:aqueduct.greetings}\pysiglinewithargsret{\bfcode{greetings}}{}{}
Returns fancy greetings of {\hyperref[aqueduct:module-aqueduct]{\code{aqueduct}}}. It has a form of ASCII-like
graphic. Currently it returns following string:

\begin{Verbatim}[commandchars=\\\{\}]
\PYGZhy{}\PYGZhy{}\PYGZhy{}\PYGZhy{}\PYGZhy{}\PYGZhy{}\PYGZhy{}\PYGZhy{}\PYGZhy{}\PYGZhy{}\PYGZhy{}\PYGZhy{}\PYGZhy{}\PYGZhy{}\PYGZhy{}\PYGZhy{}\PYGZhy{}\PYGZhy{}\PYGZhy{}\PYGZhy{}\PYGZhy{}\PYGZhy{}\PYGZhy{}\PYGZhy{}\PYGZhy{}\PYGZhy{}\PYGZhy{}\PYGZhy{}\PYGZhy{}\PYGZhy{}\PYGZhy{}\PYGZhy{}\PYGZhy{}\PYGZhy{}\PYGZhy{}\PYGZhy{}\PYGZhy{}\PYGZhy{}\PYGZhy{}\PYGZhy{}\PYGZhy{}\PYGZhy{}\PYGZhy{}\PYGZhy{}\PYGZhy{}\PYGZhy{}\PYGZhy{}\PYGZhy{}
           \PYGZti{} \PYGZti{} \PYGZti{} A Q U E D U C T \PYGZti{} \PYGZti{} \PYGZti{}
\PYGZsh{}\PYGZsh{}\PYGZsh{}\PYGZsh{}\PYGZsh{}\PYGZsh{}\PYGZsh{}\PYGZsh{}\PYGZsh{}\PYGZsh{}\PYGZsh{}\PYGZsh{}\PYGZsh{}\PYGZsh{}\PYGZsh{}\PYGZsh{}\PYGZsh{}\PYGZsh{}\PYGZsh{}\PYGZsh{}\PYGZsh{}\PYGZsh{}\PYGZsh{}\PYGZsh{}\PYGZsh{}\PYGZsh{}\PYGZsh{}\PYGZsh{}\PYGZsh{}\PYGZsh{}\PYGZsh{}\PYGZsh{}\PYGZsh{}\PYGZsh{}\PYGZsh{}\PYGZsh{}\PYGZsh{}\PYGZsh{}\PYGZsh{}\PYGZsh{}\PYGZsh{}\PYGZsh{}\PYGZsh{}\PYGZsh{}\PYGZsh{}\PYGZsh{}\PYGZsh{}\PYGZsh{}
\PYGZsh{}\PYGZsh{}\PYGZsh{}\PYGZsh{}        \PYGZsh{}\PYGZsh{}\PYGZsh{}\PYGZsh{}\PYGZsh{}\PYGZsh{}\PYGZsh{}\PYGZsh{}        \PYGZsh{}\PYGZsh{}\PYGZsh{}\PYGZsh{}\PYGZsh{}\PYGZsh{}\PYGZsh{}\PYGZsh{}        \PYGZsh{}\PYGZsh{}\PYGZsh{}\PYGZsh{}
\PYGZsh{}\PYGZsh{}            \PYGZsh{}\PYGZsh{}\PYGZsh{}\PYGZsh{}            \PYGZsh{}\PYGZsh{}\PYGZsh{}\PYGZsh{}            \PYGZsh{}\PYGZsh{}
\PYGZsh{}              \PYGZsh{}\PYGZsh{}              \PYGZsh{}\PYGZsh{}              \PYGZsh{}
\PYGZsh{}              \PYGZsh{}\PYGZsh{}              \PYGZsh{}\PYGZsh{}              \PYGZsh{}
\PYGZsh{}              \PYGZsh{}\PYGZsh{}              \PYGZsh{}\PYGZsh{}              \PYGZsh{}
\PYGZsh{}              \PYGZsh{}\PYGZsh{}              \PYGZsh{}\PYGZsh{}              \PYGZsh{}
\PYGZhy{}\PYGZhy{}\PYGZhy{}\PYGZhy{}\PYGZhy{}\PYGZhy{}\PYGZhy{}\PYGZhy{}\PYGZhy{}\PYGZhy{}\PYGZhy{}\PYGZhy{}\PYGZhy{}\PYGZhy{}\PYGZhy{}\PYGZhy{}\PYGZhy{}\PYGZhy{}\PYGZhy{}\PYGZhy{}\PYGZhy{}\PYGZhy{}\PYGZhy{}\PYGZhy{}\PYGZhy{}\PYGZhy{}\PYGZhy{}\PYGZhy{}\PYGZhy{}\PYGZhy{}\PYGZhy{}\PYGZhy{}\PYGZhy{}\PYGZhy{}\PYGZhy{}\PYGZhy{}\PYGZhy{}\PYGZhy{}\PYGZhy{}\PYGZhy{}\PYGZhy{}\PYGZhy{}\PYGZhy{}\PYGZhy{}\PYGZhy{}\PYGZhy{}\PYGZhy{}\PYGZhy{}
\end{Verbatim}
\begin{quote}\begin{description}
\item[{Returns}] \leavevmode
{\hyperref[aqueduct:module-aqueduct]{\code{aqueduct}}} fancy greetings.

\item[{Return type}] \leavevmode
str

\end{description}\end{quote}

\end{fulllineitems}



\chapter{Indices and tables}
\label{index:indices-and-tables}\begin{itemize}
\item {} 
\emph{genindex}

\item {} 
\emph{modindex}

\item {} 
\emph{search}

\end{itemize}


\renewcommand{\indexname}{Python Module Index}
\begin{theindex}
\def\bigletter#1{{\Large\sffamily#1}\nopagebreak\vspace{1mm}}
\bigletter{a}
\item {\texttt{aqueduct}}, \pageref{aqueduct:module-aqueduct}
\item {\texttt{aqueduct.geom}}, \pageref{aqueduct.geom:module-aqueduct.geom}
\item {\texttt{aqueduct.geom.cluster}}, \pageref{aqueduct.geom.cluster:module-aqueduct.geom.cluster}
\item {\texttt{aqueduct.geom.convexhull}}, \pageref{aqueduct.geom.convexhull:module-aqueduct.geom.convexhull}
\item {\texttt{aqueduct.geom.convexhull\_pyhull}}, \pageref{aqueduct.geom.convexhull_pyhull:module-aqueduct.geom.convexhull_pyhull}
\item {\texttt{aqueduct.geom.convexhull\_scipy}}, \pageref{aqueduct.geom.convexhull_scipy:module-aqueduct.geom.convexhull_scipy}
\item {\texttt{aqueduct.geom.master}}, \pageref{aqueduct.geom.master:module-aqueduct.geom.master}
\item {\texttt{aqueduct.geom.pca}}, \pageref{aqueduct.geom.pca:module-aqueduct.geom.pca}
\item {\texttt{aqueduct.geom.smooth}}, \pageref{aqueduct.geom.smooth:module-aqueduct.geom.smooth}
\item {\texttt{aqueduct.geom.traces}}, \pageref{aqueduct.geom.traces:module-aqueduct.geom.traces}
\item {\texttt{aqueduct.traj}}, \pageref{aqueduct.traj:module-aqueduct.traj}
\item {\texttt{aqueduct.traj.dumps}}, \pageref{aqueduct.traj.dumps:module-aqueduct.traj.dumps}
\item {\texttt{aqueduct.traj.inlets}}, \pageref{aqueduct.traj.inlets:module-aqueduct.traj.inlets}
\item {\texttt{aqueduct.traj.paths}}, \pageref{aqueduct.traj.paths:module-aqueduct.traj.paths}
\item {\texttt{aqueduct.traj.reader}}, \pageref{aqueduct.traj.reader:module-aqueduct.traj.reader}
\item {\texttt{aqueduct.traj.selections}}, \pageref{aqueduct.traj.selections:module-aqueduct.traj.selections}
\item {\texttt{aqueduct.utils}}, \pageref{aqueduct.utils:module-aqueduct.utils}
\item {\texttt{aqueduct.utils.helpers}}, \pageref{aqueduct.utils.helpers:module-aqueduct.utils.helpers}
\item {\texttt{aqueduct.utils.log}}, \pageref{aqueduct.utils.log:module-aqueduct.utils.log}
\item {\texttt{aqueduct.utils.multip}}, \pageref{aqueduct.utils.multip:module-aqueduct.utils.multip}
\item {\texttt{aqueduct.visual}}, \pageref{aqueduct.visual:module-aqueduct.visual}
\item {\texttt{aqueduct.visual.pymol\_connector}}, \pageref{aqueduct.visual.pymol_connector:module-aqueduct.visual.pymol_connector}
\item {\texttt{aqueduct.visual.quickplot}}, \pageref{aqueduct.visual.quickplot:module-aqueduct.visual.quickplot}
\end{theindex}

\renewcommand{\indexname}{Index}
\printindex
\end{document}
